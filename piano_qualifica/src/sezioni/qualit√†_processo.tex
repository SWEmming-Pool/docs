\section{Qualità di processo}
La qualità del prodotto finale dipende dall'efficienza e dalla qualità dei
processi che lo compongono. Per assicurare che questi ultimi soddisfino gli
standard di qualità stabiliti, è fondamentale utilizzare metriche di valutazione
adeguate. Questa sezione espone i valori di qualità accettabili e ottimali
sulla base di metriche elencate nelle \textit{Norme di Progetto v1.0.0}. È importante monitorare costantemente queste
metriche per garantire che i processi raggiungano gli obiettivi di qualità
prefissati e che il prodotto finale sia di alta qualità.

\subsection{Obbiettivi metriche}
\subsubsection{Processi primari}
\renewcommand{\arraystretch}{1.8}
\begin{xltabular}{\textwidth} {
        >{\hsize=0.6\hsize\linewidth=\hsize}X
        >{\hsize=1.40\hsize\linewidth=\hsize}X
        >{\hsize=1.00\hsize\linewidth=\hsize}X
        >{\hsize=1.00\hsize\linewidth=\hsize}X
    }
    \rowcolorhead
    \textbf{\color{white}Codice} &
    \textbf{\color{white}Nome metrica} &
    \textbf{\color{white}Valore accettabile} &
    \textbf{\color{white}Valore ottimo} \\
    \hline
    \endfirsthead

    \hline
    \rowcolorhead
    \textbf{\color{white}Codice} &
    \textbf{\color{white}Nome metrica} &
    \textbf{\color{white}Valore accettabile} &
    \textbf{\color{white}Valore ottimo} \\
    \hline
    \endhead

    \endfoot

    \endlastfoot

    \rowcolor{cyan}
    \multicolumn{4}{c}{\textbf{Fornitura}}\\
    \hline

    MPC01 &
    (EV) Earned Value   &
    $\geq 0$ &
    $\leq EAC$
    \\ \hline

    MPC02 &
    (AC) Actual Cost   &
    $\geq 0$ &
    $\leq EAC$
    \\ \hline

    MPC03 &
    (PV) Planned Value   &
    $\geq 0$ &
    $\leq Budget at
        Completion$
    \\ \hline

    MPC04 &
    (CV) Cost Variance   &
    $\geq -10\%$ &
    $\geq 0\%$
    \\ \hline

    MPC05&
    (SV) Schedule Variance  &
    $\geq -10\%$ &
    $\leq 0\%$
    \\ \hline

    MPC06 &
    (EAC) Estimated at Completion   &
    preventivo - 3\% $\leq$ EAC $\leq$ preventivo + 3\% &
    EAC = preventivo%$
    \\ \hline

    MPC07 &
    (ETC) Estimate to Complete   &
    $\geq 0$ &
    $\leq EAC$
    \\ \hline

    \rowcolor{cyan}
    \multicolumn{4}{c}{\textbf{Sviluppo}}\\

    MPC08 &
    (RSI) Requirements stability index  &
    $ 70\%$ &
    $ 100\%$
    \\ \hline

    \rowcolor{white}
    \caption{Obbiettivi metriche dei processi primari}
\end{xltabular}
\pagebreak
\subsubsection{Processi di supporto}

\begin{xltabular}{\textwidth} {
        >{\hsize=0.6\hsize\linewidth=\hsize}X
        >{\hsize=1.40\hsize\linewidth=\hsize}X
        >{\hsize=1.00\hsize\linewidth=\hsize}X
        >{\hsize=1.00\hsize\linewidth=\hsize}X
    }
    \rowcolorhead
    \textbf{\color{white}Codice} &
    \textbf{\color{white}Nome metrica} &
    \textbf{\color{white}Valore accettabile} &
    \textbf{\color{white}Valore ottimo} \\
    \hline
    \endfirsthead

    \hline
    \rowcolorhead
    \textbf{\color{white}Codice} &
    \textbf{\color{white}Nome metrica} &
    \textbf{\color{white}Valore accettabile} &
    \textbf{\color{white}Valore ottimo} \\
    \hline
    \endhead

    \endfoot

    \endlastfoot

    \rowcolor{cyan}
    \multicolumn{4}{c}{\textbf{Verifica}}\\

    MPC09 &
    (CC) Code coverage  &
    $\geq 80\%$ &
    $100\%$
    \\ \hline
    \rowcolor{cyan}
    \multicolumn{4}{c}{\textbf{Gestione della qualità}}\\

    MPC10 &
    (CC) Metrics satisfied&
    $\geq 80\%$ &
    $100\%$
    \\ \hline

    \rowcolor{white}
    \caption{Obbiettivi metriche dei processi di supporto}
\end{xltabular}

\subsection{Processi organizzativi}
\subsubsection{Processi di supporto}

\begin{xltabular}{\textwidth} {
        >{\hsize=0.6\hsize\linewidth=\hsize}X
        >{\hsize=1.40\hsize\linewidth=\hsize}X
        >{\hsize=1.00\hsize\linewidth=\hsize}X
        >{\hsize=1.00\hsize\linewidth=\hsize}X
    }
    \rowcolorhead
    \textbf{\color{white}Codice} &
    \textbf{\color{white}Nome metrica} &
    \textbf{\color{white}Valore accettabile} &
    \textbf{\color{white}Valore ottimo} \\
    \hline
    \endfirsthead

    \hline
    \rowcolorhead
    \textbf{\color{white}Codice} &
    \textbf{\color{white}Nome metrica} &
    \textbf{\color{white}Valore accettabile} &
    \textbf{\color{white}Valore ottimo} \\
    \hline
    \endhead

    \endfoot

    \endlastfoot

    \rowcolor{cyan}
    \multicolumn{4}{c}{\textbf{Gestione organizzativa}}\\

    MPC11 &
    Risk found  &
    $\leq 5$ &
    $0$
    \\ \hline
    \rowcolor{white}
    \caption{Obbiettivi metriche dei processi organizzativi}
\end{xltabular}