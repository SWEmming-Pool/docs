\section{Qualità di processo}
La qualità del prodotto finale dipende dall'efficienza e dalla qualità dei
processi che lo compongono. Per assicurare che questi ultimi soddisfino gli
standard di qualità stabiliti, è fondamentale utilizzare metriche di valutazione
adeguate. Questa sezione espone i valori di qualità accettabili e ottimali
sulla base di metriche elencate nelle Norme di Progetto v1. È importante monitorare costantemente queste
metriche per garantire che i processi raggiungano gli obiettivi di qualità
prefissati e che il prodotto finale sia di alta qualità.


\subsection{Processi primari}
\subsubsection{Obbiettivi}
\begin{xltabular}{\textwidth} {
        >{\hsize=0.5\hsize\linewidth=\hsize}X
        >{\hsize=0.8\hsize\linewidth=\hsize}X
        >{\hsize=2.10\hsize\linewidth=\hsize}X
        >{\hsize=0.60\hsize\linewidth=\hsize}X
    }
    \rowcolorhead
    \textbf{\color{white}Processo} &
    \textbf{\color{white}Descrizione} &
    \textbf{\color{white}Metriche} \\
    \hline
    \endfirsthead

    \hline
    \rowcolorhead
    \textbf{\color{white}Processo} &
    \textbf{\color{white}Descrizione} &
    \textbf{\color{white}Metriche} \\
    \hline
    \endhead

    \endfoot

    \endlastfoot

    Fornitura &
    Processo che consiste nel scegliere le procedure e le
    risorse atte a perseguire lo sviluppo del progetto. &
    La \textit{web app} fornisce all'utente non autenticato la possibilità effettuare il login tramite \textit{MetaMask}.
    \\
    \hline
    Sviluppo &
    Processo che contiene le attivit`a e i compiti per
    realizzare il prodotto software richiesto.  &
    La \textit{web app} notifica all'utente un errore nella visualizzazione dei pagamenti.
    \\
    \hline

    \rowcolor{white}
    \caption{Requisiti funzionali}
\end{xltabular}
\subsection{Processi di supporto}