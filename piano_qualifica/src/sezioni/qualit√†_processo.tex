\section{Qualità di processo}
La qualità del prodotto finale dipende dall'efficienza e dalla qualità dei
processi che lo compongono. Per assicurare che questi ultimi soddisfino gli
standard di qualità stabiliti, è fondamentale utilizzare metriche di valutazione
adeguate. Questa sezione espone i valori di qualità accettabili e ottimali
sulla base di metriche elencate nelle Norme di Progetto v1. È importante monitorare costantemente queste
metriche per garantire che i processi raggiungano gli obiettivi di qualità
prefissati e che il prodotto finale sia di alta qualità.

\subsection{Obbiettivi di qualità di processo}
\subsubsection{Processi primari}
\begin{xltabular}{\textwidth} {
        >{\hsize=0.5\hsize\linewidth=\hsize}X
        >{\hsize=0.8\hsize\linewidth=\hsize}X
        >{\hsize=2.10\hsize\linewidth=\hsize}X
        >{\hsize=0.60\hsize\linewidth=\hsize}X
    }
    \rowcolorhead
    \textbf{\color{white}Processo} &
    \textbf{\color{white}Descrizione} &
    \textbf{\color{white}Metriche} \\
    \hline
    \endfirsthead

    \hline
    \rowcolorhead
    \textbf{\color{white}Processo} &
    \textbf{\color{white}Descrizione} &
    \textbf{\color{white}Metriche} \\
    \hline
    \endhead

    \endfoot

    \endlastfoot

    Fornitura &
    Processo che consiste nel scegliere le procedure e le
    risorse atte a perseguire lo sviluppo del progetto. &
    MPC-SV, MPC-BV, MPC-EAC, MPC-EV, MPC-PV, MPC-AC, MPC-ETC
    \\
    \hline
    Sviluppo &
    Processo che contiene le attività e i compiti per
    realizzare il prodotto software richiesto.  &
    MPC-RSI
    \\
    \hline

    \rowcolor{white}
    \caption{Obbiettivi processi primari}
\end{xltabular}

\subsubsection{Processi di supporto}
\begin{xltabular}{\textwidth} {
        >{\hsize=0.5\hsize\linewidth=\hsize}X
        >{\hsize=0.8\hsize\linewidth=\hsize}X
        >{\hsize=2.10\hsize\linewidth=\hsize}X
        >{\hsize=0.60\hsize\linewidth=\hsize}X
    }
    \rowcolorhead
    \textbf{\color{white}Processo} &
    \textbf{\color{white}Descrizione} &
    \textbf{\color{white}Metriche} \\
    \hline
    \endfirsthead

    \hline
    \rowcolorhead
    \textbf{\color{white}Processo} &
    \textbf{\color{white}Descrizione} &
    \textbf{\color{white}Metriche} \\
    \hline
    \endhead

    \endfoot

    \endlastfoot

    Verifica &
    Processo che si pone come obbiettivo il controllo dello
    sviluppo software a livello di codifica. &
    MPC-CC
    \\
    \hline

    \rowcolor{white}
    \caption{Obiettivi processi di supporto}
\end{xltabular}

\subsection{Obbiettivi metriche}
\subsubsection{Processi Primari}