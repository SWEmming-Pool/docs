\section{Introduzione}

\subsection{Scopo del documento}
Questo documento ha lo scopo di mostrare le strategie di verifica e validazione adottate dal gruppo \groupName ~ al fine di garantire la qualità di prodotto e di processo. Per raggiungere questo obiettivo viene applicato un sistema di verifica continua sui processi in corso e sulle attività svolte. In questo modo è quindi possibile rilevare e correggere all'istante eventuali anomalie, riducendo al minimo lo spreco di tempo e quindi risorse.\\
Il \textit{Piano di Qualifica} è un documento la cui evoluzione avviene per l'intera durata del progetto. Molti dei contenuti del documento sono di natura instabile e alcune parti del documento sono prodotte in fasi temporali successive. Per tutte queste ragioni, il documento è prodotto in maniera agile, e i suoi contenuti iniziali sono da considerarsi incompleti: subiranno significative aggiunte e modifiche nel tempo.

\subsection{Scopo del progetto}

Al giorno d'oggi poter garantire l'autenticità e la veridicità di una
recensione\glo \: risulta essere un problema. \\ Il capitolato \capName si
pone come obiettivo quello di creare un servizio che, attraverso l'uso di uno
\textit{smart contract}\glo \:, permetta di effettuare una recensione collegata in
modo univoco ad un pagamento. \\ Il prodotto atteso sarà composto da un
contratto digitale e una web app che, attraverso il wallet\glo  \textit{MetaMask}\glo \:, ne
permetta l'interazione con esso, e da un server \textit{API REST}\glo \: che consenta ad un
e-commerce\glo \: di mostrare le recensioni ricevute all'interno del proprio sito.

\subsection{Glossario}
Al fine di evitare ambiguità nella terminologia usata all'interno del seguente
documento è stato redatto un glossario, in cui vengono riportate le definizioni
di termini tecnici, rilevanti o con un significato particolare. \\ Per indicare
la presenza di un termine all'interno del glossario si è scelto di
contrassegnarlo con \glo .\\ Per non appesantire la lettura della documentazione
verrà così contrassegnata solo la prima occorrenza di ogni termine in ciascun
documento.

Per una consultazione completa si rimanda al \textit{Glossario v1.0.0}.

\subsection{Riferimenti}
\subsubsection{Riferimenti normativi}
\begin{itemize}
    \item Capitolato d'appalto C7: \textbf{Trustify - Authentic and verifiable reviews platform}: \\
          \url{https://www.math.unipd.it/~tullio/IS-1/2022/Progetto/C7.pdf} \hfill\break [Ultimo accesso: \today]. % TODO: aggiungere data valida
\end{itemize}
\subsubsection{Riferimenti informativi}
\begin{itemize}
    \item \textbf{ISO/IEC 9126}: \\
          \url{https://it.wikipedia.org/wiki/ISO/IEC_9126} \hfill\break [Ultimo accesso: \today]; % TODO: aggiungere data valida
    \item \textbf{ISO/IEC 12207}: \\
          \url{https://www.math.unipd.it/~tullio/IS-1/2009/Approfondimenti/ISO_12207-1995.pdf} \hfill\break [Ultimo accesso: \today]. % TODO: aggiungere data valida
\end{itemize}