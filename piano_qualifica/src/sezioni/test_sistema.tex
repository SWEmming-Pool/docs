\section{Test di sistema}

Nella tabella seguente vengono inseriti i test di sistema per ciscun requisito ed avranno la seguente nomenclatura:\\
\begin{center}
    \textbf{[TS]+[Codice requisito]}
\end{center}
Dove TS indica Test di Sistema.


\renewcommand{\arraystretch}{1.8}
\begin{xltabular}{\textwidth} {
        >{\hsize=0.6\hsize\linewidth=\hsize}X
        >{\hsize=1.40\hsize\linewidth=\hsize}X
        >{\hsize=1.00\hsize\linewidth=\hsize}X
    }
    \rowcolorhead
    \textbf{\color{white}Codice} &
    \textbf{\color{white}Descrizione} &
    \textbf{\color{white}Implementazione}\\
    \hline
    \endfirsthead

    \hline
    \rowcolorhead
    \textbf{\color{white}Codice} &
    \textbf{\color{white}Descrizione} &
    \textbf{\color{white}Implementazione}\\
    \hline
    \endhead

    \endfoot

    \endlastfoot

    TSR1F1 &
    Si verifica che l'utente non autenticato ha la possibilità di effetuare il login tramite \textit{Metamask} &
    Non implementato.
    \\ \hline
    
    TSR1F1.1 &
    Si verifica che in caso di autenticazione fallita venga visualizzato un errore &
    Non implementato.
    \\ \hline

    TSR1F1.2 &
    Si verifica che in caso di mancata installazione di \textit{Metamask} venga visualizzato un errore &
    Non implementato.
    \\ \hline

    TSR1F2 &
    Si verifica la possibilità di logout da \textit{Metamask} &
    Non implementato.
    \\ \hline

    TSR1F2.1&
    Si verifica che in caso di mancato logout venga visualizzato l'errore logout fallito &
    Non implementato.
    \\ \hline

    TSR1F3&
    Si verifica l'utente autenticato possa eseguire un pagamento tramite \textit{Metamask} &
    Non implementato.
    \\ \hline
    
    TSR1F3.1&
    Si verifica che in caso di fondi insufficienti il pagamento non vada a buon fine&
    Non implementato.
    \\ \hline
    
    TSR1F3.2&
    Si verifica che in caso di indirizzo dell'utente ricevente non valido, il pagamento non vada a buon fine&
    Non implementato.
    \\ \hline
    
    TSR1F4&
    Si verifica che l'utente autenticato possa visualizzare una lista dalla lista di pagamenti effettuati, senza una recensione, dal quale scegliere quale recensire&
    Non implementato.
    \\ \hline

    TSR1F5&
    Si verifica che di acquisti non ancora recnesiti, l'utente autenticato ha la possibilità di rilasciare una recensione&
    Non implementato.
    \\ \hline

    TSR1F5.1&
    Si verifica che per ciascuna transazione esista al massimo una recensione ad essa collegata&
    Non implementato.
    \\ \hline
    
    TSR1F5.2&
    Si verifica che l'utente possa scegliere la transazione effettuata da associare alla recensione&
    Non implementato.
    \\ \hline

    TSR1F5.3&
    Si verifica che venga visualizzato un errore in caso di formato recensioen non valido&
    Non implementato.
    \\ \hline
    
    TSR1F5.3.1&
    Si verifica che la recensione abbia un titolo non superiore ai 50 caratteri&
    Non implementato.
    \\ \hline

    TSR1F5.3.2&
    Si verifica che la recensione abbia un voto compreso tra 1 e 5&
    Non implementato.
    \\ \hline

    TSR1F5.3.3&
    Si verifica che la recensione abbia un testo non superiore ai 500 caratteri&
    Non implementato.
    \\ \hline

    TSR1F6&
    Si verifica che l'utente generico possa ricercare recensioni&
    Non implementato.
    \\ \hline

    TSR1F6.1&
    Si verifica che l'utente generico possa ricercare recensioni filtrandole per indirizzo wallet&
    Non implementato.
    \\ \hline
    
    TSR1F6.1.1&
    Si verifica che l'utente generico possa ricercare recensioni filtrandole per indirizzo wallet e per autore&
    Non implementato.
    \\ \hline
    
    TSR1F6.1.2&
    Si verifica che l'utente generico possa ricercare recensioni filtrandole per indirizzo wallet e per destinatario&
    Non implementato.
    \\ \hline

    TSR3F6.2&
    Si verifica che l'utente generico possa ricercare recensioni filtrandole per voto&
    Non implementato.
    \\ \hline
    
    TSR3F6.3&
    Si verifica che l'utente generico possa ricercare recensioni filtrandole per titolo&
    Non implementato.
    \\ \hline

    TSR3F6.4&
    Si verifica che l'utente generico possa ricercare recensioni filtrandole per data&
    Non implementato.
    \\ \hline

    TSR1F7&
    Si verifica che l'utente autenticato possa ricercare recensioni rilasciate &
    Non implementato.
    \\ \hline

    TSR1F8&
    Si verifica che l'utente autenticato possa ricercare recensioni ricevute &
    Non implementato.
    \\ \hline
    
    TSR1F9&
    Si verifica che venga un visualizzato un errore quando la ricerca non va a buon fine &
    Non implementato.
    \\ \hline

    TSR1F9.1&
    Si verifica che venga un visualizzato un errore quando la ricerca per indirizzo wallet non va a buon fine &
    Non implementato.
    \\ \hline
    
    TSR3F9.2&
    Si verifica che venga un visualizzato un errore quando la ricerca per voto non va a buon fine &
    Non implementato.
    \\ \hline
    
    TSR3F9.3&
    Si verifica che venga un visualizzato un errore quando la ricerca per titolo non va a buon fine &
    Non implementato.
    \\ \hline

    TSR3F9.4&
    Si verifica che venga un visualizzato un errore quando la ricerca per data non va a buon fine &
    Non implementato.
    \\ \hline

    TSR1F10&
    Si verifica che venga visualizzata una lista di recensioni &
    Non implementato.
    \\ \hline
    

    TSR3F10.1 &
    Si verifica che l'utente generico abbia la possiblità di visualizzare una lista di recensioni,
    ordinate dal meno recente.  &
    Non implementato.
    \\ \hline

    TSR3F10.2 &
    Si verifica che l'utente generico abbia la possiblità di visualizzare una lista di recensioni,
    ordinate dal più recente.  &
    Non implementato.
    \\ \hline

    TSR1F11 &
    Si verifica che l'utente generico abbia la possiblità di visualizzare una singola recensione  &
    Non implementato.
    \\ \hline
    
    TSR1F11.1 &
    Si verifica che l'utente generico abbia la possiblità di visualizzare l'autore di una recensione  &
    Non implementato.
    \\ \hline
    
    TSR1F11.2 &
    Si verifica che l'utente generico abbia la possiblità di visualizzare il destinatario di una recensione  &
    Non implementato.
    \\ \hline

    TSR1F11.3 &
    Si verifica che l'utente generico abbia la possiblità di visualizzare il titolo di una recensione  &
    Non implementato.
    \\ \hline

    TSR1F11.4 &
    Si verifica che l'utente generico abbia la possiblità di visualizzare la data di una recensione  &
    Non implementato.
    \\ \hline

    TSR1F11.5 &
    Si verifica che l'utente generico abbia la possiblità di visualizzare il voto di una recensione  &
    Non implementato.
    \\ \hline

    TSR1F11.6 &
    Si verifica che l'utente generico abbia la possiblità di visualizzare il testo di una recensione  &
    Non implementato.
    \\ \hline

    TSR1F12 &
    Si verifica che l'utente autenticato abbia la possiblità di visualizzare le recensioni rilasciate  &
    Non implementato.
    \\ \hline

    TSR1F13 &
    Si verifica che l'utente autenticato abbia la possiblità di visualizzare le recensioni ricevute  &
    Non implementato.
    \\ \hline

    TSR1F14 &
    Si verifica che venga notificato un errore nella visualizzazione delle recensioni  &
    Non implementato.
    \\ \hline

    TSR1F15 &
    Si verifica che il server API REST fornisca all'utente la possibilità di visualizzare le recensioni a lui riferite &
    Non implementato.
    \\ \hline

    TSR1F16 &
    Si verifica che il server API REST notifichi un errore all'utente in caso di assenza di recensioni a lui collegate &
    Non implementato.
    \\ \hline


    TSR3F17 &
    Si verifica che l'utente abbia la possibilità visualizzare la lista di tutti i pagamenti
    effettuati &
    Non implementato.
    \\ \hline

    TSR3F17.1 &
    Si verifica che l'utente abbia la possibilità visualizzare la lista di tutti i pagamenti
    effettuati, ordinandoli dal meno recente. &
    Non implementato.
    \\ \hline

    TSR3F17.2 &
    Si verifica che l'utente abbia la possibilità visualizzare la lista di tutti i pagamenti
    effettuati, ordinandoli dal più recente. &
    Non implementato.
    \\ \hline

    TSR3F17.3 &
    Si verifica che l'utente abbia la possibilità visualizzare la lista di tutti i pagamenti
    effettuati, ordinandoli per importo più economico. &
    Non implementato.
    \\ \hline
    
    TSR3F17.4 &
    Si verifica che l'utente abbia la possibilità visualizzare la lista di tutti i pagamenti
    effettuati, ordinandoli per importo meno economico. &
    Non implementato.
    \\ \hline

    TSR3F18 &
    Si verifica che l'utente autenticato possa visualizzare una singola transazione &
    Non implementato.
    \\ \hline

    TSR3F18.1 &
    Si verifica che l'utente autenticato possa visualizzare l'ID di una singola transazione &
    Non implementato.
    \\ \hline


    TSR3F18.2 &
    Si verifica che l'utente autenticato possa visualizzare la data di una singola transazione &
    Non implementato.
    \\ \hline

    TSR3F18.3 &
    Si verifica che l'utente autenticato possa visualizzare l'importo di una singola transazione &
    Non implementato.
    \\ \hline

    TSR3F18.4 &
    Si verifica che l'utente autenticato possa visualizzare il l'utente pagante di una transazione &
    Non implementato.
    \\ \hline

    TSR3F18.5 &
    Si verifica che l'utente autenticato possa visualizzare il destinatario di una transazione &
    Non implementato.
    \\ \hline

    TSR3F19 &
    Si verifica che l'utente autenticato sia avvisato in caso di errore di visualizzazione dei pagamenti &
    Non implementato.
    \\ \hline


    \rowcolor{white}
    \caption{Test di sistema}
\end{xltabular}

\subsection{Tracciamento dei requisiti}

\renewcommand{\arraystretch}{1.8}
\begin{xltabular}{\textwidth} {
        >{\hsize=1\hsize\linewidth=\hsize}X
        >{\hsize=1\hsize\linewidth=\hsize}X
    }
    \rowcolorhead
    \textbf{\color{white}Codice test} &
    \textbf{\color{white}Codice requisito}\\
    \hline
    \endfirsthead

    \hline
    \rowcolorhead
    \textbf{\color{white}Codice test} &
    \textbf{\color{white}Codice requisito} \\
    \hline
    \endhead

    \endfoot

    \endlastfoot

    TSR1F1 &
    R1F1
    \\ \hline
    
    TSR1F1.1 &
    R1F1.1
    \\ \hline

    TSR1F1.2 &
    R1F1.2
    \\ \hline

    TSR1F2 &
    R1F2
    \\ \hline

    TSR1F2.1&
    R1F2.1
    \\ \hline

    TSR1F3&
    R1F3
    \\ \hline
    
    TSR1F3.1&
    R1F3.1
    \\ \hline
    
    TSR1F3.2&
    R1F3.2
    \\ \hline
    
    TSR1F4&
    R1F4
    \\ \hline

    TSR1F5&
    R1F5
    \\ \hline

    TSR1F5.1&
    R1F5.1
    \\ \hline
    
    TSR1F5.2&
    R1F5.2
    \\ \hline

    TSR1F5.3&
    R1F5.3
    \\ \hline
    
    TSR1F5.3.1&
    R1F5.3.1
    \\ \hline

    TSR1F5.3.2&
    R1F5.3.2
    \\ \hline

    TSR1F5.3.3&
    R1F5.3.3
    \\ \hline

    TSR1F6&
    R1F
    \\ \hline

    TSR1F6.1&
    R1F6.1
    \\ \hline
    
    TSR1F6.1.1&
    R1F6.1.1
    \\ \hline
    
    TSR1F6.1.2&
    R1F6.1.2
    \\ \hline

    TSR3F6.2&
    R3F6.2
    \\ \hline
    
    TSR3F6.3&
    R3F6.3
    \\ \hline

    TSR3F6.4&
    R3F6.4
    \\ \hline

    TSR1F7&
    R1F7
    \\ \hline

    TSR1F8&
    R1F8
    \\ \hline
    
    TSR1F9&
    R1F9
    \\ \hline

    TSR1F9.1&
    R1F9.1
    \\ \hline
    
    TSR3F9.2&
    R3F9.2
    \\ \hline
    
    TSR3F9.3&
    R3F9.3
    \\ \hline

    TSR3F9.4&
    R3F9.4
    \\ \hline

    TSR1F10&
    R1F10
    \\ \hline
    

    TSR3F10.1 &
    R3F10.1
    \\ \hline

    TSR3F10.2 &
    R3F10.2
    \\ \hline

    TSR1F11 &
    R1F11
    \\ \hline
    
    TSR1F11.1 &
    R1F11.1
    \\ \hline
    
    TSR1F11.2 &
    R1F11.2
    \\ \hline

    TSR1F11.3 &
    R1F11.3
    \\ \hline

    TSR1F11.4 &
    R1F11.4
    \\ \hline

    TSR1F11.5 &
    R1F11.5
    \\ \hline

    TSR1F11.6 &
    R1F11.6
    \\ \hline

    TSR1F12 &
    R1F12
    \\ \hline

    TSR1F13 &
    R1F13
    \\ \hline

    TSR1F14 &
    R1F14
    \\ \hline

    TSR1F15 &
    R1F15
    \\ \hline

    TSR1F16 &
    R1F16
    \\ \hline


    TSR3F17 &
    R3F17
    \\ \hline

    TSR3F17.1 &
    R3F17.1
    \\ \hline

    TSR3F17.2 &
    R3F17.2
    \\ \hline

    TSR3F17.3 &
    R3F17.3
    \\ \hline
    
    TSR3F17.4 &
    R3F17.4
    \\ \hline

    TSR3F18 &
    R3F18
    \\ \hline

    TSR3F18.1 &
    R3F18.1
    \\ \hline


    TSR3F18.2 &
    R3F18.2
    \\ \hline

    TSR3F18.3 &
    R3F18.3
    \\ \hline

    TSR3F18.4 &
    R3F18.4
    \\ \hline

    TSR3F18.5 &
    R3F18.5
    \\ \hline

    TSR3F19 &
    R3F19
    \\ \hline

    \rowcolor{white}
    \caption{Test di sistema con tracciamento dei requisiti}
\end{xltabular}



\pagebreak