\section{Analisi Dinamica}
\label{sec:analisi_dinamica}

    \subsection{Specifica dei test}
    \begin{itemize}
        \item Test di unità: testano singole componenti del sistema e verranno definiti durante la progettazione di dettaglio;
        \item Test di integrazione: verificano la corretta interazione tra le componenti del sistema e verranno definiti durante la progettazione di dettaglio;
        \item Test di sistema: verificano il corretto funzionamento del sistema per accertare la copertura dei requisiti in preparazione all'accettazione, verranno definiti durante la progettazione di dettaglio;
        \item Test di accettazione: effettuati assieme al proponente durante la fase di collaudo e verranno definiti durante la progettazione di dettaglio.
    \end{itemize}

    \subsection{Test di sistema}

    Nella tabella seguente vengono inseriti i test di sistema per ciascun requisito ed avranno la seguente nomenclatura:\\
    \begin{center}
        \textbf{[TS]+[Codice requisito]}
    \end{center}
    dove TS indica Test di Sistema.
    Si specifica inoltre che i requisiti fanno affidamento alla versione 1.0.2 del documento \textit{Analisi dei requisiti}.


    \renewcommand{\arraystretch}{1.8}
    \begin{xltabular}{\textwidth} {
            >{\hsize=0.6\hsize\linewidth=\hsize}X
            >{\hsize=1.40\hsize\linewidth=\hsize}X
            >{\hsize=1.00\hsize\linewidth=\hsize}X
        }
        \rowcolorhead
        \textbf{\color{white}Codice} &
        \textbf{\color{white}Descrizione} &
        \textbf{\color{white}Implementazione}\\
        \hline
        \endfirsthead

        \hline
        \rowcolorhead
        \textbf{\color{white}Codice} &
        \textbf{\color{white}Descrizione} &
        \textbf{\color{white}Implementazione}\\
        \hline
        \endhead

        \endfoot

        \endlastfoot

        TSR1F1 &
        Si verifica che l'utente non autenticato ha la possibilità di effettuare il login tramite \textit{MetaMask}. &
        Non implementato.
        \\ \hline
        
        TSR1F1.1 &
        Si verifica che in caso di autenticazione fallita venga visualizzato un errore. &
        Non implementato.
        \\ \hline

        TSR1F1.2 &
        Si verifica che in caso di mancata installazione di \textit{MetaMask} venga visualizzato un errore. &
        Non implementato.
        \\ \hline

        TSR1F2&
        Si verifica che l'utente autenticato possa eseguire un pagamento tramite \textit{MetaMask} verso un altro indirizzo wallet. &
        Non implementato.
        \\ \hline

        TSR1F2.1&
        Si verifica che in caso di errore il pagamento non vada a buon fine e che l'utente venga notifcato.&
        Non implementato.
        \\ \hline
        
        TSR1F3&
        Si verifica che l'utente autenticato possa visualizzare un elenco dalla lista di pagamenti effettuati, senza una recensione, dal quale poter scegliere quale recensire.&
        Non implementato.
        \\ \hline

        TSR1F4&
        Si verifica che per acquisti non ancora recensiti, l'utente autenticato ha la possibilità di rilasciare una recensione.&
        Non implementato.
        \\ \hline

        TSR1F4.1&
        Si verifica che per ciascuna transazione esista al massimo una recensione ad essa collegata.&
        Non implementato.
        \\ \hline
        
        TSR1F4.2&
        Si verifica che l'utente possa scegliere la transazione effettuata da associare alla recensione.&
        Non implementato.
        \\ \hline

        TSR1F4.3&
        Si verifica che l'utente abbia inserito un titolo alla recensione.&
        Non implementato.
        \\ \hline

        TSR1F4.4&
        Si verifica che l'utente abbia inserito un voto alla recensione.&
        Non implementato.
        \\ \hline

        TSR1F4.5&
        Si verifica che l'utente abbia inserito un testo alla recensione.&
        Non implementato.
        \\ \hline

        TSR1F4.6&
        Si verifica che venga visualizzato un errore in caso di formato recensione non valido.&
        Non implementato.
        \\ \hline
        
        TSR1F4.6.1&
        Si verifica che la recensione abbia un titolo non superiore ai 50 caratteri.&
        Non implementato.
        \\ \hline

        TSR1F4.6.2&
        Si verifica che la recensione abbia un voto compreso tra 1 e 5.&
        Non implementato.
        \\ \hline

        TSR1F4.6.3&
        Si verifica che la recensione abbia un testo non superiore ai 500 caratteri.&
        Non implementato.
        \\ \hline

        TSR1F5&
        Si verifica che l'utente generico possa ricercare recensioni.&
        Non implementato.
        \\ \hline

        TSR1F5.1&
        Si verifica che l'utente generico possa ricercare recensioni filtrandole per indirizzo wallet.&
        Non implementato.
        \\ \hline
        
        TSR1F5.1.1&
        Si verifica che l'utente generico possa ricercare recensioni filtrandole per indirizzo wallet e per autore.&
        Non implementato.
        \\ \hline
        
        TSR1F5.1.2&
        Si verifica che l'utente generico possa ricercare recensioni filtrandole per indirizzo wallet e per destinatario.&
        Non implementato.
        \\ \hline

        TSR3F5.2&
        Si verifica che l'utente generico possa ricercare recensioni filtrandole per voto.&
        Non implementato.
        \\ \hline
        
        TSR3F5.3&
        Si verifica che l'utente generico possa ricercare recensioni filtrandole per titolo.&
        Non implementato.
        \\ \hline

        TSR3F5.4&
        Si verifica che l'utente generico possa ricercare recensioni filtrandole per data.&
        Non implementato.
        \\ \hline

        TSR1F6&
        Si verifica che l'utente autenticato possa ricercare recensioni rilasciate. &
        Non implementato.
        \\ \hline

        TSR1F7&
        Si verifica che l'utente autenticato possa ricercare recensioni ricevute. &
        Non implementato.
        \\ \hline
        
        TSR1F8&
        Si verifica che venga visualizzato un errore quando la ricerca non va a buon fine. &
        Non implementato.
        \\ \hline

        TSR1F8.1&
        Si verifica che venga visualizzato un errore quando la ricerca per indirizzo wallet non va a buon fine. &
        Non implementato.
        \\ \hline
        
        TSR3F8.2&
        Si verifica che venga visualizzato un errore quando la ricerca per voto non va a buon fine. &
        Non implementato.
        \\ \hline
        
        TSR3F8.3&
        Si verifica che venga visualizzato un errore quando la ricerca per titolo non va a buon fine. &
        Non implementato.
        \\ \hline

        TSR3F8.4&
        Si verifica che venga visualizzato un errore quando la ricerca per data non va a buon fine. &
        Non implementato.
        \\ \hline

        TSR1F9 &
        Si verifica che l'utente generico abbia la possibilità di visualizzare una lista di recensioni.  &
        Non implementato.
        \\ \hline

        TSR1F10 &
        Si verifica che l'utente generico abbia la possibilità di visualizzare una singola recensione. &
        Non implementato.
        \\ \hline

        TSR1F10.1 &
        Si verifica che l'utente generico abbia la possibilità di visualizzare l'autore di una singola recensione.  &
        Non implementato.
        \\ \hline

        TSR1F10.2 &
        Si verifica che l'utente generico abbia la possibilità di visualizzare il destinatario di una singola recensione.  &
        Non implementato.
        \\ \hline

        TSR1F10.3 &
        Si verifica che l'utente generico abbia la possibilità di visualizzare il titolo di una singola recensione.  &
        Non implementato.
        \\ \hline

        TSR1F10.4 &
        Si verifica che l'utente generico abbia la possibilità di visualizzare la data di una singola recensione.  &
        Non implementato.
        \\ \hline
        
        TSR1F10.5 &
        Si verifica che l'utente generico abbia la possibilità di visualizzare il voto di una singola recensione.  &
        Non implementato.
        \\ \hline

        TSR1F10.6 &
        Si verifica che l'utente generico abbia la possibilità di visualizzare il testo di una singola recensione.  &
        Non implementato.
        \\ \hline

        TSR1F11 &
        Si verifica che l'utente autenticato abbia la possibilità di visualizzare le recensioni rilasciate.  &
        Non implementato.
        \\ \hline

        TSR1F12 &
        Si verifica che l'utente autenticato abbia la possibilità di visualizzare le recensioni ricevute.  &
        Non implementato.
        \\ \hline

        TSR1F13 &
        Si verifica che venga notificato un errore nella visualizzazione delle recensioni.  &
        Non implementato.
        \\ \hline

        TSR1F14 &
        Si verifica che il server API REST fornisca all'utente la possibilità di visualizzare le recensioni a lui riferite. &
        Non implementato.
        \\ \hline

        TSR1F15 &
        Si verifica che il server API REST notifichi un errore all'utente in caso di assenza di recensioni a lui collegate. &
        Non implementato.
        \\ \hline


        TSR1F16 &
        Si verifica che l'utente abbia la possibilità visualizzare la lista di tutti i pagamenti
        effettuati. &
        Non implementato.
        \\ \hline

        TSR1F17 &
        Si verifica che l'utente autenticato possa visualizzare una singola transazione. &
        Non implementato.
        \\ \hline

        TSR1F17.1 &
        Si verifica che l'utente autenticato possa visualizzare l'ID di una singola transazione. &
        Non implementato.
        \\ \hline


        TSR1F17.2 &
        Si verifica che l'utente autenticato possa visualizzare la data di una singola transazione. &
        Non implementato.
        \\ \hline

        TSR1F17.3 &
        Si verifica che l'utente autenticato possa visualizzare l'importo di una singola transazione. &
        Non implementato.
        \\ \hline

        TSR1F17.4 &
        Si verifica che l'utente autenticato possa visualizzare l'utente pagante di una transazione. &
        Non implementato.
        \\ \hline

        TSR1F17.5 &
        Si verifica che l'utente autenticato possa visualizzare il destinatario di una transazione. &
        Non implementato.
        \\ \hline

        TSR1F18 &
        Si verifica che l'utente autenticato sia avvisato in caso di errore di visualizzazione dei pagamenti. &
        Non implementato.
        \\ \hline

        TSR1F19 &
        Si verifica che la web app fornisca all'utente generico la possibilità di visualizzare una lista di recensioni ordinate. &
        Non implementato. 
        \\ \hline

        TSR1F19.1 &
        Si verifica che la web app fornisca all'utente generico la possibilità di visualizzare una lista di recensioni ordinate dal meno recente. &
        Non implementato. 
        \\ \hline

        TSR1F19.2 &
        Si verifica che la web app fornisca all'utente generico la possibilità di visualizzare una lista di recensioni ordinate dal più recente. &
        Non implementato. 
        \\ \hline

        TSR1F20 &
        Si verifica che la web app fornisca all'utente generico la possibilità di visualizzare una lista di pagamenti ordinati in base alla data del pagamento. &
        Non implementato. 
        \\ \hline

        TSR1F20.1 &
        Si verifica che l'utente abbia la possibilità di visualizzare la lista di tutti i pagamenti
        effettuati, ordinandoli dal meno recente. &
        Non implementato.
        \\ \hline

        TSR1F20.2 &
        Si verifica che l'utente abbia la possibilità di visualizzare la lista di tutti i pagamenti
        effettuati, ordinandoli dal più recente. &
        Non implementato.
        \\ \hline

        TSR1F21 &
        Si verifica che la web app fornisca all'utente generico la possibilità di visualizzare una lista di pagamenti ordinati in base all'importo del pagamento. &
        Non implementato. 
        \\ \hline

        TSR1F21.1 &
        Si verifica che l'utente abbia la possibilità visualizzare la lista di tutti i pagamenti
        effettuati, ordinandoli per importo più economico. &
        Non implementato.
        \\ \hline
        
        TSR1F21.2 &
        Si verifica che l'utente abbia la possibilità visualizzare la lista di tutti i pagamenti
        effettuati, ordinandoli per importo meno economico. &
        Non implementato.
        \\ \hline


        \rowcolor{white}
        \caption{Test di sistema}
    \end{xltabular}

    \subsection{Tracciamento dei requisiti}

    \renewcommand{\arraystretch}{1.8}
    \begin{xltabular}{\textwidth} {
            >{\hsize=1\hsize\linewidth=\hsize}X
            >{\hsize=1\hsize\linewidth=\hsize}X
        }
        \rowcolorhead
        \textbf{\color{white}Codice test} &
        \textbf{\color{white}Codice requisito}\\
        \hline
        \endfirsthead

        \hline
        \rowcolorhead
        \textbf{\color{white}Codice test} &
        \textbf{\color{white}Codice requisito} \\
        \hline
        \endhead

        \endfoot

        \endlastfoot

        TSR1F1 &
        R1F1
        \\ \hline
        
        TSR1F1.1 &
        R1F1.1
        \\ \hline

        TSR1F1.2 &
        R1F1.2
        \\ \hline

        TSR1F2&
        R1F2
        \\ \hline

        TSR1F2.1&
        R1F2.1
        \\ \hline
        
        TSR1F3&
        R1F3
        \\ \hline

        TSR1F4&
        R1F4
        \\ \hline

        TSR1F4.1&
        R1F4.1
        \\ \hline
        
        TSR1F4.2&
        R1F4.2
        \\ \hline

        TSR1F4.3&
        R1F4.3
        \\ \hline
        
        TSR1F4.4&
        R1F4.4
        \\ \hline

        TSR1F4.5&
        R1F4.5
        \\ \hline

        TSR1F4.6&
        R1F4.6
        \\ \hline
        
        TSR1F4.6.1&
        R1F4.6.1
        \\ \hline

        TSR1F4.6.2&
        R1F4.6.2
        \\ \hline

        TSR1F4.6.3&
        R1F4.6.3
        \\ \hline

        TSR1F5&
        R1F
        \\ \hline

        TSR1F5.1&
        R1F5.1
        \\ \hline
        
        TSR1F5.1.1&
        R1F5.1.1
        \\ \hline
        
        TSR1F5.1.2&
        R1F5.1.2
        \\ \hline

        TSR3F5.2&
        R3F5.2
        \\ \hline
        
        TSR3F5.3&
        R3F5.3
        \\ \hline

        TSR3F5.4&
        R3F5.4
        \\ \hline

        TSR1F6&
        R1F6
        \\ \hline

        TSR1F7&
        R1F7
        \\ \hline
        
        TSR1F8&
        R1F8
        \\ \hline

        TSR1F8.1&
        R1F8.1
        \\ \hline
        
        TSR3F8.2&
        R3F8.2
        \\ \hline
        
        TSR3F8.3&
        R3F8.3
        \\ \hline

        TSR3F8.4&
        R3F8.4
        \\ \hline

        TSR1F9&
        R1F9
        \\ \hline

        TSR1F10&
        R1F10
        \\ \hline
        
        TSR1F10.1 &
        R1F10.1
        \\ \hline
        
        TSR1F10.2 &
        R1F10.2
        \\ \hline

        TSR1F10.3 &
        R1F10.3
        \\ \hline

        TSR1F10.4 &
        R1F10.4
        \\ \hline

        TSR1F10.5 &
        R1F10.5
        \\ \hline

        TSR1F10.6 &
        R1F10.6
        \\ \hline

        TSR1F11 &
        R1F11
        \\ \hline

        TSR1F12 &
        R1F12
        \\ \hline

        TSR1F13 &
        R1F13
        \\ \hline

        TSR1F14 &
        R1F14
        \\ \hline

        TSR1F15 &
        R1F15
        \\ \hline


        TSR1F16 &
        R1F16
        \\ \hline

        TSR1F17 &
        R1F17
        \\ \hline

        TSR1F17.1 &
        R1F17.1
        \\ \hline


        TSR1F17.2 &
        R1F17.2
        \\ \hline

        TSR1F17.3 &
        R1F17.3
        \\ \hline

        TSR1F17.4 &
        R1F17.4
        \\ \hline

        TSR1F17.5 &
        R1F17.5
        \\ \hline

        TSR1F18 &
        R1F18
        \\ \hline

        TSR1F19 &
        R1F19
        \\ \hline

        TSR1F19.1 &
        R1F19.1
        \\ \hline

        TSR1F19.2 &
        R1F19.2
        \\ \hline

        TSR1F20 &
        R1F20
        \\ \hline

        TSR1F20.1 &
        R1F20.1
        \\ \hline

        TSR1F20.2 &
        R1F21.2 
        \\ \hline

        TSR1F21 &
        R1F21
        \\ \hline

        TSR1F21.1 &
        R1F21.1
        \\ \hline

        TSR1F21.2 &
        R1F21.2
        \\ \hline

        \rowcolor{white}
        \caption{Test di sistema con tracciamento dei requisiti}
    \end{xltabular}

    \pagebreak