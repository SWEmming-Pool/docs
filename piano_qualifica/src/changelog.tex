\section*{Registro delle modifiche}

\rowcolors{2}{pari_alt}{dispari_alt}
\renewcommand{\arraystretch}{1.5}
\begin{xltabular}{\textwidth} {
		>{\hsize=0.45\hsize\linewidth=\hsize}X
		>{\hsize=0.7\hsize\linewidth=\hsize}X
		>{\hsize=1.15\hsize\linewidth=\hsize}X
		>{\hsize=0.8\hsize\linewidth=\hsize}X
		>{\hsize=1.85\hsize\linewidth=\hsize}X
	}
	\rowcolorhead
	\textbf{\color{white}Versione} &
	\textbf{\color{white}Data} &
	\textbf{\color{white}Nominativo} &
	\textbf{\color{white}Ruolo} &
	\textbf{\color{white}Descrizione} \\
	\hline
	\endfirsthead

	\hline
	\rowcolorhead
	\textbf{\color{white}Versione} &
	\textbf{\color{white}Data} &
	\textbf{\color{white}Nominativo} &
	\textbf{\color{white}Ruolo} &
	\textbf{\color{white}Descrizione} \\
	\hline
	\endhead

	\endfoot
	\endlastfoot

	0.0.5 &
	2022-02-06 &
	Nicolò Trinca &
	Redattore &
	Sistemazione di alcuni problemi e rimozione di alcune metriche. \\
	\hline

	0.0.4 &
	2022-01-20 &
	Nicolò Trinca &
	Redattore &
	Inizio qualità prodotto e aggiunta formule matematiche. \\
	\hline

	0.0.3 &
	2022-01-10 &
	Nicolò Trinca &
	Redattore &
	Stesura qualità processo. \\
	\hline

	0.0.2 &
	2022-01-06 &
	Nicolò Trinca &
	Redattore &
	Stesura metriche. \\
	\hline

	0.0.1 &
	2022-01-04 &
	Nicolò Trinca &
	Redattore &
	Creata struttura del documento in \LaTeX{}. \\
	\hline

\end{xltabular}
\renewcommand{\arraystretch}{1}