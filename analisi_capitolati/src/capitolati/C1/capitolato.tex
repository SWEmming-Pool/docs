% =============================================================================
% COMANDI COMUNI
% =============================================================================

% Informazini sul capitolato
\renewcommand{\capName}{CAPTCHA: Umano o Sovrumano?} %Nome capitolato
\renewcommand{\capCode}{C1} %Numero CX
\renewcommand{\capLink}{https://www.math.unipd.it/~tullio/IS-1/2022/Progetto/C1.pdf} %Link pagina prof
\renewcommand{\capProposer}{Zucchetti} % Nome azienda

% =============================================================================
% STRUTTURA
% =============================================================================

\section{\capCode\ - \capName} \footnote{Riferimenti: \url{\capLink}}
\subsection{Descrizione generale}
L'azienda richiede lo sviluppo di un sistema \emph{CAPTCHA} che verifichi se un utente sia umano o un robot. \\
Per testare tale sistema di verifica si richiede di sviluppare un'applicazione web che permetta di effettuare il login. 

\subsection{Tecnologie richieste}
Nel capitolato sono proposti vari suggerimenti per la realizzazione di un sistema \emph{CAPTCHA}. \\
Per quanto riguarda alla pagina web di login, non è stata specificata alcuna tecnologia.

\subsection{Valutazione capitolato}
Dalle discussioni tra membri del gruppo è emerso che

\subsubsection{Fattori positivi}

\begin{itemize}
    \item Nessuno
\end{itemize}

\subsubsection{Criticità e fattori negativi}

\begin{itemize}
    \item Dal gruppo non è emmerso alcun interesse per questo capitolato.
\end{itemize}

\subsection{Riferimenti}
La presentazione del capitolato è reperibile all'indirizzo \url{\capLink}