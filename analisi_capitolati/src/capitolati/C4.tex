% =============================================================================
% COMANDI COMUNI
% =============================================================================

% Informazioni sul capitolato
\renewcommand{\capName}{Piattaforma Localizzazione Testi} %Nome capitolato
\renewcommand{\capCode}{C4} %Numero CX
\renewcommand{\capLink}{https://www.math.unipd.it/~tullio/IS-1/2022/Progetto/C3.pdf} %Link pagina prof
\renewcommand{\capProposer}{zero12} % Nome azienda

% =============================================================================
% STRUTTURA
% =============================================================================

\subsection{\capCode\ - \capName}
\textbf{Azienda proponente:} zero12
\subsubsection{Descrizione generale}
L'azienda richiede lo sviluppo di una piattaforma che permetta di gestire le traduzioni di testi in diverse lingue. \\
Tale piattaforma deve supportare un'architettura \textit{multi-tenant}; deve cioè essere in grado di erogare il servizio a diverse tipologie di utenti, isolando le risorse a disposizione di ognuno.

\subsubsection{Tecnologie richieste}
L'azienda raccomanda l'utilizzo delle seguenti tecnologie, in quanto può garantire supporto riguardo ad esse in caso di necessità:
\begin{itemize}
    \item Server:
          \begin{itemize}
              \item \emph{AWS fargate}: servizio serverless per la gestione a container
              \item \emph{AWS Aurora Serverless}: servizio serverless per la gestione di database SQL
          \end{itemize}
    \item Linguaggi/framework:
          \begin{itemize}
              \item \emph{Node.js}: per lo sviluppo di API Restful
              \item \emph{Typescript}: per lo sviluppo di una web app
              \item \emph{Swift}: per lo sviluppo di un'applicazione iOS
              \item \emph{Kotlin}: per lo sviluppo di un'applicazione per Android
          \end{itemize}
\end{itemize}

\subsubsection{Valutazione capitolato}
Il capitolato in questione costituisce la seconda scelta del gruppo, con i motivi elencati di seguito.

\paragraph{Fattori positivi}

L'azienda si dimostra molto disponibile a fornire assistenza per qualsiasi problema riscontrabile durante lo sviluppo del progetto; vengono inoltre proposti corsi di formazione su alcune delle tecnologie richieste.

\paragraph{Fattori negativi}

Secondo l'opinione del gruppo la tematica del capitolato risulta non molto interessante, in quanto sono già presenti sul mercato servizi simili.

\subsubsection{Riferimenti}
La presentazione del capitolato è reperibile all'indirizzo \url{\capLink} \hfill\break [Ultimo accesso: 28/10/2022]