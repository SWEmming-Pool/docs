% =============================================================================
% COMANDI COMUNI
% =============================================================================

% Informazini sul capitolato
\renewcommand{\capName}{\textit{Trustify}} %Nome capitolato
\renewcommand{\capCode}{C7} %Numero CX
\renewcommand{\capLink}{https://www.math.unipd.it/~tullio/IS-1/2022/Progetto/C7.pdf} %Link pagina prof
\renewcommand{\capProposer}{SyncLab} % Nome azienda

% =============================================================================
% STRUTTURA
% =============================================================================

\subsection{\capCode\ - \capName}
\textbf{Azienda proponente:} SyncLab
\subsubsection{Descrizione generale}
Il capitolato si propone di risolvere il problema del \textit{review bombing}, legando ogni possibile recensione su un dato sito di e-commerce a una specifica transazione generata (sul sito in questione).

\subsubsection{Tecnologie richieste}
Le tecnologie richieste per risolvere il problema appartengono in gran parte al campo \textit{Web3}; in particolare, è stato richiesto/consigliato l'utilizzo delle seguenti:
\begin{itemize}
    \item blockchain \textit{Ethereum-compatibile};
    \item linguaggio \textit{Solidity}
    \item framework \textit{Java Spring} per API REST;
    \item framework \textit{Angular};
    \item librerie \texttt{web3js} e \texttt{web3j};
    \item fornitore terzo RPC (es. Infura, Moralis ecc.);
    \item wallet MetaMask.
\end{itemize}
\subsubsection{Valutazione capitolato}
Dalle discussioni tra membri del gruppo è emerso che questo capitolato costituirà la scelta definitiva; la valutazione ha tenuto conto dei fattori positivi e negativi riportati di seguito.

\paragraph{Fattori positivi}

 Dal nostro punto di vista, le parti più stimolanti di questo capitolato riguardano l'avanguardia delle tecnologie richieste: la maggior parte dei componenti del gruppo non ha mai avuto opportunità di lavorare in ambito Web3 e blockchain, e riteniamo che questa possa essere l'occasione migliore per imparare ad interfacciarsi con tali strumenti.

\paragraph{Fattori negativi}

Non abbiamo trovato particolari punti negativi in questo capitolato, e per questo motivo possiamo confermare che sceglieremo questo come progetto da affrontare.
\subsubsection{Riferimenti}
La presentazione del capitolato è reperibile all'indirizzo \url{\capLink}