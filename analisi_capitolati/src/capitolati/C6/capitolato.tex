% =============================================================================
% COMANDI COMUNI
% =============================================================================

% Informazini sul capitolato
\renewcommand{\capName}{ShowRoom3D} %Nome capitolato
\renewcommand{\capCode}{C6} %Numero CX
\renewcommand{\capLink}{https://www.math.unipd.it/~tullio/IS-1/2022/Progetto/C6.pdf} %Link pagina prof
\renewcommand{\capProposer}{SanMarco Informatica} % Nome azienda

% =============================================================================
% STRUTTURA
% =============================================================================

\section{\capCode\ - \capName}
\textbf{Azienda proponente:} Sanmarco Informatica SPA
\subsection{Descrizione generale}
Il capitolato di questa azienda punta a risolvere alcune problematiche intrinseche degli showroom tradizionali quali costi, distanze fisiche dai clienti e periodi di chiusura. Per risolvere tali problemi, si vuole creare un sistema di showroom virtuali navigabili via web dagli utenti.

\subsection{Tecnologie richieste}
L'azienda consiglia di utilizzare la libreria JavaScript Three.js per la visualizzazione di oggetti 3D via web, ma consente in alternativa l'utilizzo di Unity o Unreal Engine.

\subsection{Valutazione capitolato}
Questo capitolato non rientra tra quelli di cui effettuare un analisi più approfondita per una eventuale scelta; nessun componente del gruppo ha infatti esperienze pregresse con la modellazione 3D nè è interessato a intraprenderne in questo momento.

\subsection{Riferimenti}
La presentazione del capitolato è reperibile all'indirizzo \url{\capLink}