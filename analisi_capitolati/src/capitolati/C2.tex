% =============================================================================
% COMANDI COMUNI
% =============================================================================

% Informazioni sul capitolato
\renewcommand{\capName}{\textit{Lumos Minima}} %Nome capitolato
\renewcommand{\capCode}{C2} %Numero CX
\renewcommand{\capLink}{https://www.math.unipd.it/~tullio/IS-1/2022/Progetto/C2.pdf} %Link pagina prof
\renewcommand{\capProposer}{ImolaInformatica} % Nome azienda

% =============================================================================
% STRUTTURA
% =============================================================================

\subsection{\capCode\ - \capName}
\textbf{Azienda proponente:} Imola Informatica
\subsubsection{Descrizione generale}
Con questo progetto, l'azienda proponente si pone l'obiettivo di trovare una soluzione ai consumi energetici, che negli ultimi anni hanno avuto un tasso di crescita continua. Il capitolato propone lo sviluppo di un applicazione web responsive per gestire gli impianti elettrici andando a ridurre in modo “intelligente” i consumi di questi.
\subsubsection{Tecnologie richieste}
Il capitolato non menziona tecnologie specifiche, ad eccezione dei dispositivi \textit{Internet of Things}; date le restanti richieste si deduce però che saranno necessari linguaggi e framework che permettano lo sviluppo di una web app responsive, in quanto gli operatori del servizio saranno dotati di dispositivi mobile Android o iOS.

\subsubsection{Valutazione capitolato}
Il capitolato è stato inizialmente considerato come terza scelta e si è organizzato un incontro con i proponenti. Sono stati valutati i fattori riportati di seguito.

\paragraph{Fattori positivi}

\begin{itemize}
    \item L'obiettivo finale del capitolato risulta interessante in quanto si propone di attenuare uno dei principali problemi della società odierna, cioè il consumo energetico.
    \item Sono risultati interessanti anche il supporto e i materiali forniti dall'azienda (macchine virtuali, dispositivi IoT).
\end{itemize}

\paragraph{Fattori negativi}
La presenza di IoT ed elementi di elettronica rende il progetto meno appetibile dal nostro punto di vista.

\subsubsection{Riferimenti}
La presentazione del capitolato è reperibile all'indirizzo \url{\capLink} \hfill\break [Ultimo accesso: \today]