% =============================================================================
% COMANDI COMUNI
% =============================================================================

% Informazioni sul capitolato
\renewcommand{\capName}{\textit{Personal Identity Wallet}} %Nome capitolato
\renewcommand{\capCode}{C3} %Numero CX
\renewcommand{\capLink}{https://www.math.unipd.it/~tullio/IS-1/2022/Progetto/C3.pdf} %Link pagina prof
\renewcommand{\capProposer}{InfoCert} % Nome azienda

% =============================================================================
% STRUTTURA
% =============================================================================

\subsection{\capCode\ - \capName}
\textbf{Azienda proponente:} InfoCert
\subsubsection{Descrizione generale}
Il capitolato presentato dall'azienda vuole porre le basi per lo sviluppo di un \textit{digital identity wallet} conforme a degli standard ben precisi riconosciuti a livello internazionale.

\subsubsection{Tecnologie richieste}
Non sono richieste tecnologie nè framework specifici, anche per l'attuale mancanza di protocolli a livello europeo.

\subsubsection{Valutazione capitolato}
Il capitolato, anche dopo un incontro con lo scopo di descrivere meglio il progetto proposto, non ha colpito a pieno l'interesse del gruppo. Si chiedeva di operare in un campo ancora troppo poco definito a livello di norme e protocolli.

\subsubsection{Riferimenti}
La presentazione del capitolato è reperibile all'indirizzo \url{\capLink} \hfill\break [Ultimo accesso: \today]