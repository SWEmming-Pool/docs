% =============================================================================
% COMANDI COMUNI
% =============================================================================

% Informazioni sul capitolato
\renewcommand{\capName}{CAPTCHA: Umano o Sovrumano?} %Nome capitolato
\renewcommand{\capCode}{C1} %Numero CX
\renewcommand{\capLink}{https://www.math.unipd.it/~tullio/IS-1/2022/Progetto/C1.pdf} %Link pagina prof
\renewcommand{\capProposer}{Zucchetti} % Nome azienda

% =============================================================================
% STRUTTURA
% =============================================================================

\subsection{\capCode\ - \capName}
\textbf{Azienda proponente:} Zucchetti
\subsubsection{Descrizione generale}
L'azienda richiede lo sviluppo di un sistema \emph{CAPTCHA} che verifichi se un utente sia umano o robot. \\
Per effettuare tale verifica si richiede di sviluppare un'applicazione web che permetta di eseguire un login comprensivo di tale controllo.

\subsubsection{Tecnologie richieste}
Nel capitolato sono proposti vari suggerimenti per la realizzazione di un sistema \emph{CAPTCHA}. \\
Per quanto riguarda alla pagina web di login è richiesto l'utilizzo di HTML, CSS e JavaScript per la parte di frontend, e Java o PHP per la parte di backend.

\subsubsection{Valutazione capitolato}
Questo capitolato non ha suscitato interesse da parte dei membri del gruppo, ed è stato quindi escluso da quelli da valutare per una eventuale candidatura.

\subsubsection{Riferimenti}
La presentazione del capitolato è reperibile all'indirizzo \url{\capLink} \hfill\break [Ultimo accesso: 28/10/2022]