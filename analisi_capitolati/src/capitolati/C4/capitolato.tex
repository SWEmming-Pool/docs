% =============================================================================
% COMANDI COMUNI
% =============================================================================

% Informazini sul capitolato
\renewcommand{\capName}{Piattaforma Localizzazione Testi} %Nome capitolato
\renewcommand{\capCode}{C4} %Numero CX
\renewcommand{\capLink}{https://www.math.unipd.it/~tullio/IS-1/2022/Progetto/C3.pdf} %Link pagina prof
\renewcommand{\capProposer}{zero12} % Nome azienda

% =============================================================================
% STRUTTURA
% =============================================================================

\section{\capCode\ - \capName} \footnote{Riferimenti: \url{\capLink}}
\subsection{Descrizione generale}
L'azienda richiede lo sviluppo di una piattaforma che permetta di gestire le traduzioni di testi in diverse lingue. \\
Tale piattaforma deve supportare l'architettura multi-tenant, ovvero deve essere in grado di erogare il servizio a più clienti, isolando le risorse di ogni cliente. \\
Inoltre l'azienda lascia la libertà di scelta, di una sola delle tre proposte, per quanto riguarda lato front-end. \\


\subsection{Tecnologie richieste}
L'azienda raccomanda le seguenti tecnologie, in quanto garantisce un supporto in caso di necessita' di assistenza:
\begin{itemize}
    \item \textbf{Server}: 
        \begin{itemize}
            \item \emph{AWS fargate}: servizio serverless per la gestione a container
            \item \emph{AWS Aurora Serverless}: servizio serverless per la gestione di database SQL
        \end{itemize}
    \item \textbf{Linguaggi di Programmazione}: 
        \begin{itemize}
            \item \emph{Node.js}: per lo sviluppo di API Restful
            \item \emph{Typescript}: per lo sviluppo di una webapp
            \item \emph{Swift}: per lo sviluppo di un'applicazione iOS
            \item \emph{Kotlin}: per lo sviluppo di un'applicazione per Android
        \end{itemize}
\end{itemize}

\subsection{Valutazione capitolato}
Dalle discussioni tra membri del gruppo è emerso che il capitolato in questione ricade come scelta secondaria, dovuti ai seguenti motivi:

\subsubsection{Fattori positivi}

\begin{itemize}
    \item \textbf{Disponibilità dell'azienda}: l'azienda è disponibile a fornire assistenza per qualsiasi problema riscontrabile durante lo sviluppo del progetto. \\
    Inoltre è in grado di fornire corsi di formazione su alcune delle tecnologie richieste.
\end{itemize}

\subsubsection{Criticità e fattori negativi}

\begin{itemize}
    \item \textbf{Tematica del capitolato}: per l'opinione del gruppo la tematica del capitolato non è risultata molto interessante, in quanto sono già presenti sul mercato servizi simili.
\end{itemize}

\subsection{Riferimenti}
La presentazione del capitolato è reperibile all'indirizzo \url{\capLink}