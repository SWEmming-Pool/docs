% =============================================================================
% COMANDI COMUNI
% =============================================================================

% Informazini sul capitolato
\renewcommand{\capName}{\textit{Lumos Minima}} %Nome capitolato
\renewcommand{\capCode}{C2} %Numero CX
\renewcommand{\capLink}{https://www.math.unipd.it/~tullio/IS-1/2022/Progetto/C2.pdf} %Link pagina prof
\renewcommand{\capProposer}{ImolaInformatica} % Nome azienda

% =============================================================================
% STRUTTURA
% =============================================================================

\section{\capCode\ - \capName} 
\subsection{Descrizione generale}
L'Azienda Imola Informatica, con questo progetto, si pone l'obiettivo di trovare una soluzione ai consumi energetici, che negli ultimi anni hanno avuto un tasso di crescita continua. Il capitolato propone lo sviluppo di un applicazione web responsive per gestire gli impianti elettrici andando a ridurre in modo “intelligente” i consumi di questi.
\subsection{Tecnologie richieste}
\begin{itemize}
    \item MQTT
    \item Internet of Things
\end{itemize}

\subsection{Valutazione capitolato}
Il capitolato è stato considerato come terza scelta, e sono stati valutati i fattori riportati di seguito.

\subsubsection{Fattori positivi}

\begin{itemize}
    \item L'obiettivo finale del capitolato risulta interessante in quanto si propone di attenuare uno dei principali problemi della società odierna, cioè il consumo energetico.
    \item Sono interessanti anche i materiali forniti dall'azienda macchine virtuali per l'utilizzo dei componenti applicativi, dispositivi di Internet of Things per aiutare il team a comprendere ed apprendere queste nuove tecnologie, supporto da parte di esperti.
\end{itemize}

\subsubsection{Criticità e fattori negativi}
La presenza di IoT ed elementi di elettronica non rende il progetto interessante dal nostro punto di vista.

\subsection{Riferimenti}
La presentazione del capitolato è reperibile all'indirizzo \url{\capLink}