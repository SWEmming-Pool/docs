% =============================================================================
% COMANDI COMUNI
% =============================================================================

% Informazini sul capitolato
\renewcommand{\capName}{\textit{SmartLog}} %Nome capitolato
\renewcommand{\capCode}{C5} %Numero CX
\renewcommand{\capLink}{https://www.math.unipd.it/~tullio/IS-1/2022/Progetto/C5.pdf} %Link pagina prof
\renewcommand{\capProposer}{Socomec} % Nome azienda

% =============================================================================
% STRUTTURA
% =============================================================================

\subsection{\capCode\ - \capName}
\textbf{Azienda proponente:} Socomec
\subsubsection{Descrizione generale}
Il sistema proposto dall'azienda permette di analizzare contemporaneamente eventi riguardanti un insieme di apparecchiature per il supporto all'energia di servizi critici, in modo da individuare (e successivamente risolvere) eventuali problemi. In particolare, sono richieste due applicazioni: una per analizzare singoli file di \textit{log}, e una per estrarre informazioni statistiche da un insieme di file di \textit{log}.
\subsubsection{Tecnologie richieste}
L'azienda non impone l'utilizzo di nessuna tecnologia in particolare, ma consiglia l'utilizzo del linguaggio Python per la parte di analisi dei dati.
\subsubsection{Valutazione capitolato}
Questo capitolato non è risultato interessante per il gruppo, in quanto si tratta di un sistema applicato ad ambiti che non appassionano particolarmente nessuno dei membri.

\subsubsection{Riferimenti}
La presentazione del capitolato è reperibile all'indirizzo \url{\capLink}