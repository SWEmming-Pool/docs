\section{Processi Organizzativi}
    \subsection{Gestione dei Processi}
        \subsubsection{Scopo}
            L'obiettivo di tale sezione è quello di definire e gestire i processi di progetto secondo lo standard \textbf{ISO/IEC 12207:1995}. \\
            Nello specifico:
            \begin{itemize}
                \item stabilire il lavoro da svolgere in modo da raggiungere gli obiettivi del progetto;
                \item stabilire quale modello di sviluppo adottare;
                \item definire scadenze, assegnare i compiti, quantificare i rischi e le risorse necessarie;
                \item stabilire i criteri di qualità e di verifica;
                \item definire le modalità e strumenti di comunicazione;
            \end{itemize}

        \subsubsection{Attività di gestione}
            \begin{itemize}
                \item Definizione dell'obiettivo;
                \item Istanziazione dei processi;
                \item Pianificazione delle scadenze, costi e risorse;
                \item Assegnazione dei compiti e ruoli;
                \item Esecuzione dei processi;
                \item Revisione e valutazione dei risultati;
            \end{itemize}

        \subsubsection{Gestione delle comunicazioni}
            \paragraph{Comunicazione interne} ~
            \begin{itemize}
                \item \textbf{Comunicazione orale}: la comunicazione orale è il metodo più veloce e diretto di comunicazione. È importante che la comunicazione orale sia chiara e concisa, in modo da evitare equivoci e malintesi. \\
                Lo strumento adottato dal gruppo è \textbf{Discord}.
                \item \textbf{Comunicazione via chat}: la comunicazione via chat è utile per comunicare informazioni urgenti, per chiarire dubbi e fissare incontri.  \\
                Lo strumento adottato dal gruppo è \textbf{Telegram}.
            \end{itemize}

            \paragraph{Comunicazione esterne} ~
            
            Lo strumento principale per le comunicazione con soggetti esterni al gruppo è la posta elettronica all'indirizzo \groupMail. \\
            La comunicazione via mail è utile per chiarire dubbi e fissare incontri.  \\
            L'azienda \companyName ha fornito un canale \texttt{Discord} apposito per le comunicazioni orali con il gruppo.
        

    \subsection{Gestione delle Infrastrutture}

    \subsection{Miglioramento del Processo}

    \subsection{Formazione del Personale}
