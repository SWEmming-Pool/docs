\section{Processi Organizzativi}
    \subsection{Gestione dei Processi}
        \subsubsection{Scopo}
            L'obiettivo di tale sezione è quello di definire e gestire i processi di progetto secondo lo standard \textbf{ISO/IEC 12207:1995}. \\
            Nello specifico:
            \begin{itemize}
                \item stabilire il lavoro da svolgere in modo da raggiungere gli obiettivi del progetto;
                \item stabilire quale modello di sviluppo adottare;
                \item definire scadenze, assegnare i compiti, quantificare i rischi e le risorse necessarie;
                \item stabilire i criteri di qualità e di verifica;
                \item definire le modalità e strumenti di comunicazione.
            \end{itemize}

        \subsubsection{Attività di gestione}
            \begin{itemize}
                \item Definizione dell'obiettivo;
                \item Istanziazione dei processi;
                \item Pianificazione delle scadenze, costi e risorse;
                \item Assegnazione dei compiti e ruoli;
                \item Esecuzione dei processi;
                \item Revisione e valutazione dei risultati;
            \end{itemize}

        \subsubsection{Gestione delle comunicazioni}
            \begin{itemize}
                \item \textbf{Comunicazioni interne}: comunicazioni tra i membri del gruppo;
                \begin{itemize}
                    \item \textbf{Comunicazione orale}: la comunicazione orale è il metodo più veloce e diretto di comunicazione. È importante che questo tipo di comunicazione sia chiara e concisa, in modo da evitare equivoci e malintesi. \\
                    Lo strumento adottato dal gruppo a tale scopo è \textit{Discord}.
                    \item \textbf{Comunicazione via chat}: la comunicazione via chat è utile per comunicare informazioni urgenti, chiarire dubbi e fissare incontri.  \\
                    Lo strumento adottato dal gruppo è \textit{Telegram}.
                \end{itemize}
                \item \textbf{Comunicazioni esterne}: comunicazioni con il committente e con il proponente. \\
                Lo strumento principale per le comunicazione con soggetti esterni al gruppo è la posta elettronica all'indirizzo \groupMail. \\
                La comunicazione via mail è utile per chiarire dubbi e fissare incontri.  \\
                Inoltre l'azienda \companyName ha fornito un canale \textit{Discord} apposito per le comunicazioni via chat con il gruppo.
            
            \end{itemize}

        \subsubsection{Gestione delle riunioni}
            \begin{itemize}
                \item \textbf{Riunioni interne}: è compito del responsabile di progetto, in concordanza con il gruppo, organizzare le riunioni interne. \\
                Gli incontri possono affrontare argomenti di vario tipo: definizione delle attività da svolgere, allineamento su ciò che è stato fatto con revisione e validazione, ecc. \\
                Si ritiene valida una riunione quando sono presenti almeno quattro membri del gruppo.
                    \begin{itemize}
                        \item \textbf{Verbali}: viene redatto un verbale dopo ogni riunione in cui si sono approvate le attività svolte nel periodo precedente.
                    \end{itemize}
                \item \textbf{Riunioni esterne}: è compito del responsabile di progetto, in concordanza con il gruppo e il proponente, organizzare le riunioni esterne.
                \begin{itemize}
                    \item \textbf{Verbali}: viene redatto un verbale al termine di ogni riunione esterna in modo da riportarne il contenuto, che dovrà poi essere approvato dal responsabile.
                \end{itemize}
            \end{itemize}

    \subsection{Gestione delle Infrastrutture}
        \subsubsection{Ticketing}
        \subsubsection{Gestione dei Rischi}
        \subsubsection{Strumenti}
            \begin{itemize}
                \item \textbf{Telegram}: per le comunicazione rapide tra i membri del gruppo;
                \item \textbf{Discord}: per le riunioni (interne ed esterne);
                \item \textbf{GSuite}: per la gestione delle email, delle bozze di documenti, diari di bordo e del calendario;
                \item \textbf{Git}: per il controllo di versionamento;
                \item \textbf{GitHub}: per la condivisione e salvataggio remoto del codice sorgente e documentazione;
            \end{itemize}
            
            A livello di sistema operativo non sono presenti vincoli prefissati dal proponente, pertanto ogni membro del gruppo utilizza il sistema operativo con cui ha maggior familiarità.
            Vengono utilizzati \textit{Windows}, \textit{Linux} e \textit{MacOS}.

        \subsubsection{Configurazione degli Strumenti}
            Il gruppo come prima cosa ha creato un canale \textit{Discord} per le comunicazioni.
            Successivamente ha creato un account \textit{Google} per poter usufruire dei vari servizi offerti da \textit{GSuite}, quali \textit{Gmail}, \textit{Google Drive} e \textit{Google Calendar}.
            Infine è stato creato una organizzazione del gruppo su \textit{GitHub}, nella quale si è deciso di creare tre \textit{repository}:
                \begin{itemize}
                    \item \textbf{Documentazione}: per la condivisione della documentazione definitiva;
                    \item \textbf{docs}: per gestire la documentazione in fase di redazione;
                    \item \textbf{DA DEFINIRE}: per la condivisione del codice sorgente.
                \end{itemize}

    \subsection{Miglioramento del Processo}

    \subsection{Formazione del Personale}
        Ciascun membro del gruppo è tenuto a provvedere alla propria formazione, eventualmente con l'aiuto degli altri componenti del team, al fine di garantire una qualità di lavoro adeguata.
        Nello specifico il gruppo dovrà fare riferimento alla seguente documentazione;
                \begin{itemize}
                    \item {\LaTeX}: \href{https://www.overleaf.com/learn}{https://www.overleaf.com/learn};
                    \item \textit{Git}: \href{https://git-scm.com/docs}{https://git-scm.com/docs};
                    \item \textit{GitHub}: \href{https://support.github.com}{https://support.github.com};
                \end{itemize}
