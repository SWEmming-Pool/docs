\section{Processi Organizzativi}
    \subsection{Gestione Organizzativa}
        \subsubsection{Scopo}
            L'obiettivo di tale sezione è quello di definire e gestire i processi di progetto secondo lo standard \textbf{ISO/IEC 12207:1995}. \\
            Nello specifico:
            \begin{itemize}
                \item Stabilire il lavoro da svolgere in modo da raggiungere gli obiettivi del progetto;
                \item Stabilire quale modello di sviluppo adottare;
                \item definire scadenze, assegnare i compiti, quantificare i rischi e le risorse necessarie;
                \item Stabilire i criteri di qualità e di verifica;
                \item Definire le modalità e strumenti di comunicazione;
                \item Definire i ruoli di progetto e i relativi compiti.
            \end{itemize}

        \subsubsection{Attività di gestione}
            \begin{itemize}
                \item Definizione dell'obiettivo;
                \item Istanziazione dei processi;
                \item Pianificazione delle scadenze, costi e risorse;
                \item Assegnazione dei compiti e ruoli;
                \item Esecuzione dei processi;
                \item Revisione e valutazione dei risultati;
            \end{itemize}

        \subsubsection{Gestione delle comunicazioni}
            \begin{itemize}
                \item \textbf{Comunicazioni interne}: comunicazioni tra i membri del gruppo;
                \begin{itemize}
                    \item \textbf{Comunicazione orale}: la comunicazione orale è il metodo più veloce e diretto di comunicazione. È importante che questo tipo di comunicazione sia chiara e concisa, in modo da evitare equivoci e malintesi. \\
                    Lo strumento adottato dal gruppo a tale scopo è \textit{Discord}.
                    \item \textbf{Comunicazione via chat}: la comunicazione via chat è utile per comunicare informazioni urgenti, chiarire dubbi e fissare incontri.  \\
                    Lo strumento adottato dal gruppo è \textit{Telegram}.
                \end{itemize}
                \item \textbf{Comunicazioni esterne}: comunicazioni con il committente e con il proponente. \\
                Lo strumento principale per le comunicazione con soggetti esterni al gruppo è la posta elettronica all'indirizzo \groupMail. \\
                La comunicazione via mail è utile per chiarire dubbi e fissare incontri.  \\
                Inoltre l'azienda \companyName\ ha fornito un canale \textit{Discord} apposito per le comunicazioni via chat con il gruppo.
                La comunicazione via mail viene usata per le comunicazioni formali. \\
            
            \end{itemize}

        \subsubsection{Gestione delle riunioni}
            \begin{itemize}
                \item \textbf{Riunioni interne}: è compito del responsabile di progetto, in concordanza con il gruppo, organizzare le riunioni interne. \\
                Gli incontri possono affrontare argomenti di vario tipo: definizione delle attività da svolgere, allineamento su ciò che è stato fatto con revisione e validazione, ecc. \\
                Si ritiene valida una riunione quando sono presenti almeno quattro membri del gruppo.
                    \begin{itemize}
                        \item \textbf{Verbali}: viene redatto un verbale dopo ogni riunione in cui si sono approvate le attività svolte nel periodo precedente.
                    \end{itemize}
                \item \textbf{Riunioni esterne}: è compito del responsabile di progetto, in concordanza con il gruppo e il proponente, organizzare le riunioni esterne, che avranno luogo su \textit{Google Meet}.
                \begin{itemize}
                    \item \textbf{Verbali}: viene redatto un verbale al termine di ogni riunione esterna in modo da riportarne il contenuto, che dovrà poi essere approvato dal responsabile.
                \end{itemize}
            \end{itemize}
        \subsubsection{Ruoli di progetto}
        Di seguito vengono descritti i ruoli di progetto e le relative responsabilità:

        \paragraph{Responsabile di progetto} ~

        Il responsabile di progetto e la figura principale in quanto deve pianificare e coordinare tutte le attività del gruppo. Deve inoltre occuparsi delle comunicazioni con il proponente e con il committente, gestire le risorse e dare l'approvazione finale dei documenti.

        \paragraph{Amministratore di progetto} ~

        L’Amministratore di Progetto è incaricato di gestire, controllare e curare le risorse e l'ambiente di lavoro.\\
        Si occupa della redazione del documento \textit{Norme di Progetto} e si assicura della messa in opera delle regole definite al suo interno.\\
        Infine deve tenere traccia del versionamento del prodotto.

        \paragraph{Analista} ~

        L'analista è un ruolo essenziale nella fase iniziale in cui viene analizzato e studiato il problema in modo da definirne i requisiti, le funzionalità e le caratteristiche. Deve dunque avere una conoscenza del dominio applicativo e una notevole esperinza personale.\\
        L'analista deve inoltre occuparsi della redazione del documento \textit{Analisi dei Requisiti}.

        \paragraph{Progettista} ~

        Il progettista segue il lavoro dell'analista, occupandosi di trovare una soluzione al problema che soddisfi i requisiti definiti. Deve dunque avere una competenza tecnica e tecnologica aggiornata.

        \paragraph{Programmatore} ~

        Il programmatore si occupa della realizzazione del prodotto, codificandone le soluzioni stabilite dal progettista.
        È inoltre responsabile di definirne i test per verificarne la correttezza l'implementazione.

        \paragraph{Verificatore} ~

        Il verificatore è un ruolo che accompagna tutto lo svolgimento del progetto, in quanto si occupa di verificare e validare il prodotto in modo che soddisfi le norme e le attese prefissate.
        \bigskip
        \begin{flushleft}
        A fini accademici ciascun membro del gruppo deve ricoprire ciascun ruolo, definiti in precedenza, per almeno una volta.
        \end{flushleft}


    \subsection{Gestione delle Infrastrutture}
        \subsubsection{Ticketing} 
        Il ticketing è un sistema di gestione delle attività che permette di tenere traccia delle attività svolte e di quelle da svolgere. In questa maniera il responsabile di progetto ha sempre sotto occhio lo stato di avanzamento del progetto. \\
        Si è scelto di usare lo strumento fornito da \textit{GitHub} per la gestione dei ticket; tramite l'integrazione tra l'\textit{Issue Tracking} e le sue \textit{Project Board}, questo si presenta sotto forma di tabella in cui ciascuna attività si trova necessariamente in uno stato definito tra quelli elencati di seguito:
        \begin{itemize}
            \item \textbf{To Do}: attività da svolgere;
            \item \textbf{In Progress}: attività in corso di svolgimento;
            \item \textbf{Review}: attività in attesa di revisione;
            \item \textbf{Done}: attività completata.
        \end{itemize}

        \subsubsection{Gestione dei Rischi} 
        Il responsabile di progetto deve tenere traccia dei possibili rischi e delle relative azioni correttive.
        Le tipologie di rischio posso essere:
        \begin{itemize}
            \item \textbf{RT: Rischi Tecnologici}: sono quelli che possono essere associati a una tecnologia o a un processo di sviluppo;
            \item \textbf{RO: Rischi Organizzativi}: sono quelli che possono essere associati a una mancanza di organizzazione o di pianificazione;
            \item \textbf{RI: Rischi Interpersonali}: sono quelli che possono essere associati a una mancanza di comunicazione o di collaborazione tra i membri del gruppo;
        \end{itemize}
        \subsubsection{Strumenti}
            \begin{itemize}
                \item \textit{Telegram}: per le comunicazione rapide tra i membri del gruppo;
                \item \textit{Discord}: per le riunioni (interne ed esterne);
                \item \textit{GSuite}: per la gestione delle email, delle bozze di documenti, diari di bordo e del calendario;
                \item \textit{Git}: per il controllo di versionamento;
                \item \textit{GitHub}: per la condivisione e salvataggio remoto del codice sorgente e documentazione;
            \end{itemize}
            
            A livello di sistema operativo non sono presenti vincoli prefissati dal proponente, pertanto ogni membro del gruppo utilizza il sistema operativo con cui ha maggior familiarità.
            Vengono utilizzati \textit{Windows}, \textit{Linux} e \textit{MacOS}.

        \subsubsection{Configurazione degli Strumenti}
            Il gruppo come prima cosa ha creato un canale \textit{Discord} per le comunicazioni.
            Successivamente ha creato un account \textit{Google} per poter usufruire dei vari servizi offerti da \textit{GSuite}, quali \textit{Gmail}, \textit{Google Drive} e \textit{Google Calendar}.
            Infine è stato creato una organizzazione del gruppo su \textit{GitHub}, nella quale si è deciso di creare dei \textit{repository} divisi tra pubblici e privati:
            \begin{itemize}
                \item Repository pubbliche
                \begin{itemize}
                    \item \textbf{Documentazione}: per la condivisione della documentazione definitiva;
                    \item \textbf{PoC}: per la condivisione del codice sorgente del \textit{Proof of Concept} definitivo;
                \end{itemize}
                \item Repository private
                \begin{itemize}
                    \item \textbf{docs}: per gestire la documentazione in fase di redazione;
                    \item \textbf{poc-smartcontract}: per gestire la codifica della parte legata allo smart contract del PoC;
                    \item \textbf{poc-frontend}: per gestire la codifica della parte legata al frontend del PoC;
                    \item \textbf{poc-api}: per gestire la codifica della parte legata all'API del PoC;
                \end{itemize}
            \end{itemize}
            L'indirizzo della repository è \href{https://github.com/SWEmming-Pool}{https://github.com/SWEmming-Pool}.

        \paragraph{Tipi di file}
        Le cartelle pubbliche di documentazione contengono solamente \textit{pdf}. Nelle cartelle utilizzate per la redazione sono contenuti anche i vari file \textit{.tex} e le immagini a supporto dei documenti, oltre ai file relativi agli strumenti di sviluppo come ad esempio i \textit{.gitignore} per nascondere i file superflui o configurazione delle \textit{GitHub Action} e gli script di build o per la gestione dei container per uno sviluppo su ambiente isolato e riproducibile. \\
        Le cartelle contenenti codice sorgente per i prodotti software seguiranno la stessa metodologia applicata per i documenti, in modo da contenere strumenti supporto come script di build e container, oltre ai classici file di configurazione di \textit{GitHub} come \textit{.gitignore} e \textit{GitHub Action}.


    \subsection{Formazione del Personale}
        Ciascun membro del gruppo è tenuto a provvedere alla propria formazione, eventualmente con l'aiuto degli altri componenti del team, al fine di garantire una qualità di lavoro adeguata.
        Nello specifico il gruppo dovrà fare riferimento alla seguente documentazione;
                \begin{itemize}
                    \item \textit{LaTeX}:
                        \begin{itemize}
                            \item  \href{https://www.overleaf.com/learn}{overleaf};
                            \item \href{http://www.lorenzopantieri.net/LaTeX_files/LaTeXpedia.pdf}{LaTeXpedia};
                        \end{itemize}
                    \item \textit{Git}: \href{https://git-scm.com/docs}{documentazione ufficiale};
                    \item \textit{GitHub}: \href{https://support.github.com}{documentazione ufficiale};
                \end{itemize}