\section{Introduzione}\
\subsection{Scopo del documento}
Il seguente documento ha l'obiettivo di definire le linee guida per tutti i
processi del gruppo \groupName. \\ Al suo interno sono presenti le norme, le
tecnologie e gli strumenti che il team intende adottare. \\ Ogni membro del
gruppo si impegna ad adottare queste “misure” per lo svolgimento di ogni
attività finalizzata al progetto.

\subsection{Scopo del progetto}
Al giorno d'oggi poter garantire l'autenticità e la veridicità di una
recensione\glo \: risulta essere un problema. \\ Il capitolato \textit{Trustify} si
pone come obiettivo quello di creare un servizio che, attraverso l'uso di uno
\textit{smart contract}\glo \:, permetta di effettuare una recensione collegata in
modo univoco ad un pagamento. \\ Il prodotto atteso sarà composto da un
contratto digitale e una web app che, attraverso il wallet \textit{MetaMask}\glo \:, ne
permetta l'interazione con esso, e da un server \textit{API REST}\glo \: che consenta ad un
e-commerce\glo \: di mostrare le recensioni ricevute all'interno del proprio sito.

\subsection{Glossario}
Al fine di evitare ambiguità nella terminologia usata all'interno del seguente
documento è stato redatto un glossario, in cui vengono riportate le definizioni
di termini tecnici, rilevanti o con un significato particolare. \\ Per indicare
la presenza di un termine all'interno del glossario si è scelto di
contrassegnarlo con \glo .\\ Per non appesantire la lettura della documentazione
verrà così contrassegnata solo la prima occorrenza di ogni termine in ciascun
documento.

Per una consultazione completa si rimanda al \textit{Glossario v1.0.0}.

\subsection{Riferimenti}
\subsubsection{Riferimenti normativi}
\begin{itemize}
    \item Capitolato d'appalto C7: \textbf{Trustify - Authentic and verifiable reviews platform}: \\
          \url{https://www.math.unipd.it/~tullio/IS-1/2022/Progetto/C7.pdf}
          \hfill\break [Ultimo accesso: \today]
\end{itemize}
\subsubsection{Riferimenti informativi}
\begin{itemize}
    \item \textbf{Standard ISO/IEC 12207:1995}: \url{https://www.math.unipd.it/~tullio/IS-1/2009/Approfondimenti/ISO_12207-1995.pdf} \hfill\break [Ultimo accesso: \today];
    \item \textbf{Piano di Progetto} - \textit{Piano di Progetto - v1.0.0};
    \item \textbf{Piano di Qualifica} - \textit{Piano di Qualifica - v1.0.0};
    \item \textbf{Software Engineering - Ian Sommerville - 9th Edition (2010)}.
\end{itemize}
