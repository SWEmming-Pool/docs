\section{Standard ISO/IEC 12207:1997}
La norma ISO/IEC 12207:1997 è uno standard internazionale che fornisce linee guida per la gestione del ciclo di vita del software. Lo standard definisce un insieme di processi, attività e ruoli per lo sviluppo, la manutenzione e la gestione del software, e si applica a tutti i tipi di software, inclusi i sistemi informativi, i software embedded e i software di controllo.\\
I processi sono divisi in tre categorie:
\begin{itemize}
\item Processi primari;
\item Processi organizzativi;
\item Processi di supporto.
\end{itemize}

\subsection{Processi Primari}
Comprendono le attività direttamente legate allo sviluppo del software.
Le attività sono le seguenti:
\begin{itemize}
\item Acquisition;
\item Supply;
\item Development;
\item Maintenance.
\end{itemize}

\subsubsection{Acquisition}
Questo processo ha lo scopo di ottenere il prodotto richiesto dal cliente ed è suddiviso nelle seguenti
attività:
\begin{itemize}
    \item Preparazione dell’acquisizione;
    \item Scelta del fornitore;
    \item Monitoraggio dei fornitori;
    \item Accettazione del cliente.
\end{itemize}

\subsubsection{Supply}

Questo processo ha lo scopo di fornire al cliente il prodotto/servizio che soddisfa i requisiti concordati.\\

È suddiviso nelle seguenti attività:

\begin{itemize}

  \item Proposal preparation;

  \item Contract;

  \item Planning;

  \item Execution and control;

  \item Review and evaluation;

  \item Release and completion.

\end{itemize}


\subsubsection{Development}

Questo processo ha lo scopo di sviluppare un prodotto software che indirizzi le esigenze del cliente.\\

Le attività del processo sono suddivise rispetto al ruolo dello sviluppatore e a quello del cliente e sono le seguenti:

\begin{itemize}

  \item Requirements elicitation;

  \item System requirements analysis;

  \item System architecture design;

  \item Software requirements analysis;

  \item Software architecture design;

  \item Software construction (code and unit test);

  \item Software integration;

  \item Software testing;

  \item System integration;

  \item System testing;

  \item Software installation.

\end{itemize}

\subsubsection{Maintenance}

È svolto simultaneamente alla precedente fase di Esercizio.\\

Questo processo ha lo scopo di modificare il prodotto software dopo il suo rilasci per correggere i difetti, migliorare le sue prestazioni o altri attributi o adattarlo a cambiamenti nell'ambiente operativo.\\

È suddiviso nelle seguenti attività:

\begin{itemize}

  \item Defect or request for change analysis;

  \item Change implementation;

  \item Review/acceptance of changes;

  \item Migration;

  \item Software withdraw.

\end{itemize}


\subsection{Processi Organizzativi}
Sono fondamentali per garantire che l'organizzazione coinvolta nello sviluppo del software sia in grado di fornire un ambiente adeguato e supportare efficacemente gli altri processi del ciclo di vita del software.
Le attività sono le seguenti:
\begin{itemize}
    \item \textbf{Formazione del personale:} questo processo ha lo scopo di fornire all'organizzazione risorse umane adeguate e di mantenere le loro competenze consistenti con le necessità del business;
    \item \textbf{Miglioramento del processo:} questo processo ha lo scopo di stabilire, valutare, controllare e migliorare il ciclo di vita del software;
    \item \textbf{Gestione delle infrastrutture:} questo processo ha lo scopo di mantenere un'infrastruttura stabile ed affidabile, necessaria a supportare le prestazioni di qualsiasi processo.
    L'infrastruttura può includere hardware, software, metodi, tools, tecniche, standard ed utilità per lo sviluppo, operatività o manutenzione;
    
    \item \textbf{Gestione dei processi:} Questo processo ha lo scopo di organizzare, monitorare e controllare l'avvio e le prestazioni di un processo per il raggiungimento dei loro obiettivi in accordo con quelli di business dell'organizzazione.
    Il processo è stabilito da una organizzazione per assicurare la consistente applicazione di pratiche per l'uso dall'organizzazione e nei progetti.
    
    
    
\end{itemize}

\subsection{Processi di Supporto}
I processi di supporto forniscono supporto a questi processi primari, come la gestione della configurazione, la gestione della qualità e la gestione dei test. Infine, i processi organizzativi forniscono il supporto necessario all'organizzazione, come la gestione delle risorse umane e la gestione delle infrastrutture.\\
Le attività sono le seguenti:
\begin{itemize}
    \item Documentazione;
    \item Gestione delle versioni e della configurazione;
    \item Risoluzione dei problemi;
    \item Verifica e validazione.
    
\end{itemize}

\subsubsection{Gestione della documentazione}

Questo processo garantisce lo sviluppo e la manutenzione delle informazioni prodotte e registrate relativamente al prodotto software.


\subsubsection{Gestione della configurazione}

Questo processo ha lo scopo di definire e mantenere l'integrità di tutti i componenti della configurazione e di renderli accessibili a chi ne ha diritto.


\subsubsection{Gestione della qualità}

Questo processo ha lo scopo di assicurare che tutti i prodotti di fase siano conformi con i piani e gli standard definiti.


\subsubsection{Verifica}

Questo processo ha lo scopo di confermare che ciascun prodotto o servizio realizzato da un processo soddisfi i requisiti specificati.\\
Il processo di Verifica deve essere integrato nei processi di Sviluppo, Fornitura e Manutenzione.


\subsubsection{Validazione}

Questo processo ha lo scopo di confermare che i requisiti siano rispettati quando uno specifico prodotto sia utilizzato nell'ambiente destinatario.

\subsubsection{Risoluzione dei problemi}

Questo processo ha lo scopo di assicurare che tutti i problemi individuati siano analizzati e risolti secondo trend riconosciuti.


\subsubsection{Usabilità}

Questo processo ha lo scopo di assicurare che siano prese in considerazione, ed opportunamente indirizzate, le considerazioni espresse dalle parti interessate relativamente alla facilità d'uso del prodotto finale da parte degli utenti a cui è rivolto, al supporto che ne riceverà, alla formazione, all'incremento della produttività, alla qualità del lavoro, all'accettazione del prodotto stesso.


\subsubsection{Valutazione del prodotto}

Questo processo ha lo scopo principale di assicurare, tramite test e misure, che il prodotto soddisfi le necessità esplicite ed implicite degli utilizzatori del prodotto stesso.

\section{Standard di qualità ISO/IEC 9126}

\textbf{ISO/IEC 9126} è uno standard internazionale per la valutazione della qualità software.
Il gruppo \groupName{} ha deciso di fare riferimento a questo standard poiché considerato un caposaldo in materia di qualità.
Questo standard permette di diffondere una comprensione comune degli obiettivi di un progetto software.

\subsection{Modello della qualità del software} \label{subsection:modello_qualitaSW}
Il modello di qualità stabilito dallo standard si articola in sei caratteristiche principali, ognuna delle quali a sua volta presenta delle sottocaratteristiche, misurabili tramite delle metriche di qualità.
Di seguito sono esposte le sei caratteristiche sopra citate:
\begin{itemize}
    \item \textbf{Funzionalità:} insieme di attributi riguardanti un insieme di funzioni e le loro proprietà.
          Tali funzioni mirano a soddisfare requisiti stabiliti o implicitamente dedotti.
          Le sottocaratteristiche della funzionalità sono:
          \begin{itemize}
              \item \textbf{Adeguatezza:} capacità del prodotto di fornire un insieme di funzioni in grado di svolgere compiti e soddisfare obiettivi prefissati;
              \item \textbf{Accuratezza:} capacità del prodotto di fornire i risultati desiderati con la precisione richiesta;
              \item \textbf{Interoperabilità:} capacità del prodotto di interagire ed operare con uno o più sistemi;
              \item \textbf{Sicurezza:} capacità del prodotto di proteggere informazioni e dati monitorando gli accessi ad essi;
              \item \textbf{Aderenza alle funzionalità:} grado di adesione del prodotto agli standard scelti dal gruppo, alle convenzioni e ai regolamenti;
          \end{itemize}
    \item \textbf{Affidabilità:} insieme di attributi riguardanti la capacità del prodotto di mantenere un dato livello di performance sotto condizioni di esecuzione prestabilite.
          Le sottocaratteristiche dell'affidabilità sono:
          \begin{itemize}
              \item \textbf{Maturità:} capacità del prodotto di evitare errori e risultati non corretti durante l'esecuzione;
              \item \textbf{Tolleranza ai guasti:} capacità del prodotto di conservare il livello di prestazioni anche in caso di malfunzionamenti o di uso inappropriato del prodotto;
              \item \textbf{Recuperabilità:} capacità di un prodotto di ripristinare il livello di prestazioni e di recupero delle informazioni rilevanti, a seguito di un malfunzionamento. Il periodo di inaccessibilità del prodotto a seguito di un errore è valutato proprio dalla recuperabilità;
              \item \textbf{Aderenza all'affidabilità:} grado di adesione del prodotto a standard, regole e convenzioni inerenti all'affidabilità.
          \end{itemize}
    \item \textbf{Efficienza:} insieme di attributi riguardanti il rapporto tra il livello delle prestazioni e la quantità di risorse usate durante la loro esecuzione, sotto condizioni prestabilite.
          Le sottocaratteristiche dell'efficienza sono:
          \begin{itemize}
              \item \textbf{Comportamento rispetto al tempo:} capacità di un prodotto di fornire appropriati tempi di risposta e di elaborazione e quantità di lavoro eseguendo le funzionalità richieste in date condizioni di lavoro;
              \item \textbf{Utilizzo delle risorse:} capacità di un prodotto di usare appropriati numero e tipo di risorse durante la fase di esecuzione sotto condizioni di utilizzo date;
              \item \textbf{Aderenza all'efficienza:} grado di adesione del prodotto a standard, regole e convenzioni inerenti all'efficienza.
          \end{itemize}
    \item \textbf{Usabilità:} insieme di attributi riguardanti lo sforzo necessario all'utilizzo del prodotto e la valutazione individuale su tale uso da parte di un insieme di utenti.
          Le sottocaratteristiche dell'usabilità sono:
          \begin{itemize}
              \item \textbf{Comprensibilità:} facilità di comprensione dei concetti base del prodotto, per permettere all'utente di capire se il prodotto è appropriato;
              \item \textbf{Apprendibilità:} facilità con cui un utente medio può comprendere il funzionamento del prodotto ed imparare ad usarlo;
              \item \textbf{Operabilità:} misura della possibilità degli utenti di usare il prodotto in vari contesti e di adattarlo ai propri scopi;
              \item \textbf{Attrattività:} misura della gradevolezza e dell'essere "attraente" del prodotto durante l'uso;
              \item \textbf{Aderenza all'usabilità:} grado di adesione del prodotto a standard, regole e convenzioni inerenti all'usabilità.
          \end{itemize}
    \item \textbf{Manutenibilità:} insieme di attributi riguardanti lo sforzo richiesto per apportare modifiche specifiche al prodotto.
          Le sottocaratteristiche della manutenibilità sono:
          \begin{itemize}
              \item \textbf{Analizzabilità:} misura della difficoltà incontrata nel diagnosticare un errore nel prodotto;
              \item \textbf{Modificabilità:} facilità di apportare modifiche al prodotto originale o di introdurre nuove funzionalità; per modifiche si intendono cambiamenti al codice, alla progettazione o alla documentazione;
              \item \textbf{Stabilità:} capacità del prodotto di evitare effetti indesiderati a causa di modifiche;
              \item \textbf{Provabilità:} capacità del prodotto di essere verificato;
              \item \textbf{Aderenza alla manutenibilità:} grado di adesione del prodotto a standard, regole e convenzioni inerenti alla manutenibilità.
          \end{itemize}
    \item \textbf{Portabilità:} insieme di attributi riguardanti la capacità del software di essere trasferito da un ambiente di esecuzione ad un altro.
          Le sottocaratteristiche della portabilità sono:
          \begin{itemize}
              \item \textbf{Adattabilità:} capacità del prodotto di essere adattato per differenti ambienti operativi senza richiedere azioni specifiche diverse da quelle previste dal prodotto per quell'attività; include la scalabilità delle capacità interne del prodotto;
              \item \textbf{Installabilità:} capacità del prodotto di essere installato in un dato ambiente;
              \item \textbf{Coesistenza:} capacità di un prodotto di coesistere con altre applicazioni in ambienti comuni e condividere le risorse;
              \item \textbf{Sostituibilità:} capacità di sostituire un altro software specifico indipendente, per lo stesso scopo e nello stesso ambiente di sviluppo;
              \item \textbf{Aderenza alla portabilità:} capacità del prodotto di aderire a standard e convenzioni relative alla portabilità.
          \end{itemize}
\end{itemize}


\subsection{Qualità interna} \label{subsection:qualita_interna}
\subsubsection{Metriche per la qualità interna}
Le metriche interne si applicano al prodotto non eseguibile durante la progettazione e la codifica. Sono anche dette misure statiche.
Le misure effettuate permettono di prevedere il livello di qualità esterna ed in uso del prodotto finale, poiché gli attributi interni influiscono su quelli esterni e quelli in uso.
Le metriche interne permettono di individuare eventuali problemi che potrebbero inficiare la qualità finale del prodotto prima che sia realizzato il prodotto eseguibile.



\subsection{Qualità esterna} \label{subsection:qualita_esterna}
\subsubsection{Metriche per la qualità esterna}
Le metriche esterne misurano i comportamenti del prodotto sulla base dei test, dall'operatività e dall'osservazione durante la sua esecuzione, in funzione degli obiettivi stabiliti.
Le metriche esterne sono scelte in base alle caratteristiche che il prodotto finale dovrà dimostrare durante la sua esecuzione.

\subsection{Qualità in uso} \label{subsection:qualita_uso}
La qualità in uso rappresenta il punto di vista dell'utente sul prodotto. Il livello di qualità in uso è raggiunto quando sono state conseguite sia la qualità esterna che quella interna.
\subsubsection{Metriche per la qualità in uso}
La qualità in uso permette di abilitare specificati utenti ad ottenere dati obiettivi con efficacia, produttività, sicurezza e soddisfazione.
\begin{itemize}
    \item \textbf{Efficacia:} capacità del prodotto di consentire agli utenti di raggiungere gli obiettivi specificati con accuratezza e completezza;
    \item \textbf{Produttività:} capacità di consentire agli utenti di adoperare un'appropriata quantità di risorse rispetto all'efficacia ottenuta in uno dato contesto d'uso;
    \item \textbf{Soddisfazione:} capacità del prodotto di corrispondere alle aspettative degli utenti;
    \item \textbf{Sicurezza:} capacità del prodotto di raggiungere accettabili livelli di rischio di danni a persone, software, apparecchiature tecniche e all'ambiente d'uso.
\end{itemize}