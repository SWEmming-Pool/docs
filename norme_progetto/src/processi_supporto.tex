\section{Processi di Supporto}
\subsection{Scopo} In questa sezione vengono descritte le procedure per le seguenti fasi del progetto:
\begin{itemize}
    \item {documentazione;}
    \item {configurazione;}
    \item {qualità;}
    \item {verifica;}
    \item {validazione;}
    \item {controllo costi;}
    \item {risoluzione dei problemi.}
\end{itemize}
\subsection{Gestione della documentazione}
Vengono documentate tutte le informazioni riguardanti il ciclo di vita di ciascun processo o attività.
Queste attività si dividono nelle sezioni riportate di seguito.
\subsubsection{Implementazione}
Per tutti i documenti redatti vengono stabiliti titolo, scopo, destinatario,  procedure di tutti i ruoli e le scadenze per le revisioni intermedie e finali.
\subsubsection{Design e sviluppo}
\paragraph{Documentazione}~

\noindent È stato creato un \textit{template} per uniformare lo stile di tutti i documenti prodotti e agevolarne la produzione.\\
Nello specifico sono presenti:
\begin{itemize}
    \item \textbf{Prima pagina} contenente:
          \begin {itemize}
    \item logo;
    \item indirizzo email del gruppo;
    \item titolo del documento;
    \item responsabile, redattori e verificatori;
    \item uso (interno o esterno);
    \item destinatari;
    \item sommario.
\end{itemize}
\item \textbf{Registro delle modifiche}: in ogni documento viene inserita una tabella contenente il registro delle modifiche che riporta:
\begin{itemize}
    \item versione;
    \item data;
    \item autore;
    \item breve descrizione della modifica.
\end{itemize}
\item \textbf{Indice}: ha lo scopo di fornire una visione gerarchica dei contenuti del documento e di facilitare la ricerca di informazioni specifiche;
\item \textbf{Contenuto principale}: qui è presente l'effettivo contenuto del documento.
\end{itemize}

\paragraph{Verbali}~

\noindent È stato creato un template per uniformare lo stile di tutti i verbali prodotti e agevolarne la produzione.
Nello specifico sono presenti:
\begin{itemize}
    \item \textbf{Prima pagina} contenente:
          \begin {itemize}
    \item logo;
    \item indirizzo email del gruppo;
    \item titolo del documento;
    \item responsabile, redattori e verificatori;
    \item uso (interno o esterno);
    \item destinatari;
    \item sommario.
\end{itemize}
\item \textbf{Indice};
\item \textbf{Ordine} contenente:
\begin {itemize}
\item ora di inizio/termine;
\item luogo;
\item partecipanti interni e esterni;
\item ordine del giorno.
\end{itemize}
\item \textbf{Resoconto}: contiene una descrizione dettagliata degli argomenti trattati e delle decisioni prese.
\end{itemize}

\paragraph{Norme tipografiche}

\paragraph{Convenzioni nomenclatura file}~

\noindent Escludendo l'estensione, le regole di nomenclatura dei file sono le seguenti:
\begin{itemize}
    \item i nomi dei file devono essere completamente in minuscolo;
    \item viene utilizzato il carattere underscore \texttt{\_} per separare le parole;
    \item omissione delle preposizioni.
\end{itemize}

\paragraph{Formato data} ~

\noindent Viene utilizzato il formato \textbf{YYYY-MM-DD} per le date.
\begin{itemize}
    \item \textbf{YYYY} rappresenta l'anno (4 cifre);
    \item \textbf{MM} rappresenta il mese (2 cifre);
    \item \textbf{DD} rappresenta il giorno (2 cifre).
\end{itemize}

\paragraph{Elenchi} ~

\begin{itemize}
    \item \textbf{Elenco puntato}: viene utilizzato per elencare elementi in cui non ha importanza l'ordine;
    \item \textbf{Elenco ordinato}: viene utilizzato per elencare elementi in cui ha importanza l'ordine o la priorità.
\end{itemize}

\paragraph{Glossario} ~

\noindent Le convenzioni adottate per fare riferimento a termini specifici presenti nel glossario sono le seguenti:
\begin{itemize}
    \item un termine presente nel glossario viene marcato con \glo ;
    \item in ogni documento un termine viene marcato solo alla sua prima occorrenza.
\end{itemize}

\paragraph{Stile del testo} ~

\begin{itemize}
    \item \texttt{Monospaced}: viene utilizzato per i comandi e snippet;
    \item \textit{Italic}: viene utilizzato per i termini di dominio e per citare altri documenti;
    \item \textbf{Bold}: viene utilizzato per i termini da enfatizzare, da definire e per citare eventuali standard.
\end{itemize}

\paragraph{Sigle} ~

\noindent Nella stesura della documentazione vengono utilizzate alcune sigle per termini ricorrenti. Vengono scritte in maiuscolo le iniziali dei sostantivi e in minuscolo le iniziali delle eventuali preposizioni.

Le sigle adottate sono le seguenti:
\begin{itemize}
    \item \textbf{AdR}: \textit{Analisi dei Requisiti};
    \item \textbf{MU}: \textit{Manuale Utente};
    \item \textbf{MS}: \textit{Manuale Sviluppatore};
    \item \textbf{PdP}: \textit{Piano di Progetto};
    \item \textbf{PdQ}: \textit{Piano di Qualifica};
    \item \textbf{NdP}: \textit{Norme di Progetto};
    \item \textbf{ST}: \textit{Specifica Tecnica};
    \item \textbf{SdF}: \textit{Studio di Fattibilità};
    \item \textbf{RTB}: \textit{Requirements and Technology Baseline};
    \item \textbf{PB}: \textit{Product Baseline};
    \item \textbf{CA}: \textit{Customer Acceptance};
    \item \textbf{PoC}: \textit{Proof of Concept}.
\end{itemize}

Per definire le sigle dei ruoli si è deciso di aggiungere la seconda lettere in minuscolo per evitare omonimia:
\begin{itemize}
    \item \textbf{Am}: Amministratore;
    \item \textbf{An}: Analista;
    \item \textbf{Pr}: Programmatore;
    \item \textbf{Pt}: Progettista;
    \item \textbf{Re}: Responsabile di progetto;
    \item \textbf{Ve}: Verificatore.
\end{itemize}

\subsubsection{Produzione documenti}
Per agevolare la produzione dei documenti è stato creato un template che contiene tutte le informazioni necessarie per la redazione di un documento; a tale scopo è stata definita una cartella \texttt{componenti\_comuni} che contiene:
\begin{itemize}
    \item \texttt{src}: contiene i file \texttt{command.tex} e \texttt{packages.tex} che contengono comandi e pacchetti necessari per la produzione e compilazione del documento;
    \item \texttt{img}: contiene tutte le immagini utilizzate nel template.
\end{itemize}

\subsubsection{Elementi grafici}
\paragraph{Tabelle} ~

\noindent In ciascun documento prodotto, qualora fosse presente una tabella, essa viene accompagnata da una didascalia contenete la numerazione della tabella stessa e un breve titolo che ne riassuma il contenuto.

\paragraph{Immagini} ~

\noindent In ogni immagine inserita all'interno dei documenti prodotti viene fornita una breve didascalia per descriverne chiaramente il contenuto

\paragraph{Diagrammi UML} ~

\noindent I diagrammi \textit{UML} inseriti rispettano lo standard UML 2.0. Anch'essi vengono accompagnati da una breve didascalia.

\subsubsection{Strumenti}
Per la redazione dei documenti vengono utilizzati i seguenti strumenti:
\begin{itemize}
    \item \LaTeX: per la redazione dei documenti;
    \item \textit{VSCode}: per la gestione del codice sorgente e della documentazione;
    \item \textit{StarUML}: per la creazione dei diagrammi \textit{UML};
    \item \textit{GanttProject}: per la creazione dei diagrammi di Gantt;
\end{itemize}

\subsection{Gestione della configurazione}

\subsubsection{Utilizzo di Git}
Per la gestione della configurazione del progetto viene utilizzato il sistema di versionamento \textbf{Git}. Il gruppo ha adottato il modello di branching \textbf{GitFlow} per la gestione delle versioni del progetto.

\subsection{Gestione del controllo qualità}
\subsubsection{Scopo}
Il processo di Gestione della Qualità ha l'obiettivo di definire indici
precisi per tutte le attività che riguardano la verifica e la validazione.
Questo permette di garantire che il livello di qualità desiderato venga
rispettato. Per maggiori informazioni sulle metriche utilizzate,
si fa riferimento al \textit{Piano di Qualifica v2.0.0}.
Questo documento fornirà una trattazione dettagliata su come
vengono misurati i risultati e garantire la qualità nei progetti.
\subsubsection{Denominazione metriche}
Le metriche utilizzate sono identificate tramite un codice univoco:
\begin{center}
    \textbf{M[Utilizzo]-[Acronimo]}
\end{center}
Nella quale \textbf{[Utilizzo]} indica se la metrica è:
\begin{itemize}
    \item \textbf{PC}: per la qualità di processo;
    \item \textbf{PD}:  per la qualità di prodotto.
\end{itemize}
Mentre \textbf{[Acronimo]} indica il nome della metrica.
\subsection{Gestione di verifica}
\subsubsection{Scopo} 
Il processo di verifica ha lo scopo di garantire che il prodotto finale soddisfi i requisiti richiesti, fornendo un risultato corretto, coeso e completo. Tale processo riceve in input quanto prodotto fino a quello momento e lo restituisce in uno stato conforme alle aspettative.
Per ottenere ciò vengono utilizzati i processi di test e analisi.

\paragraph{Analisi statica} ~

\noindent Consiste nell'analizzare il codice sorgente e la documentazione del prodotto per verificarne la correttezza (assenza di difetti e/o errori).

I metodi per effettuare analisi statica sono catalogabili in due categorie: manuali e automatici.
Quelli manuali sono due:
\begin{itemize}
    \item \textbf{Walkthrough:} mira a rivelare la presenza di difetti tramite una lettura critica a largo spettro e senza assunzioni di alcun genere. Il Walkthrough va pianificato e, in seguito, va discusso quanto individuato per correggere i difetti, documentando le attività svolte;
    \item \textbf{Inspection:} mira a rivelare la presenza di difetti tramite una lettura mirata del prodotto da parte di verificatori distinti dai programmatori secondo una strategia che si focalizza sui punti ritenuti critici. 
\end{itemize}

\paragraph{Analisi dinamica} ~

\noindent Consiste nell'analizzare il prodotto in esecuzione per verificare che il comportamento sia conforme a quanto previsto e che non si presentino anomalie o malfunzionamenti.
I criteri da seguire per definire i test per l'analisi dinamica vanno inseriti nel \textit{Piano di Qualifica}. Un test ha lo scopo di far fallire il programma e se ci riesce, quel test va aggiunto alla suite dei test del progetto in modo permanente.

Ci sono vari tipi di test:
\begin{itemize}
    \item \textbf{Test di unità:} \textit{unit test}, sono test che vengono eseguiti su singole unità di codice, per verificare che il codice funzioni correttamente;
    \item \textbf{Test di integrazione:} \textit{integration test}, sono test che vengono eseguiti su più unità di codice, per verificare una corretta interazione tra le componenti;
    \item \textbf{Test di sistema:} \textit{system test}, sono test che verificano la copertura dei requisiti e vengono definiti durante la fase di analisi dei requisiti;
    \item \textbf{Test di regressione:} \textit{regression test}, sono test che vengono eseguiti dopo una modifica del codice per verificare che le modifiche non abbiano introdotto errori.
\end{itemize}

\subsection{Gestione di validazione}
\subsubsection{Scopo} 
La validazione è la conferma finale che il prodotto sia conforme alle attese del committente. Dopo di essa il prodotto può essere rilasciato.


