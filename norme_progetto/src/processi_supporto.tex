\section{Processi di Supporto}
\subsection{Scopo} In questa sezione vengono descritte le procedure per le seguenti fasi del progetto:
\begin{itemize}
    \item {Documentazione}
    \item {Configurazione}
    \item {Qualità}
    \item {Verifica}
    \item {Validazione}
    \item {Controllo costi}
    \item {Risoluzione dei problemi}
\end{itemize}
\subsection{Gestione della documentazione}
    Vengono documentate tutte le informazioni riguardanti il ciclo di vita di ciascun processo o attività.
    Queste attività si dividono in:
    \subsubsection{Implementazione} 
    Per tutti i documenti redatti viene stabilito il titolo, lo scopo, il destinatario, le procedure di tutti i ruoli e le scadenze per le revisioni intermedie e finali.
    \subsubsection{Design e sviluppo} 
        \paragraph{Documentazione} ~

        È stato creato un template per uniformare lo stile di tutti i documenti prodotti e agevolarne la produzione.\\
        Nello specifico sono presenti:
        \begin{itemize}
            \item \textbf{Prima pagina} contenente:
            \begin {itemize} 
                \item logo
                \item indirizzo email del gruppo
                \item titolo del documento
                \item responsabile, redattori e verificatori
                \item uso (interno o esterno)
                \item destinatari
                \item sommario
            \end{itemize}
            \item \textbf{Registro delle modifiche}: in ogni documento viene inserita una tabella contenente il registro delle modifiche che riporta: 
            \begin{itemize}
                \item versione
                \item data
                \item autore
                \item breve descrizione della modifica
            \end{itemize}
            \item \textbf{Indice}: ha lo scopo di fornire una visione gerarchica dei contenuti del documento e di facilitare la ricerca di informazioni specifiche.
            \item \textbf{Contenuto principale}: % DA DEFINIRE
        \end{itemize}

        \paragraph{Verbali} ~

        È stato creato un template per uniformare lo stile di tutti i verbali prodotti e agevolarne la produzione.\\
        Nello specifico sono presenti:
        \begin{itemize}
            \item \textbf{Prima pagina} contenente:
            \begin {itemize} 
                \item logo
                \item indirizzo email del gruppo
                \item titolo del documento
                \item responsabile, redattori e verificatori
                \item uso (interno o esterno)
                \item destinatari
                \item sommario
            \end{itemize}
            \item \textbf{Indice}
            \item \textbf{Ordine} contenente:
            \begin {itemize} 
                \item ora di inizio/termine
                \item luogo
                \item partecipanti interni e esterni
                \item ordine del giorno
            \end{itemize}
            \item \textbf{Resoconto}: contiene una descrizione dettagliata degli argomenti trattati e delle decisioni prese.
        \end{itemize}

        \paragraph{Norme tipografiche} ~

        \paragraph{Convenzioni nomenclatura file} ~

        Escludendo l'estensione, le regole di nomenclatura dei file sono le seguenti:
            \begin{itemize}
                \item i nomi dei file devono essere completamente in minuscolo
                \item viene utilizzato il carattere underscore \texttt{\_} per separare le parole
                \item omissione delle preposizioni (DA VALUTARE)
            \end{itemize}

        \paragraph{Formato data} ~

         Viene utilizzato il formato \texttt{YYYY-MM-DD} per le date.
            \begin{itemize}
                \item \texttt{YYYY} rappresenta l'anno (4 cifre)
                \item \texttt{MM} rappresenta il mese (2 cifre)
                \item \texttt{DD} rappresenta il giorno (2 cifre)
            \end{itemize}

        \paragraph{Elenchi} ~

        \begin{itemize}
            \item \textbf{Elenco puntato}: viene utilizzato per elencare elementi in cui non ha importanza l'ordine.
            \item \textbf{Elenco numerato}: viene utilizzato per elencare elementi in cui ha importanza l'ordine o la priorità.
        \end{itemize}

        \paragraph{Glossario} ~

         Le convenzioni adottate per fare riferimento a termini specifici presenti nel glossario sono le seguenti:
            \begin{itemize}
                \item un termine presente nel glossario viene indicato con il comando \glo
                \item in ogni documento un termine viene marcato solo alla sua prima occorrenza
            \end{itemize}

        \paragraph{Stile del testo} ~

        \paragraph{Sigle} ~

    \subsubsection{Produzione documenti}
    Per agevolare la produzione dei documenti è stato creato un template che contiene tutte le informazioni necessarie per la redazione di un documento.\\
    Precisamente è stata definita una cartella \texttt{componenti\_comuni} che contiene:
    \begin{itemize}
        \item \texttt{src}: contiene i file \texttt{command.tex} e \texttt{packages.tex} che contengono i comandi e i pacchetti necessari per la produzione e compilazione del documento
        \item \texttt{img}: contiene tutte le immagini utilizzate nel template
    \end{itemize}

    \subsubsection{Elementi grafici}
        \paragraph{Tabelle} ~

        \paragraph{Immagini} ~

        \paragraph{Diagrammi UML} ~

    \subsubsection{Strumenti}
        Per la redazione dei documenti vengono utilizzati i seguenti strumenti:
        \begin{itemize}
            \item \textbf{LaTeX}: per la redazione dei documenti;
            \item \textbf{VSCode}: per la gestione del codice sorgente e della documentazione;
            \item \textbf{StarUML}: per la creazione dei diagrammi UML;
        \end{itemize}

    \subsection{Gestione della configurazione}

        \subsubsection{Utilizzo di Git}
        Per la gestione della configurazione del progetto viene utilizzato il sistema di versionamento Git.\\
        Il gruppo ha adottato il modello di branching \textbf{GitFlow} DA CONTINUARE

    \subsection{Gestione del controllo qualità}
    \subsection{Gestione di verifica}
    \subsection{Gestione di validazione}
    \subsection{Gestione del controllo dei costi}
    \subsection{Gestione della risoluzione dei problemi}

