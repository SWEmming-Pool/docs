\section{Processi Primari}
\subsection{Fornitura}
\subsubsection{Scopo}
Lo scopo del processo di fornitura è quello analizzare e definire le operazioni e risorse necessarie per fornire il prodotto finale in modo tale che rispetti i requisiti imposti dal committente.
\subsubsection{Attività}
\begin{itemize}
    \item \textbf{Avvio:} si procede con un'analisi preliminare dei capitolati. Vengono effettuate riunioni tra i membri del gruppo per discutere le proposte e scambiarsi le opinioni.
    %Viene effettuato lo \textbf{Studio di Fattibilità};
    \item \textbf{Lettera di candidatura:} si redige una lettera di candidatura in cui si indica il termine ultimo di consegna e il preventivo dei costi per la realizzazione del progetto;
    \item \textbf{Aggiudicazione appalto:} fase in cui si riceve ufficialmente in carico il progetto;
    \item \textbf{Piano di Progetto:} il responsabile e gli amministratori redigono il \textit{Piano di Progetto} che contiene le seguenti informazioni:
        \begin{itemize}
                \item {Analisi dei rischi.}
                \item {Analisi dei costi.}
                \item {Pianificazione delle attività.}
                \item {Modello di sviluppo.}
        \end{itemize}
    \item \textbf{Piano di Qualifica:} i verificatori redigono il \textit{Piano di Qualifica} che contiene le linee guida da adottare per garantire la qualità prefissata per il prodotto finale. Il documento conterrà i seguenti punti:
        \begin{itemize}
                \item {Qualità di processo.}
                \item {Qualità di prodotto.}
                \item {Specifiche dei test.}
                \item {Standard di qualità.}
            \end{itemize}
\end{itemize}
\subsection{Sviluppo}
\subsubsection{Scopo} Questo processo contiene tutte le azioni e attività finalizzate ad analisi dei requisiti, design, codifica e test.
\subsubsection{Attività}

\paragraph{Analisi dei requisiti}: viene effettuata la stesura dell'\textit{Analisi dei Requisiti}, che comprende elenco e analisi dei requisiti funzionali, non funzionali e di qualità. Questo documento si prefigge l'obiettivo di descrivere in modo chiaro e preciso tutte le caratteristiche richieste esplicitamente o implicitamente dal committente.
    Al suo interno vengono analizzati:
    \begin{itemize}
        \item \textbf{Casi d'uso:} consiste in una descrizione di un insieme di azioni che si possono eseguire su un sistema per ottenere un determinato risultato.
        Le caratteristiche di un caso d'uso sono:
        \begin{itemize}
            \item {codice identificativo.}
            \item {nominativo.}
            \item {diagramma UML.}
            \item {attore primario.}
            \item {attori secondario.}
            \item {pre-condizioni.}
            \item {post-condizioni.}
            \item {scenario principale.}
            \item {estensioni.}
        \end{itemize}
        Il codice utilizzato in ciascun caso d'uso è formato dalle iniziali di Use Case seguite dal numero progressivo del caso d'uso in questione e i suoi eventuali sottocasi relativi.
        \item \textbf{Requisiti:} ciascun requisito viene definito seguendo lo standard di codifica:\\
        \textbf{R[Importanza][Tipologia][Codice]}
        \begin {itemize}
            \item{importanza:} può essere \textbf{1} obbligatorio, \textbf{2} desiderabile e \textbf{3} opzionale.
            \item {tipologia:} può essere \textbf{F} per funzionale, \textbf{NF} per non funzionale, \textbf{Q} per qualità e \textbf{V} per vincolo.
            \item {codice:} identificatore univoco del requisito in forma gerarchica.
        \end {itemize}
    \end{itemize}

    Alla versione \textit{v1.0.2} del documento \textit{Analisi dei Requisiti} la struttura è la seguente:
    \begin{itemize}
        \item \textbf{Introduzione}: contiene una breve descrizione del documento e del prodotto software;
        \item \textbf{Descrizione generale}: contiene una dezcrizione degli obiettivi del prodotto software;
        \item \textbf{Casi d'uso}: contiene l'elenco dei vari casi d'uso con i relativi diagrammi e descrizioni;
        \item \textbf{Requisiti}: contiene l'elenco dei requisiti con le relative descrizioni e tracciamento con i casi d'uso;
    \end{itemize}

\paragraph{Progettazione}: questa fase prevede la realizzazione della specifica architetturale del prodotto finale, partendo dai requisiti definiti e individuando le diverse parti corrispettive, che dovranno essere poi raggrupate fino ad arrivare ad ottenere un unico sistema. \newline
Inoltre, in parallelo allo sviluppo dell'architettura, si procede alla definizione dei test di unità, di integrazione e di sistema. \newline
Tale fase dovrà essere documentata in un documento denominato \textit{Specifica Tecnica}.

[MANCA LA STRUTTURA DEL DOCUMENTO SPECIFICA TECNICA]

\subparagraph{Requirements and Technology Baseline}: fase in cui viene studiato il problema per individuare i requisiti opportuni per il suo soddisfacimento e le tecnologie da utilizzare per la realizzazione del prodotto finale. \newline
Si conclude con lo sviluppo di un Proof of Concept (PoC) che permetta di verificare la fattibilità del progetto;

\subparagraph{Product Baseline}: fase in cui viene definita l'architettura del prodotto finale, infine viene effettuata la codifica, seguita dalla fase di testing.


\paragraph{Codifica}: fase in cui si procede con la codifica del prodotto finale, concretizzando così i risultati delle fasi precedenti.
\subparagraph{Norme di codifica}: il codice sorgente deve essere scritto in modo chiaro e leggibile, seguendo le seguenti norme:
\begin{itemize}
    \item \textbf{Web App}, le cui convenzioni di stile sono imposte dal framework \textit{Angular}:
    \begin{itemize}
        \item \textbf{Nomenclatura}
        \begin{itemize}
            \item \textbf{Cartelle}:le cartelle devono avere il nome dell'elemento, che deve essere in minuscolo e se è composto da piu parole, devono essere separate con \textit{-}. \newline
            Ad esempio il componente \textit{component-name} avrà la cartella con lo stesso nome;
            \item \textbf{File}: il nome dei file deve essere in minuscolo e se è composto da piu parole, devono essere separate con \textit{-}. \newline
            Inoltre, in base alla tipologia di elemento, il file deve avere un'estensione specifica:
            \begin{itemize}
                \item \textit{.component}: per i componenti;
                \item \textit{.service}: per i servizi;
                \item \textit{.pipe}: per i filtri;
                \item \textit{.directive}: per le direttive.
            \end{itemize}
            \item \textbf{Classi}: la nomenclatura delle classi è basata sulle regole di \textit{upper CamelCase} e devono essere coerenti con il nome dell'elemento corrispettivo. \newline
            Ad esempio il componente \textit{component-name} avrà il seguente nome della classe \textit{ComponentName};
            \item \textbf{Metodi}: la nomenclatura dei metodi è basata sulle regole di \textit{upper camelCase};
            \item \textbf{Variabili e argomenti delle funzione}: la nomenclatura delle variabili e degli argomenti delle funzioni sono basate sulle regole di \textit{lower camelCase};
            \item \textbf{Costanti}: la nomenclatura delle costanti deve essere espressa in maiuscolo e se è composta da piu parole, devono essere separate con \textit{\textunderscore};
            \item \textbf{Commenti}: i commenti dovranno essere inseriti prima dell’inizio di un  costrutto e presentati in
            lingua italiana.
        \end{itemize}
        \item \textbf{Struttura progetto}: i file del progetto sono organizzati e suddivisi in cartelle, in base all'elemento a cui appartengono. \newline
        Tutti i file e cartelle devono essere contenuti all'interno della cartella \textit{src}. \newline
        Ad esempio il componente \textit{component-name} avrà la seguente struttura:
        \begin{itemize}
            \item \textit{component-name.component.ts}: file contenente la logica del componente;
            \item \textit{component-name.component.html}: file contenente il template HTML del componente;
            \item \textit{component-name.component.css}: file contenente gli stili CSS del componente;
            \item \textit{component-name.component.spec.ts}: file contenente i test del componente;
        \end{itemize}
        È possibile inoltre, all'interno della cartella di un componente, creare altri componenti figli.
        \item \textbf{Identazione}: l'indentazione deve essere di 4 spazi.
    \end{itemize}
    \textbf{Smart Contract}, alcune convenzioni di stile sono imposte dal linguaggio \textit{Solidity}:
    \begin{itemize}
        \item \textbf{Nomenclatura}:
        \begin{itemize}
            \item \textbf{File}: la nomenclatura dei file è basata sulle regole di \textit{upper CamelCase} e devono essere coerenti con il nome del contratto corrispettivo. \newline
            Se all'interno di un file si trovano più contratti, il nome di tale file deve corrispondere a quello del contratto principale;
            \item \textbf{Contratti e Librerie}: la nomenclatura dei contratti e delle librerie è basata sulle regole di \textit{upper CamelCase} e deve essere coerente con il nome del file corrispettivo;
            \item \textbf{Metodi}: la nomenclatura dei metodi è basata sulle regole di \textit{lower camelCase};
            \item \textbf{Variabili}; la nomenclatura delle variabili è basata sulle regole di \textit{lower camelCase};
            \item \textbf{Argomenti delle funzione}: la nomenclatura degli argomenti delle funzioni è basata sulle regole di \textit{lower camelCase} e il carattere iniziale deve essere \textit{\textunderscore};
            \item \textbf{Costanti}: la nomenclatura delle costanti deve essere espressa in maiuscolo e se è composta da piu parole, devono essere separate con \textit{\textunderscore};
            \item \textbf{Commenti}: i commenti dovranno essere inseriti prima dell’inizio di un  costrutto e presentati in lingua italiana.
        \end{itemize}
        \item \textbf{Struttura progetto}: convenzione imposta dal framework \textit{Truffle}:
        \begin{itemize}
            \item \textit{contracts}: cartella contenente i contratti;
            \item \textit{build}: cartella contenente i file compilati;
            \item \textit{test}: cartella contenente i test;
        \end{itemize}
        \item \textbf{Identazione}: l'indentazione deve essere di 4 spazi.
    \end{itemize}
    \textbf{API REST} alcune delle seguenti convenzioni seguono le medesime regole di \textit{Java}:
    \begin{itemize}
        \item \textbf{Nomenclatura}:
        \item \begin{itemize}
            \item \textbf{File}: la nomenclatura dei file è basata sulle regole di \textit{upper CamelCase};
            \item \textbf{Classi}: la nomenclatura delle classi è basata sulle regole di \textit{upper CamelCase};
            \item \textbf{Metodi}: la nomenclatura dei metodi è basata sulle regole di \textit{upper CamelCase};
            \item \textbf{Variabili e argomenti delle funzione}: la nomenclatura delle variabili e degli argomenti delle funzioni sono basate sulle regole di \textit{lower camelCase};
            \item \textbf{Costanti}: la nomenclatura delle costanti deve essere espressa in maiuscolo e se è composta da piu parole, devono essere separate con \textit{\textunderscore};
            \item \textbf{Commenti}: i commenti dovranno essere inseriti prima dell’inizio di un  costrutto e presentati in lingua italiana.
        \end{itemize}
        \item \textbf{Struttura progetto} convenzioni imposta dal framework \textit{Spring Boot}:
        \begin{itemize}
            \item \textit{src/main/java}: cartella contenente il codice sorgente;
            \item \textit{src/main/resources}: cartella contenente i file di configurazione;
            \item \textit{src/test/java}: cartella contenente i test.
        \end{itemize}
        \item \textbf{Identazione}: l'indentazione deve essere di 2 spazi.
    \end{itemize}
\end{itemize}

\subparagraph{Qualità del codice}: la qualità del codice è definita dalle metriche, che fanno riferimento al \textit{Piano di Qualifica v1.0.0} e dai requisiti di qualità, descritti nell'\textit{Analisi dei Requisiti v1.0.2}. 

\subparagraph{Strumenti di codifica} gli strumenti utilizzati per la codifica e di supporto ad essa sono:
\begin{itemize}
    \item \textbf{Visual Studio Code}: l'IDE utilizzato per la stesura del codice per la web app e dello smart contract;
    \item \textbf{IntelliJ IDEA}: l'IDE utilizzato per la stesura del codice per l'API REST;
    \item \textbf{Angular}: framework utilizzato per la realizzazione della web app;
    \item \textbf{Solidity}: linguaggio utilizzato per la realizzazione dello smart contract;
    \item \textbf{Spring Boot}: framework utilizzato per la realizzazione dell'API REST;
    \item \textbf{Node.js}: runtime environment utilizzato per l'esecuzione di codice JavaScript;
    \item \textbf{Metamask}: wallet utilizzato per effettuare delle transazioni;
    \item \textbf{Truffle}: framework utilizzato per la realizzazione e la compilazione dello smart contract;
\end{itemize}

\paragraph{Test}: fase in cui si procede con la verifica della qualità del prodotto finale, attraverso l'esecuzione di test di sistema, ideati e implementati nelle fasi precedenti. \newline
I test di sistema sono stati descritti nel documento \textit{Piano di Qualifica v.1.0.0}, sono tracciati con i relativi requisiti e indicato il loro stato di implementazione. \newline
Il loro scopo è quello di determinare se il prodotto soddisfa i requisiti richiesti e se rispetta le norme di qualità stabilite.

% DESCRIVERE I TEST DI UNITÀ NELLA FASE DI CODIFICA??
