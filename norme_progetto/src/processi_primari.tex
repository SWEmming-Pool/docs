\section{Processi Primari}
\subsection{Fornitura}
\subsubsection{Scopo}
Lo scopo del processo di fornitura è quello analizzare e definire le operazioni e risorse necessarie per fornire il prodotto finale in modo tale che rispetti i requisiti imposti dal committente.
\subsubsection{Attività}
\begin{itemize}
    \item \textbf{Avvio:} si procede con un'analisi preliminare dei capitolati. Vengono effettuate riunioni tra i membri del gruppo per discutere le proposte e scambiarsi le opinioni.
    %Viene effettuato lo \textbf{Studio di Fattibilità};
    \item \textbf{Lettera di candidatura:} si redige una lettera di candidatura in cui si indica il termine ultimo di consegna e il preventivo dei costi per la realizzazione del progetto;
    \item \textbf{Aggiudicazione appalto:} fase in cui si riceve ufficialmente in carico il progetto;
    \item \textbf{Piano di Progetto:} il responsabile e gli amministratori redigono il Piano di Progetto che contiene le seguenti informazioni:
        \begin{itemize}
                \item {Analisi dei rischi.}
                \item {Analisi dei costi.}
                \item {Pianificazione delle attività.}
                \item {Modello di sviluppo.}
        \end{itemize}
    \item \textbf{Piano di Qualifica:} i verificatori redigono il Piano di Qualifica che contiene le linee guida da adottare per garantire la qualità prefissata per il prodotto finale. Il documento conterrà i seguenti punti:
        \begin{itemize}
                \item {Qualità di processo.}
                \item {Qualità di prodotto.}
                \item {Specifiche dei test.}
                \item {Standard di qualità.}
            \end{itemize}
\end{itemize}
\subsection{Sviluppo}
\subsubsection{Scopo} Questo processo contiene tutte le azioni e attività finalizzate ad analisi dei requisiti, design, scrittura di codice, test e installazione dei prodotti software\glo\:.
\subsubsection{Attività}
\paragraph{Analisi dei requisiti:} viene effettuata la stesura dell'Analisi dei Requisiti, che comprende elenco e analisi dei requisiti funzionali, non funzionali e di qualità. Questo documento si prefigge l'obiettivo di descrivere in modo chiaro e preciso tutte le caratteristiche richieste esplicitamente o implicitamente dal committente.
    Al suo interno vengono analizzati:
    \begin{itemize}
        \item \textbf{Casi d'uso:} consiste in una descrizione di un insieme di azioni che si possono eseguire su un sistema per ottenere un determinato risultato.
        Le caratteristiche di un caso d'uso sono:
        \begin{itemize}
            \item {codice identificativo.}
            \item {nominativo.}
            \item {diagramma UML.}
            \item {attore primario.}
            \item {attori secondario.}
            \item {pre-condizioni.}
            \item {post-condizioni.}
            \item {scenario principale.}
            \item {estensioni.}
        \end{itemize}
        Il codice utilizzato in ciascun caso d'uso è formato dalle iniziali di Use Case seguite dal numero progressivo del caso d'uso in questione e i suoi eventuali sottocasi relativi.
        \item \textbf{Requisiti:} ciascun requisito viene definito seguendo lo standard di codifica:\\
        \textbf{R[Importanza][Tipologia][Codice]}
        \begin {itemize}
            \item{importanza:} può essere \textbf{1} obbligatorio, \textbf{2} desiderabile e \textbf{3} opzionale.
            \item {tipologia:} può essere \textbf{F} per funzionale, \textbf{NF} per non funzionale, \textbf{Q} per qualità e \textbf{V} per vincolo.
            \item {codice:} identificatore univoco del requisito in forma gerarchica.
        \end {itemize}
    \end{itemize}
