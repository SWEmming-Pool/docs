\section{Processi Primari}
\subsection{Fornitura}
\subsubsection{Scopo}
Lo scopo del processo di fornitura è quello analizzare e definire le operazioni e risorse necessarie per fornire il prodotto finale in modo tale che rispetti i requisiti imposti dal committente.
\subsubsection{Attività}
\begin{itemize}
    \item \textbf{Avvio:} si procede con un'analisi preliminare dei capitolati offerti e viene effettuato lo \textbf{Studio di Fattibilità};
    \item \textbf{Lettera di candidatura:} si redige una lettera di candidatura in cui si indica il termine ultimo di consegna e il preventivo dei costi;
    \item \textbf{Aggiudicazione appalto} fase in cui si riceve in carico il progetto;
    \item \textbf{Pianificazione:} si definisce il framework di progetto che ne garantisca la corretta gestione e qualità di realizzazione. Vengono studiati gli strumenti consigliati dal committente e se ne valuta il loro impiego. Vengono redatti il \textbf{Piano di Progetto} e \textbf{Piano di Qualifica}, al loro interno vengono descritti:
    \begin{itemize}
        \item {la struttura organizzativa del gruppo}
        \item {l'ambiente di lavoro}
        \item {modello di sviluppo}
        \item {la suddivisione delle attività}
        \item {la revisione qualità}
        \item {analisi e gestione dei rischi}
        \item {gestione delle risorse}
        \item {gestione del ticketing}
        \item {formazione dei membri del gruppo}
\end{itemize}
    \item \textbf{Esecuzione e controllo:} si procede con la realizzazione del progetto secondo i criteri impostati nel punto precedente. Si effettua un controllo costante del progetto per verificare che rispetti i requisiti del committente.
    \item \textbf{Revisione:} vengono coordinate le attività di verifica delle attività svolte.
    %e vengono effettuati degli incontri informali con il committente per verificare che il progetto sia conforme alle sue esigenze
    \item \textbf{Consegna:} viene consegnato il prodotto finale.
\end{itemize}
\subsection{Sviluppo}
\subsubsection{Scopo} Questo processo contiene tutte le azioni e attività finalizzate all'analisi dei requisiti, design, stesura di codice, testing e installazione dei prodotti software. 
\subsubsection{Attività}
\begin{itemize}
    \item \textbf{Analisi dei requisiti} 
    \item \textbf{Design} 
    \item \textbf{Codifica} 
    \item \textbf{Testing} 
    \item \textbf{Installazione} 
\end{itemize}

\subsection{Operazioni}


