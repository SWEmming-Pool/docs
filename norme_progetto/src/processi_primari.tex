\section{Processi Primari}
\subsection{Fornitura}
\subsubsection{Scopo}
Lo scopo del processo di fornitura è quello analizzare e definire le operazioni e risorse necessarie per fornire il prodotto finale in modo tale che rispetti i requisiti imposti dal committente.
\subsubsection{Attività}
\begin{itemize}
    \item \textbf{Avvio:} si procede con un'analisi preliminare dei capitolati. Vengono effettuate riunioni tra i membri del gruppo per discutere le proposte e scambiarsi le opinioni.
    %Viene effettuato lo \textbf{Studio di Fattibilità};
    \item \textbf{Lettera di candidatura:} si redige una lettera di candidatura in cui si indica il termine ultimo di consegna e il preventivo dei costi per la realizzazione del progetto;
    \item \textbf{Aggiudicazione appalto:} fase in cui si riceve ufficialmente in carico il progetto;
    \item \textbf{Piano di Progetto:} il responsabile e gli amministratori redigono il \textit{Piano di Progetto} che contiene le seguenti informazioni:
        \begin{itemize}
                \item {Analisi dei rischi.}
                \item {Analisi dei costi.}
                \item {Pianificazione delle attività.}
                \item {Modello di sviluppo.}
        \end{itemize}
    \item \textbf{Piano di Qualifica:} i verificatori redigono il \textit{Piano di Qualifica} che contiene le linee guida da adottare per garantire la qualità prefissata per il prodotto finale. Il documento conterrà i seguenti punti:
        \begin{itemize}
                \item {Qualità di processo.}
                \item {Qualità di prodotto.}
                \item {Specifiche dei test.}
                \item {Standard di qualità.}
            \end{itemize}
\end{itemize}
\subsection{Sviluppo}
\subsubsection{Scopo} Questo processo contiene tutte le azioni e attività finalizzate ad analisi dei requisiti, design, codifica, test e installazione dei prodotti software\glo \:.
\subsubsection{Attività}

\paragraph{Analisi dei requisiti}: viene effettuata la stesura dell'\textit{Analisi dei Requisiti}, che comprende elenco e analisi dei requisiti funzionali, non funzionali e di qualità. Questo documento si prefigge l'obiettivo di descrivere in modo chiaro e preciso tutte le caratteristiche richieste esplicitamente o implicitamente dal committente.
    Al suo interno vengono analizzati:
    \begin{itemize}
        \item \textbf{Casi d'uso:} consiste in una descrizione di un insieme di azioni che si possono eseguire su un sistema per ottenere un determinato risultato.
        Le caratteristiche di un caso d'uso sono:
        \begin{itemize}
            \item {codice identificativo.}
            \item {nominativo.}
            \item {diagramma UML.}
            \item {attore primario.}
            \item {attori secondario.}
            \item {pre-condizioni.}
            \item {post-condizioni.}
            \item {scenario principale.}
            \item {estensioni.}
        \end{itemize}
        Il codice utilizzato in ciascun caso d'uso è formato dalle iniziali di Use Case seguite dal numero progressivo del caso d'uso in questione e i suoi eventuali sottocasi relativi.
        \item \textbf{Requisiti:} ciascun requisito viene definito seguendo lo standard di codifica:\\
        \textbf{R[Importanza][Tipologia][Codice]}
        \begin {itemize}
            \item{importanza:} può essere \textbf{1} obbligatorio, \textbf{2} desiderabile e \textbf{3} opzionale.
            \item {tipologia:} può essere \textbf{F} per funzionale, \textbf{NF} per non funzionale, \textbf{Q} per qualità e \textbf{V} per vincolo.
            \item {codice:} identificatore univoco del requisito in forma gerarchica.
        \end {itemize}
    \end{itemize}

\paragraph{Sviluppo PoC}: fase in cui viene sviluppata una demo eseguibile che permetta di verificare la fattibilità del progetto, stabilendone inoltre, le tecnologie da utilizzare per la realizzazione del prodotto finale. \newline
Si può collocare tra la fase di analisi dei requisiti e quella di progettazione logica, oppure all'interno di quest'ultima, il gruppo ha deciso di seguire la seconda opzione (DUBBIO). \newline % dubbio: intrecci temporali
Per lo sviluppo del PoC si deve aver completata l'attività di \textit{Analisi dei Requisiti}, in quanto necessario per definire l'architettura logica del prodotto finale e per stabilire quali tecnologie utilizzare. \newline
Si sono dunque pianificate ed seguite le seguenti attività:
\begin{itemize}
    \item \textit{progettazione logica}: si è proceduto con la definizione dell'architettura logica del sistema e le sue componenti (ELENCARE TALI COMPONENTI?);
    \item \textit{scelta delle tecnologie}: avvenuta in base ai requisiti, caratteristiche richieste per il prodotto finale e agli incontri con l'azienda proponente \companyName;
    \item \textit{studio delle tecnologie}: avvenuto tramite la lettura di documentazione e ricerca su internet, autonomamente effettuata dai singoli membri del gruppo;
    \item \textit{codifica}: implementazione delle tecnologie, codificando le funzionalità base e principali per l'interazione tra i componenti precedentemente definiti.
\end{itemize}

\paragraph{Progettazione:} questa fase prevede la realizzazione della specifica architetturale del prodotto finale, la quale dovrà essere documentata in un documento denominato \textit{Specifica Tecnica}, che comprende:
\begin{itemize}
    \item \textbf{L'\textit{architettura logica}}\glo: la struttura logica del sistema, ovvero la suddivisione in componenti e la loro interazione; (DA DEFINIRE NEL GLOSSARIO)
    \item l'\textit{architettura di deployment}\glo: la locazione dei componenti nel sistema di esecuzione; (DA DEFINIRE NEL GLOSSARIO)
    \item i \textit{design pattern}\glo utilizzati e determinati dalle tecnologie adottate; (DA DEFINIRE NEL GLOSSARIO)
    \item gli \textit{idiomi}\glo: pattern di più basso livello eventualemente impiegati. (DA DEFINIRE NEL GLOSSARIO)
\end{itemize}
Per identificare al meglio le componenti si è pensato di analizzare il flusso di esecuzione del sistema dal punto di vista dell'utente utilizzatore, per attribuire ad ogni interazione dell'utente un corrispetiva azione da parte del sistema. \newline
Così facendo si sono potute definire le classi principali del sistema: raggrupando le funzionalità (metodi) e attributi (variabili di stato) in base al loro scopo e alla loro utilità, rispettando inoltre i requisiti presenti nel documento \textit{Analisi dei Requisiti}. \newline
Per raffinare ulteriormente la progettazione e perseguire lo stato dell'arte si è proceduto con la definizione delle classi di supporto, ovvero quelle che non sono direttamente utilizzate dall'utente, ma che sono necessarie per il corretto funzionamento del sistema, individuate dallo studio e applicazione di eventuali \textit{design pattern}. \newline
Una volta raggiunta l'architettura definitiva del sistema si è proceduto con la definizione della suite di test necessari da implementare per garantire la qualità del prodotto finale e misurare il raggiungimento degli obiettivi definiti dai requisiti. 

\paragraph{Codifica}

\paragraph{Test}


