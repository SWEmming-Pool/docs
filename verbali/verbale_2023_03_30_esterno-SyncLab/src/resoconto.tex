\section{Resoconto}

\subsection{Risultati ottenuti}

\begin{enumerate}
      \item Il nostro sistema deve agire al momento del pagamento, in modo da poter verificare se il pagamento è stato effettuato correttamente e per potersi salvare le informazioni necessarie per la successiva recensione.
      \item Visto che lo scopo del progetto è incentrato sulla parte di recensione, la parte di pagamento deve essere implementata in modo molto semplice per avere possibilità di dimostrare l'effettivo funzionamento del prodotto.
      \item Il proponente ci ha fornito una spiegazione più dettagliata del design del prodotto, in particolare ha confermato le nostre ipotesi sulle informazioni che il contratto dovrà andare ad utilizzare. 
\end{enumerate}

C'è stata una discussione sul salvataggio delle recensioni come dei token, una struttura offerta dalla piattaforma Ethereum. Il proponente ci ha spiegato che questa struttura non è troppo adatta per il nostro scopo, in quanto per renderla adatta al nostro scopo dovremmo andare a rimuovere delle funzionalità dallo standard ERC-20, cosa che ovviamente non è consigliata. Inoltre, il proponente ci ha spiegato che la struttura dei token è più adatta per la gestione di valute digitali, mentre per la gestione di recensioni è più adatta una struttura più semplice, come una mappa.

\subsection{Decisioni prese}

Risolto il dubbio del gruppo, si è deciso di mantenere la versione del design che si stava già portando avanti.