\section{Resoconto}

\subsection{Risultati ottenuti}
% Retrospettiva sull'esito della prima revisione di avanzamento;
Il gruppo ha analizzato il documento contenente gli esiti della revisione, cercando di capire gli errori e difetti segnalati e i motivi che hanno portato alla loro segnalazione. I problemi rilevati sono stati quindi raccolti in una lista di \textit{issues} all'interno della \textit{project board} in \textit{GitHub}, per essere risolti dai membri del gruppo.

% Programmazione degli obiettivi futuri per lo sviluppo del prodotto;
Il nostro progetto si trova ora nella fase di progettazione. Prima di arrivare ad un nuovo incontro con il proponente è necessario produrre una prima bozza di effettivo design sulla base della quale poter discutere. Sono sorte inoltre alcune domande che andranno poste al proponente riguardo l'utilizzo di alcuni strumenti e strutture offerte dai linguaggi di programmazione che stiamo utilizzando; in particolare, è necessario capire se è possibile utilizzare delle \textit{classi} per la gestione delle informazioni relative ai vari tipi di oggetti che andremo a gestire, o se è più opportuno valutare l'utilizzo di alternative differenti.

\subsection{Decisioni prese}

\begin{enumerate}
    \item Definizione delle \textit{issues} relative ai problemi riscontrati nella revisione di avanzamento;
\end{enumerate}

\subsection{Issues}

\begin{center}
    \rowcolors{2}{pari_alt}{dispari_alt}
    \renewcommand{\arraystretch}{1.5}
    \begin{tabular}{c|l|l}
        \rowcolorhead
        \textbf{\color{white}ID} & \textbf{\color{white}Descrizione} & \textbf{\color{white}Assegnatario} \\
        \#29 & Unione paragrafi 4 e 5 PdQ (rif. $\leq$ v1.0.0) & Sebastiano Sanson \\
        \#30 & Aggiunta paragrafo 6 PdQ & \\
        \#31 & Data ultimo accesso risorse web & Nicolò Trinca \\
        \#32 & Versione dei documenti soggetti a ciclo di vita & Ennio Italiano \\
        \#33 & Completare analisi dei rischi PdP &  \\
        \#34 & Completare processi di sviluppo NdP & Sebastiano Sanson \\
        \#35 & Cambio modello di sviluppo da incrementale a agile & Nicolò Trinca \\
        \#36 & Ragionamenti retrospettivi su consuntivo di periodo PdP & \\
        \#37 & Maiuscole titoli PdP e PdQ & Nicolò Trinca \\ 
    \end{tabular}
\end{center}