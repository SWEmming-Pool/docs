\section{Resoconto}

\subsection{Risultati ottenuti}
% Retrospettiva sull'esito della prima revisione di avanzamento;
Il gruppo ha analizzato il documento contenente gli esiti della revisione, cercando di capire quali errori e difetti sono stati segnalati e quali sono stati i motivi che hanno portato alla loro segnalazione. I problemi rilevati sono stati quindi raccolti in una lista di cose da fare, sotto forma di \textit{issues} all'interno della \textit{project board} in \textit{GitHub}, che verranno poi svolte dal gruppo.

% Programmazione degli obiettivi futuri per lo sviluppo del prodotto;
Il nostro progetto ora si trova alla fase di progettazione. Prima di arrivare ad un nuovo incontro con il committente è necessario produrre una prima bozza di effettivo design di cui discutere. Altre domande sorte che andranno poste al proponente riguardano l'utilizzo di alcuni strumenti e strutture offerte dai linguaggi di programmazione che stiamo utilizzando. In particolare, è necessario capire se è possibile utilizzare delle \textit{classi} per la gestione delle informazioni relative ai vari tipi di oggetti che andremo a gestire, oppure se è necessario utilizzare delle alternative migliori.

\subsection{Decisioni prese}

\begin{enumerate}
    \item Definizione delle issue relative ai problemi riscontrati nella revisione di avanzamento;
\end{enumerate}

\subsection{Issue}

\begin{center}
    \begin{tabular}{c|l|l}
        \rowcolor{pari_alt}
        \textbf{ID} & \textbf{Descrizione} & \textbf{Assegnatario} \\
        \midrule
        \#29 & Unione paragrafi 4 e 5 PdQ & \\
        \#30 & Aggiunta paragrafo 6 PdQ & \\
        \#31 & Data ultimo accesso risorse web & \\
        \#32 & Versione dei documenti soggetti a ciclo di vita & \\
        \#33 & Completare analisi dei rischi PdP & \\
        \#34 & Completare processi di sviluppo NdP & \\
        \#35 & Cambio modello di sviluppo da incrementale a agile & \\
        \#36 & Ragionamenti retrospettivi su consuntivo di periodo PdP & \\
        \#37 & Maiuscole titoli PdP e PdQ & Nicolò Trinca \\ 
    \end{tabular}
\end{center}