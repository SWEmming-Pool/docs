\section{Resoconto}

\subsection{Risultati ottenuti}

\begin{enumerate}
    \item Tipologie di attori: le differenze tra utente non registrato, utente registrato e utente business che erano stati trovati in precedenza. Ogni utente è identificato tramite il suo wallet e per il nostro sistema non esiste un profilo aggiuntivo e necessità di registrazione e login.
    \begin{itemize}
        \item Non vi è distinzione tra utente business e normale. Uno stesso utente può inviare e ricevere recensioni.
        \item Va differenziata la casistica di utente connesso tramite metamask e non connesso.
        \item Un utente non collegato è limitato alle sole operazioni di lettura.
    \end{itemize}
    \item Dubbio su contratto come attore esterno: dopo il dubbio sollevato dal \cardin sulla necessità di inserire lo smart contract come attore esterno nei casi d'uso e la spiegazione del proponente si è arrivati alla conclusione che forse il vero attore esterno da inserire è un RPC che va effettivamente a collegarsi. Rimane da confermare la cosa insieme al professore.
    \item Consigli futuri sull'ottimizzazione dell'ottenimento delle recensioni dalla blockchain.
    \item Dopo una piccola discussione sul sistema di API che dovranno essere fornite si è deciso di andare a creare un nuovo sistema all'interno del diagramma dei casi d'uso.  
\end{enumerate}

\subsection{Decisioni prese}
È necessario rivedere i casi d'uso per uniformarli alle nuove analisi e successivamente rivedere i vari documenti prodotti.