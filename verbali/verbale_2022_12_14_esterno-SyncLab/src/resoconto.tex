\section{Resoconto}

\subsection{Risultati ottenuti}

\begin{enumerate}
    \item Sono state riviste le differenze precedentemente individuate tra utente non
          registrato, utente registrato e utente business. Ogni utente è identificato
          tramite il suo wallet e per il nostro sistema non esiste un profilo aggiuntivo
          e necessità di registrazione e login. In particolare:
          \begin{itemize}
              \item non vi è distinzione tra utente business e normale; infatti, un qualsiasi
                    utente deve poter inviare e ricevere recensioni;
              \item c'è differenza tra utente autenticato e non autenticato con Metamask;
              \item un utente non autenticato è limitato alle sole operazioni di lettura (es.
                    visualizzazione recensioni).
          \end{itemize}
    \item Il servizio di API REST è totalmente scollegato dalla web app; è quindi
          necessario analizzare i relativi casi d'uso separatamente.
    \item Lo Smart Contract non va inserito nel diagramma dei casi d'uso come attore
          esterno; un RPC è probabilmente più sensato come attore esterno. Il proponente
          ha inoltre dato consigli sull'ottimizzazione dell'ottenimento delle recensioni
          dalla blockchain.
\end{enumerate}

\subsection{Decisioni prese}
È necessario rivedere i casi d'uso per uniformarli alle nuove analisi e successivamente rivedere i vari documenti prodotti. In particolare, vanno rimosse alcune tipologie di utenti e i relativi casi, e si è deciso di creare un nuovo sistema relativo alle API REST all'interno del diagramma dei casi d'uso.