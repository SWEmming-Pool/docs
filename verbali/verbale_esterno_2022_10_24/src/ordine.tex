\section{Informazioni sul verbale}

\subsection{Informazioni sulla riunione}

\subsubsection{Prima parte}
La prima parte della riunione si è svolta insieme ai proponenti sulla piattaforma da loro proposta.

\begin{center}
	\begin{tabular}{r|p{0.6\linewidth}}
		\toprule
		\textbf{Luogo} & Google Meet \\
		\textbf{Ora di inizio} & 17:00 \\
		\textbf{Ora di fine} & 17:30 \\
		\textbf{Partecipanti interni} & \groupTeam \\
		\textbf{Partecipanti esterni} & Ing. Fabio Pallaro, Dott. Matteo Galvagni
	\end{tabular}
\end{center}

\subsubsection{Seconda parte}
Dopo questa discussione il gruppo si è spostato nel proprio \textit{server Discord}.

\begin{center}
	\begin{tabular}{r|p{0.6\linewidth}}
		\toprule
		\textbf{Luogo} & Discord \\
		\textbf{Ora di inizio} & 17:30 \\
		\textbf{Ora di fine} & 18:30 \\
		\textbf{Partecipanti interni} & \groupTeam
	\end{tabular}
\end{center}

\medskip

\subsection{Ordine del giorno}
L'incontro con l'azienda è stato richiesto per avere delucidazioni in merito al capitolato. I principali dubbi esposti al proponente sono stati:
\begin{itemize}
	\item presenza di documentazione interna all'azienda riguardante framework e tecnologie consigliate;
	\item chiarimenti sul concetto di transazione esposto durante la presentazione del
	capitolato;
	\item chiarimenti sul tipo di pratiche scorrette da prevenire (\emph{review bombing} e/o recensioni false);
	\item eventuale presenza in azienda di persone con esperienza in ambito \emph{Web3};
	\item piattaforme di comunicazione preferite dall'azienda e frequenza di incontro
	consigliata.
\end{itemize}
