\section*{Registro delle modifiche}

\rowcolors{2}{pari_alt}{dispari_alt}
\renewcommand{\arraystretch}{1.5}
\begin{xltabular}{\textwidth} {
		>{\hsize=0.45\hsize\linewidth=\hsize}X
		>{\hsize=0.65\hsize\linewidth=\hsize}X
		>{\hsize=1.10\hsize\linewidth=\hsize}X
		>{\hsize=0.90\hsize\linewidth=\hsize}X
		>{\hsize=1.85\hsize\linewidth=\hsize}X
	}
	\rowcolorhead
	\textbf{\color{white}Versione} &
	\textbf{\color{white}Data} &
	\textbf{\color{white}Nominativo} &
	\textbf{\color{white}Ruolo} &
	\textbf{\color{white}Descrizione} \\
	\hline
	\endfirsthead

	\hline
	\rowcolorhead
	\textbf{\color{white}Versione} &
	\textbf{\color{white}Data} &
	\textbf{\color{white}Nominativo} &
	\textbf{\color{white}Ruolo} &
	\textbf{\color{white}Descrizione} \\
	\hline
	\endhead

	\endfoot
	\endlastfoot

    1.0.0 &
    2023-04-30 &
    Sebastiano Sanson &
    \roleAdministrator &
    Approvato per il rilascio \\

    0.1.0 &
    2023-04-29 &
    Enrico Bacci Bonivento &
    \roleVerifier &
    Verifica documento \\

	0.0.3 &
	2023-04-28 &
	Elia Pasquali &
    \roleProgrammer &
	Stesura sezione §3 API REST \\
	\hline

	0.0.2 &
	2023-04-28 &
	Ennio Italiano &
    \roleProgrammer &
	Stesura sezione §2 Web app  \\
	\hline

	0.0.1 &
	2023-04-26 &
	Ennio Italiano, Elia Pasquali &
    \roleProgrammer &
	Creata struttura del documento in \LaTeX e sezione §1\\
	\hline

\end{xltabular}
\renewcommand{\arraystretch}{1}