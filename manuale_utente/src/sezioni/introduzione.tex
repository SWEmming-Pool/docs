\section{Introduzione}

\subsection{Scopo del documento}
Lo scopo del documento è quello di illustrare le funzionalità dell'applicazione e di fornire le istruzioni per l'utilizzo del software. L'utente sarà messo a conoscenza dei requisiti necessari per il corretto funzionamento del sistema \capName.

\subsection{Glossario}
Al fine di evitare ambiguità nella terminologia usata all'interno del seguente
documento è stato redatto un glossario, in cui vengono riportate le definizioni
di termini tecnici, rilevanti o con un significato particolare.

Per indicare
la presenza di un termine all'interno del glossario si è scelto di
contrassegnarlo con \glo . Per non appesantire la lettura della documentazione
verrà così contrassegnata solo la prima occorrenza di ogni termine in ciascun
documento.
Per una consultazione completa si rimanda al \textit{Glossario v2.0.0}

\subsection{Cos'è Trustify}
Trustify è un sistema che si basa sull'utilizzo di smart contract per offrire un servizio di pagamento con la possibilità di rilasciare recensioni che vengono memorizzate nella blockchain. Grazie a questo sistema, le recensioni diventano pubblicamente verificabili e immutabili, impedendo alle attività di modificare o eliminare le recensioni degli utenti senza essere scoperte. Inoltre, viene disincentivata la creazione di recensioni false in massa, in quanto viene reso obbligatorio un pagamento per poter rilasciare una recensione.

Una web app permette di interagire con un apposito smart contract tramite MetaMask; attraverso di essa, è possibile rilasciare e visualizzare recensioni. Si fornisce inoltre un servizio API REST per il recupero delle recensioni da parte dei commercianti.