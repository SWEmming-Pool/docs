\section{Pianificazione}
\groupName\:ha deciso di suddividere la pianificazione di progetto in tre fasi:
\begin{itemize}
    \item Analisi;
    \item Progettazione di dettaglio e codifica;
    \item Validazione e collaudo.
\end{itemize}

\subsection{Analisi}
Periodo: dal \textbf{2022/11/04} al \textbf{2023/02/08} \newline
Questo periodo ha inizio con l'assegnazione del capitolato d'appalto e termina con l'inizio del periodo di \textit{Proof of Concept}\glo.
Prima di tutto vengono definiti gli strumenti per la comunicazione del gruppo e gli strumenti da utilizzare per il lavoro collaborativo e
per la redazione dei documenti. Successivamente si analizza il capitolato per individuare i requisiti fondamentali, comunicando anche con
il proponente.
Inoltre si è deciso di svolgere una fase di autoformazione per studiare ed apprendere le nuove tecnologie da utilizzare per lo sviluppo del prodotto.
Le attività di questo periodo sono:
\begin{itemize}
    \item \textbf{Autoformazione}: i membri del gruppo si impegnano ad apprendere le nuove tecnologie scelte;
    \item \textbf{Analisi dei requisiti}: l'\roleAnalyst, attraverso lo studio del capitolato e la comunicazione con il proponente,
          individua i requisiti iniziali del prodotto e ne ricava le funzioni principali. Le funzioni sono rappresentate nel documento
          con l'ausilio di diagrammi UML\glo . I requisiti potrebbero evolvere nel tempo in base ai feedback del proponente;
    \item \textbf{Norme di progetto}: in questo documento vengono sono decisi gli strumenti da utilizzare nello sviluppo del prodotto
          e le regole a cui il team dovrà attenersi per la stesura di tutti i documenti. L'\roleAdministrator\:ha il compito di emanare
          le norme;
    \item \textbf{Piano di progetto}: il \roleProjectManager\:illustra un prospetto di pianificazione dettagliata, con attività e compiti,
          a cui il gruppo dovrà attenersi;
    \item \textbf{Piano di qualifica}: il documento ha lo scopo di decidere le procedura per la verifica della qualità e
          offre una visione sull'esito delle verifiche sul prodotto e i suoi componenti;
    \item \textbf{Glossario}: al fine di garantire chiarezza ed evitare ambiguità, viene redatto un glossario contenente tutti i termini necessari di definizione.
\end{itemize}
\subsubsection{\rom{1} Periodo}
Dal \textbf{2022/11/14} al \textbf{2022/11/19}
\newline
Nei primi giorni dell'analisi si discutono le norme per il lavoro collaborativo e si decidono i primi strumenti da utilizzare per
la stesura e la condivisione della documentazione relativa al prodotto. Inoltre si redigono i primi verbali relativi agli incontri
del gruppo.

\subsubsection{\rom{2} Periodo}
Dal \textbf{2022/11/20} al \textbf{2023/01/28}
\newline
In questo periodo viene redatta la documentazione del prodotto. Il gruppo decide di tenere degli incontri settimanali per discutere di
eventuali problemi riscontrati e per verificare i progressi fatti. La comunicazione con il proponente avviene tramite un canale
\textit{Discord}\glo\:dedicato.

\subsubsection{\rom{3} Periodo}
Dal \textbf{2023/01/29} al \textbf{2023/02/05}
\newline
Nell'ultimo periodo di analisi viene verificata la correttezza della documentazione basandosi anche su quanto scritto
sulle \textit{Norme di progetto v1.0.0}. Viene aggiornato il \textit{Glossario v1.0.0} con gli ultimi termini.

\begin{figure}[H]
    \centering
    \includegraphics[scale=0.3]{src/img/Gantt analisi.png}
    \caption{Diagramma di Gantt\glo\:per l'attività di analisi}
\end{figure}

\subsection{Proof of Concept}
Periodo: dal \textbf{2023/02/06} al \textbf{2023/02/20} \newline
Questo periodo inizia al termine della verifica della documentazione e la sua fine coincide con la scadenza
di consegna dei documenti per la revisione \textit{RTB}\glo.
Al termine di questo periodo verrà organizzato un incontro con il proponente per presentare un prototipo del prodotto.\newline
Si hanno due attività:
\begin{itemize}
    \item \textbf{Proof of Concept}: viene realizzato un \textit{Proof of Concept} che dovrà implementare la maggior
          parte delle tecnologie necessarie e svolgerà alcune funzioni principali del prodotto. Il \textit{PoC} è un dimostrabile eseguibile
          e verrà usato come base di partenza per gli incrementi futuri;
    \item \textbf{Modifiche e verifica sui documenti}: i documenti redatti durante la fase di analisi vengono aggiornati e migliorati.
\end{itemize}
\subsubsection{\rom{1} Periodo}
Dal \textbf{2023/02/06} al \textbf{2023/02/08}
\newline
Il gruppo si impegna a studiare ed apprendere il funzionamento delle tecnologie utili e necessarie per la produzione del \textit{PoC}.

\subsubsection{\rom{2} Periodo}
Dal \textbf{2023/02/09} al \textbf{2023/02/17}
\newline
Viene sviluppato il \textit{PoC} secondo le tecnologie scelte e si implementano le funzioni principali del prodotto.

\subsubsection{\rom{3} Periodo}
Dal \textbf{2023/02/18} al \textbf{2023/02/20}
\newline
Nell'ultimo periodo viene redatta la presentazione per la \textit{Requirements and Technology Baseline}.

\begin{figure}[H]
    \centering
    \includegraphics[scale=0.4]{src/img/Gantt PoC.png}
    \caption{Diagramma di Gantt dell'attività di \textit{PoC}}
\end{figure}

\subsection{Progettazione di dettaglio e codifica}
Periodo: dal \textbf{2023/03/20} al \textbf{2023/05/07} \newline
Questo periodo inizia solo se si è superata la \textit{Requirements and Technology Baseline}. Viene utilizzato il \textit{PoC}
come base di partenza per il prodotto. Le attività che compongono questo periodo sono:
\begin{itemize}
    \item \textbf{Product Baseline\glo }: presenta l'architettura del prodotto attraverso il diagramma delle classi;
    \item \textbf{Codifica}: i programmatori sviluppano il codice delle funzionalità del prodotto, aggiornando e migliorando quello già presente nel \textit{PoC};
    \item \textbf{Test}: vengono sviluppati i test;
    \item \textbf{Manuali}: Redazione del \textit{Manuale utente} e del \textit{Manuale sviluppatore} per l'utilizzo del prodotto;
    \item \textbf{Modifiche ai documenti}: i documenti redatti durante le fasi precedenti vengono aggiornati e migliorati.
\end{itemize}

\subsubsection{\rom{1} Sprint}
Dal \textbf{2023/03/20} al \textbf{2023/03/26}
\newline
Gli obiettivi che il gruppo si prepone per questo \textit{sprint} sono:
\begin{itemize}
    \item Correzione della documentazione in base alle segnalazioni ricevute;
    \item Impostazioni delle attività di progettazione e codifica di dettaglio.
\end{itemize}
Le attività svolte sono:
\begin{itemize}
    \item Documentazione: correzioni mirate sui documenti: \textit{NdP, PdP, Pdq};
    \item Progettazione: stesura bozza iniziale.
\end{itemize}
\begin{figure}[H]
    \centering
    \includegraphics[width=\textwidth]{src/img/GanttSprint1.png}
    \caption{Diagramma di Gantt per il primo \textit{sprint}}
\end{figure}
\subsubsection{\rom{2} Sprint}
Dal \textbf{2023/03/27} al \textbf{2023/04/02}
\newline
Dopo un incontro con l'azienda committente sono stati chiariti alcuni dubbi sorti durante le riunioni tra i membri del gruppo.
\newline
Gli obiettivi che il gruppo si prepone per questo \textit{sprint} sono:
\begin{itemize}
    \item Studio approfondito di tutte le tecnologie in uso;
    \item Inizio progettazione di dettaglio con principali classi;
    \item Continuazione stesura documentazione.
\end{itemize}
Le attività svolte sono:
\begin{itemize}
    \item Documentazione: aggiornamento \textit{AdR, PdP, Pdq};
    \item Progettazione: \textit{smart contract}, \textit{API REST} e \textit{frontend};
    \item Apprendimento nuove funzionalità.

\end{itemize}
\begin{figure}[H]
    \centering
    \includegraphics[width=\textwidth]{src/img/Sprint 2.png}
    \caption{Diagramma di Gantt per il secondo \textit{sprint}}
\end{figure}
\subsubsection{\rom{3} Sprint}
Dal \textbf{2023/04/03} al \textbf{2023/04/09}
\newline
Gli obiettivi che il gruppo si prepone per questo \textit{sprint} sono:
\begin{itemize}
    \item \textit{Refactoring} dello \textit{smart contract};
    \item Continuazione stesura della documentazione da correlare al prodotto software.
\end{itemize}
Le attività svolte sono:
\begin{itemize}
    \item Documentazione: aggiornamento \textit{NdP, PdP, Pdq};
    \item Sviluppo: struttura principale \textit{smart contract}.
\end{itemize}
\begin{figure}[H]
    \centering
    \includegraphics[width=\textwidth]{src/img/Sprint 3.png}
    \caption{Diagramma di Gantt per il terzo \textit{sprint}}
\end{figure}
\subsubsection{\rom{4} Sprint}
Dal \textbf{2023/04/10} al \textbf{2023/04/16}
\newline
Gli obiettivi che il gruppo si prepone per questo \textit{sprint} sono:
\begin{itemize}
    \item Inizio \textit{refactoring} del \textit{frontend} associato;
    \item Stesura test di unità;
    \item Continuazione stesura della documentazione da correlare al prodotto software;
\end{itemize}
Le attività svolte sono:
\begin{itemize}
    \item Documentazione: aggiornamento \textit{PdP, Pdq};
    \item Sviluppo: ampliamento \textit{frontend}.
\end{itemize}
\begin{figure}[H]
    \centering
    \includegraphics[width=\textwidth]{src/img/Sprint 4.png}
    \caption{Diagramma di Gantt per il quarto \textit{sprint}}
\end{figure}
\subsubsection{\rom{5} Sprint}
Dal \textbf{2023/04/17} al \textbf{2023/04/23}
\newline
Gli obiettivi che il gruppo si prepone per questo \textit{sprint} sono:
\begin{itemize}
    \item Stesura \textit{Specifica Tecnica};
    \item Continuazione implementazione \textit{frontend};
    \item Continuazione stesura della documentazione da correlare al prodotto software.
\end{itemize}
Le attività svolte sono:
\begin{itemize}
    \item Documentazione: stesura \textit{Specifica Tecnica} e aggiornamento \textit{PdP, Pdq};
    \item Sviluppo: integrazione del \textit{frontend} con \textit{smart contract} e \textit{API REST}.
\end{itemize}
\begin{figure}[H]
    \centering
    \includegraphics[width=\textwidth]{src/img/Sprint 5.png}
    \caption{Diagramma di Gantt per il quinto \textit{sprint}}
\end{figure}
\subsubsection{\rom{6} Sprint}
Dal \textbf{2023/04/24} al \textbf{2023/04/30}
\newline
Gli obiettivi che il gruppo si prepone per questo \textit{sprint} sono:
\begin{itemize}
    \item Revisione e avanzamento progettazione;
    \item Conclusione \textit{smart contract};
    \item Sviluppo e conclusione \textit{API REST};
    \item Stesura \textit{Manuale Utente};
    \item Continuazione stesura della documentazione da correlare al prodotto software.
\end{itemize}
Le attività svolte sono:
\begin{itemize}
    \item Documentazione: stesura \textit{Manuale Utente} e aggiornamento \textit{AdR, ST, Pdq};
    \item Sviluppo: continuazione sviluppo \textit{smart contract} e \textit{API REST};
    \item Progettazione: correzione di alcuni elementi della progettazione già reallizzata.
\end{itemize}
\begin{figure}[H]
    \centering
    \includegraphics[width=\textwidth]{src/img/Sprint 6.png}
    \caption{Diagramma di Gantt per il sesto \textit{sprint}}
\end{figure}
\subsubsection{\rom{7} Sprint}
Dal \textbf{2023/05/01} al \textbf{2023/05/07}
\newline
Gli obiettivi che il gruppo si prepone per questo \textit{sprint} sono:
\begin{itemize}
    \item Verifica finale dei test;
    \item Conclusione stesura della documentazione da correlare al prodotto software.
\end{itemize}
Le attività svolte sono:
\begin{itemize}
    \item Documentazione: aggiornamento e verifica finale di tutti i documenti prodotti;
    \item Test: esecuzione di tutti i test stabiliti.
\end{itemize}
\begin{figure}[H]
    \centering
    \includegraphics[width=\textwidth]{src/img/Sprint 7.png}
    \caption{Diagramma di Gantt per il settimo \textit{sprint}}
\end{figure}

% \subsection{VALIDAZIONE???}

\section{Preventivo dei costi}
\subsection{Analisi}
\subsubsection{Prospetto orario}
\rowcolors{2}{pari_alt}{dispari_alt}
\renewcommand{\arraystretch}{1.8}

\begin{xltabular}{\textwidth} {
        >{\hsize=1.70\hsize\linewidth=\hsize}X
        >{\hsize=0.5\hsize\linewidth=\hsize}X
        >{\hsize=0.5\hsize\linewidth=\hsize}X
        >{\hsize=0.5\hsize\linewidth=\hsize}X
        >{\hsize=0.5\hsize\linewidth=\hsize}X
        >{\hsize=0.5\hsize\linewidth=\hsize}X
        >{\hsize=0.5\hsize\linewidth=\hsize}X
        >{\hsize=0.8\hsize\linewidth=\hsize}X
    }
    \rowcolorhead
    \textbf{\color{white}Componente} &
    \textbf{\color{white}Re} &
    \textbf{\color{white}Pt} &
    \textbf{\color{white}An} &
    \textbf{\color{white}Am} &
    \textbf{\color{white}Pr} &
    \textbf{\color{white}Ve} &
    \textbf{\color{white}Totale} \\
    \hline
    \endfirsthead

    \hline
    \rowcolorhead
    \textbf{\color{white}Componente} &
    \textbf{\color{white}Re} &
    \textbf{\color{white}Pt} &
    \textbf{\color{white}An} &
    \textbf{\color{white}Am} &
    \textbf{\color{white}Pr} &
    \textbf{\color{white}Ve} &
    \textbf{\color{white}Totale} \\
    \hline
    \endhead

    \endfoot

    \endlastfoot

    Elia Pasquali & 4 & 0 & 11 & 7 & 0 & 6 & 28 \\
    Ennio Italiano & 4 & 0 & 13 & 6 & 0 & 7 & 30 \\
    Enrico Bacci Bonivento & 4 & 0 & 11 & 6 & 0 & 7 & 28 \\
    Fabio Pantaleo & 4 & 0 & 11 & 6 & 0 & 7 & 28 \\
    Nicolò Trinca & 4 & 0 & 12 & 7 & 0 & 6 & 29 \\
    Sebastiano Sanson & 6 & 0 & 12 & 7 & 0 & 6 & 31 \\
    Totale & 26 & 0 & 70 & 39 & 0 & 39 & 174\\
    % & & & & & & &
    \rowcolor{white}
    \caption{Distribuzione delle ore nel periodo di analisi}
\end{xltabular}

\subsubsection{Prospetto economico}
\rowcolors{2}{pari_alt}{dispari_alt}
\renewcommand{\arraystretch}{1.8}

\begin{xltabular}{\textwidth} {
        >{\hsize=1\hsize\linewidth=\hsize}X
        >{\hsize=0.5\hsize\linewidth=\hsize}X
        >{\hsize=0.5\hsize\linewidth=\hsize}X
    }
    \rowcolorhead
    \textbf{\color{white}Ruolo} &
    \textbf{\color{white}Totale ore} &
    \textbf{\color{white}Costo totale} \\
    \hline
    \endfirsthead

    \hline
    \rowcolorhead
    \textbf{\color{white}Ruolo} &
    \textbf{\color{white}Totale ore} &
    \textbf{\color{white}Costo totale} \\
    \hline
    \endhead

    \endfoot

    \endlastfoot

    Responsabile & 26 & 780 \\
    Progettista & 0 & 0 \\
    Analista & 70 & 1750\\
    Amministratore & 39 & 780 \\
    Programmatore & 0 & 0  \\
    Verificatore & 39 & 585 \\
    Totale & 174 & 3895 \\
    % & & & & & & &
    \rowcolor{white}
    \caption{Prospetto dei costi per ruolo nel periodo di analisi}
\end{xltabular}

\subsection{Proof of Concept}
\subsubsection{Prospetto orario}
\rowcolors{2}{pari_alt}{dispari_alt}
\renewcommand{\arraystretch}{1.8}

\begin{xltabular}{\textwidth} {
        >{\hsize=1.70\hsize\linewidth=\hsize}X
        >{\hsize=0.5\hsize\linewidth=\hsize}X
        >{\hsize=0.5\hsize\linewidth=\hsize}X
        >{\hsize=0.5\hsize\linewidth=\hsize}X
        >{\hsize=0.5\hsize\linewidth=\hsize}X
        >{\hsize=0.5\hsize\linewidth=\hsize}X
        >{\hsize=0.5\hsize\linewidth=\hsize}X
        >{\hsize=0.8\hsize\linewidth=\hsize}X
    }
    \rowcolorhead
    \textbf{\color{white}Componente} &
    \textbf{\color{white}Re} &
    \textbf{\color{white}Pt} &
    \textbf{\color{white}An} &
    \textbf{\color{white}Am} &
    \textbf{\color{white}Pr} &
    \textbf{\color{white}Ve} &
    \textbf{\color{white}Totale} \\
    \hline
    \endfirsthead

    \hline
    \rowcolorhead
    \textbf{\color{white}Componente} &
    \textbf{\color{white}Re} &
    \textbf{\color{white}Pt} &
    \textbf{\color{white}An} &
    \textbf{\color{white}Am} &
    \textbf{\color{white}Pr} &
    \textbf{\color{white}Ve} &
    \textbf{\color{white}Totale} \\
    \hline
    \endhead

    \endfoot

    \endlastfoot

    Elia Pasquali & 2 & 4 & 2 & 5 & 5 & 2 & 20 \\
    Ennio Italiano & 1 & 4 & 2 & 6 & 6 & 2 & 21 \\
    Enrico Bacci Bonivento & 2 & 5 & 1 & 4 & 6 & 2 & 20 \\
    Fabio Pantaleo & 1 & 6 & 1 & 2 & 7 & 2 & 19 \\
    Nicolò Trinca & 1 & 5 & 0 & 3 & 8 & 2 & 19 \\
    Sebastiano Sanson & 2 & 6 & 0 & 3 & 6 & 2 & 19 \\
    Totale & 9 & 30 & 6 & 23 & 38 & 12 & 118 \\
    % & & & & & & &
    \rowcolor{white}
    \caption{Distribuzione delle ore nel periodo di \textit{Proof of Concept}}
\end{xltabular}

\subsubsection{Prospetto economico}
\rowcolors{2}{pari_alt}{dispari_alt}
\renewcommand{\arraystretch}{1.8}

\begin{xltabular}{\textwidth} {
        >{\hsize=1\hsize\linewidth=\hsize}X
        >{\hsize=0.5\hsize\linewidth=\hsize}X
        >{\hsize=0.5\hsize\linewidth=\hsize}X
    }
    \rowcolorhead
    \textbf{\color{white}Ruolo} &
    \textbf{\color{white}Totale ore} &
    \textbf{\color{white}Costo totale} \\
    \hline
    \endfirsthead

    \hline
    \rowcolorhead
    \textbf{\color{white}Ruolo} &
    \textbf{\color{white}Totale ore} &
    \textbf{\color{white}Costo totale} \\
    \hline
    \endhead

    \endfoot

    \endlastfoot

    Responsabile & 9 & 270 \\
    Progettista & 30 & 750 \\
    Analista & 6 & 150\\
    Amministratore & 23 & 460 \\
    Programmatore & 38 & 570  \\
    Verificatore & 12 & 180 \\
    Totale & 118 & 2380 \\
    % & & & & & & &
    \rowcolor{white}
    \caption{Prospetto dei costi per ruolo nel periodo di \textit{Proof of Concept}}
\end{xltabular}

\subsection{Progettazione di dettaglio e codifica dei requisiti}
\subsubsection{Sprint 1}
\paragraph{Prospetto orario}

\rowcolors{2}{pari_alt}{dispari_alt}
\renewcommand{\arraystretch}{1.8}

\begin{xltabular}{\textwidth} {
        >{\hsize=1.70\hsize\linewidth=\hsize}X
        >{\hsize=0.5\hsize\linewidth=\hsize}X
        >{\hsize=0.5\hsize\linewidth=\hsize}X
        >{\hsize=0.5\hsize\linewidth=\hsize}X
        >{\hsize=0.5\hsize\linewidth=\hsize}X
        >{\hsize=0.5\hsize\linewidth=\hsize}X
        >{\hsize=0.5\hsize\linewidth=\hsize}X
        >{\hsize=0.8\hsize\linewidth=\hsize}X
    }
    \rowcolorhead
    \textbf{\color{white}Componente} &
    \textbf{\color{white}Re} &
    \textbf{\color{white}Pt} &
    \textbf{\color{white}An} &
    \textbf{\color{white}Am} &
    \textbf{\color{white}Pr} &
    \textbf{\color{white}Ve} &
    \textbf{\color{white}Totale} \\
    \hline
    \endfirsthead

    \hline
    \rowcolorhead
    \textbf{\color{white}Componente} &
    \textbf{\color{white}Re} &
    \textbf{\color{white}Pt} &
    \textbf{\color{white}An} &
    \textbf{\color{white}Am} &
    \textbf{\color{white}Pr} &
    \textbf{\color{white}Ve} &
    \textbf{\color{white}Totale} \\
    \hline
    \endhead

    \endfoot

    \endlastfoot

    Elia Pasquali           & 1 & 0 & 1 & 0 & 0 & 1 & 3 \\
    Ennio Italiano          & 0 & 2 & 0 & 0 & 0 & 1 & 3 \\
    Enrico Bacci Bonivento  & 0 & 1 & 0 & 0 & 1 & 1 & 3 \\
    Fabio Pantaleo          & 1 & 1 & 0 & 0 & 0 & 1 & 3 \\
    Nicolò Trinca           & 0 & 1 & 0 & 0 & 1 & 0 & 2 \\
    Sebastiano Sanson       & 0 & 1 & 1 & 1 & 0 & 0 & 3 \\
    Totale                  & 2 & 6 & 2 & 1 & 2 & 4 & 17 \\
    % & & & & & & &
    \rowcolor{white}
    \caption{Distribuzione delle ore nel primo \textit{sprint}}
\end{xltabular}

\paragraph{Prospetto economico}
\rowcolors{2}{pari_alt}{dispari_alt}
\renewcommand{\arraystretch}{1.8}

\begin{xltabular}{\textwidth} {
        >{\hsize=1\hsize\linewidth=\hsize}X
        >{\hsize=0.5\hsize\linewidth=\hsize}X
        >{\hsize=0.5\hsize\linewidth=\hsize}X
    }
    \rowcolorhead
    \textbf{\color{white}Ruolo} &
    \textbf{\color{white}Totale ore} &
    \textbf{\color{white}Costo totale} \\
    \hline
    \endfirsthead

    \hline
    \rowcolorhead
    \textbf{\color{white}Ruolo} &
    \textbf{\color{white}Totale ore} &
    \textbf{\color{white}Costo totale} \\
    \hline
    \endhead

    \endfoot

    \endlastfoot

    Responsabile & 2 & 60 \\
    Progettista & 6 & 150 \\
    Analista & 2 & 50 \\
    Amministratore & 1 & 20 \\
    Programmatore & 2 & 30  \\
    Verificatore & 4 & 60 \\
    Totale & 17 & 370 \\
    % & & & & & & &
    \rowcolor{white}
    \caption{Prospetto dei costi per ruolo nel primo \textit{sprint}}
\end{xltabular}
\subsubsection{Sprint 2}
\paragraph{Prospetto orario}

\rowcolors{2}{pari_alt}{dispari_alt}
\renewcommand{\arraystretch}{1.8}

\begin{xltabular}{\textwidth} {
        >{\hsize=1.70\hsize\linewidth=\hsize}X
        >{\hsize=0.5\hsize\linewidth=\hsize}X
        >{\hsize=0.5\hsize\linewidth=\hsize}X
        >{\hsize=0.5\hsize\linewidth=\hsize}X
        >{\hsize=0.5\hsize\linewidth=\hsize}X
        >{\hsize=0.5\hsize\linewidth=\hsize}X
        >{\hsize=0.5\hsize\linewidth=\hsize}X
        >{\hsize=0.8\hsize\linewidth=\hsize}X
    }
    \rowcolorhead
    \textbf{\color{white}Componente} &
    \textbf{\color{white}Re} &
    \textbf{\color{white}Pt} &
    \textbf{\color{white}An} &
    \textbf{\color{white}Am} &
    \textbf{\color{white}Pr} &
    \textbf{\color{white}Ve} &
    \textbf{\color{white}Totale} \\
    \hline
    \endfirsthead

    \hline
    \rowcolorhead
    \textbf{\color{white}Componente} &
    \textbf{\color{white}Re} &
    \textbf{\color{white}Pt} &
    \textbf{\color{white}An} &
    \textbf{\color{white}Am} &
    \textbf{\color{white}Pr} &
    \textbf{\color{white}Ve} &
    \textbf{\color{white}Totale} \\
    \hline
    \endhead

    \endfoot

    \endlastfoot

    Elia Pasquali           & 0 & 5 & 1 & 0 & 0 & 0 & 6 \\
    Ennio Italiano          & 1 & 3 & 0 & 0 & 0 & 0 & 4 \\
    Enrico Bacci Bonivento  & 1 & 3 & 0 & 0 & 0 & 0 & 4 \\
    Fabio Pantaleo          & 0 & 4 & 0 & 0 & 1 & 0 & 5 \\
    Nicolò Trinca           & 0 & 4 & 0 & 1 & 0 & 0 & 5 \\
    Sebastiano Sanson       & 0 & 4 & 1 & 0 & 0 & 0 & 5 \\
    Totale                  & 2 & 23 & 2 & 1 & 1 & 0 & 29 \\
    % & & & & & & &
    \rowcolor{white}
    \caption{Distribuzione delle ore nel secondo \textit{sprint}}
\end{xltabular}

\paragraph{Prospetto economico}
\rowcolors{2}{pari_alt}{dispari_alt}
\renewcommand{\arraystretch}{1.8}

\begin{xltabular}{\textwidth} {
        >{\hsize=1\hsize\linewidth=\hsize}X
        >{\hsize=0.5\hsize\linewidth=\hsize}X
        >{\hsize=0.5\hsize\linewidth=\hsize}X
    }
    \rowcolorhead
    \textbf{\color{white}Ruolo} &
    \textbf{\color{white}Totale ore} &
    \textbf{\color{white}Costo totale} \\
    \hline
    \endfirsthead

    \hline
    \rowcolorhead
    \textbf{\color{white}Ruolo} &
    \textbf{\color{white}Totale ore} &
    \textbf{\color{white}Costo totale} \\
    \hline
    \endhead

    \endfoot

    \endlastfoot

    Responsabile & 2 & 60 \\
    Progettista & 23 & 575 \\
    Analista & 2 & 50 \\
    Amministratore & 1 & 20 \\
    Programmatore & 1 & 15  \\
    Verificatore & 0 & 0 \\
    Totale & 29 & 670 \\
    % & & & & & & &
    \rowcolor{white}
    \caption{Prospetto dei costi per ruolo nel secondo \textit{sprint}}
\end{xltabular}
\subsubsection{Sprint 3}
\paragraph{Prospetto orario}

\rowcolors{2}{pari_alt}{dispari_alt}
\renewcommand{\arraystretch}{1.8}

\begin{xltabular}{\textwidth} {
        >{\hsize=1.70\hsize\linewidth=\hsize}X
        >{\hsize=0.5\hsize\linewidth=\hsize}X
        >{\hsize=0.5\hsize\linewidth=\hsize}X
        >{\hsize=0.5\hsize\linewidth=\hsize}X
        >{\hsize=0.5\hsize\linewidth=\hsize}X
        >{\hsize=0.5\hsize\linewidth=\hsize}X
        >{\hsize=0.5\hsize\linewidth=\hsize}X
        >{\hsize=0.8\hsize\linewidth=\hsize}X
    }
    \rowcolorhead
    \textbf{\color{white}Componente} &
    \textbf{\color{white}Re} &
    \textbf{\color{white}Pt} &
    \textbf{\color{white}An} &
    \textbf{\color{white}Am} &
    \textbf{\color{white}Pr} &
    \textbf{\color{white}Ve} &
    \textbf{\color{white}Totale} \\
    \hline
    \endfirsthead

    \hline
    \rowcolorhead
    \textbf{\color{white}Componente} &
    \textbf{\color{white}Re} &
    \textbf{\color{white}Pt} &
    \textbf{\color{white}An} &
    \textbf{\color{white}Am} &
    \textbf{\color{white}Pr} &
    \textbf{\color{white}Ve} &
    \textbf{\color{white}Totale} \\
    \hline
    \endhead

    \endfoot

    \endlastfoot

    Elia Pasquali           & 0 & 3 & 0 & 0 & 4 & 0 & 7 \\
    Ennio Italiano          & 0 & 3 & 0 & 0 & 3 & 2 & 8 \\
    Enrico Bacci Bonivento  & 0 & 3 & 1 & 0 & 3 & 0 & 7 \\
    Fabio Pantaleo          & 0 & 3 & 0 & 1 & 4 & 0 & 8 \\
    Nicolò Trinca           & 1 & 2 & 0 & 0 & 4 & 2 & 9 \\
    Sebastiano Sanson       & 1 & 2 & 0 & 1 & 5 & 0 & 9 \\
    Totale                  & 2 & 16 & 1 & 2 & 23 & 4 & 48 \\
    % & & & & & & &
    \rowcolor{white}
    \caption{Distribuzione delle ore nel terzo \textit{sprint}}
\end{xltabular}

\paragraph{Prospetto economico}
\rowcolors{2}{pari_alt}{dispari_alt}
\renewcommand{\arraystretch}{1.8}

\begin{xltabular}{\textwidth} {
        >{\hsize=1\hsize\linewidth=\hsize}X
        >{\hsize=0.5\hsize\linewidth=\hsize}X
        >{\hsize=0.5\hsize\linewidth=\hsize}X
    }
    \rowcolorhead
    \textbf{\color{white}Ruolo} &
    \textbf{\color{white}Totale ore} &
    \textbf{\color{white}Costo totale} \\
    \hline
    \endfirsthead

    \hline
    \rowcolorhead
    \textbf{\color{white}Ruolo} &
    \textbf{\color{white}Totale ore} &
    \textbf{\color{white}Costo totale} \\
    \hline
    \endhead

    \endfoot

    \endlastfoot

    Responsabile & 2 & 60 \\
    Progettista & 16 & 400 \\
    Analista & 1 & 25 \\
    Amministratore & 2 & 40 \\
    Programmatore & 23 & 345  \\
    Verificatore & 4 & 60 \\
    Totale & 48 & 930 \\
    % & & & & & & &
    \rowcolor{white}
    \caption{Prospetto dei costi per ruolo nel terzo \textit{sprint}}
\end{xltabular}
\subsubsection{Sprint 4}
\paragraph{Prospetto orario}

\rowcolors{2}{pari_alt}{dispari_alt}
\renewcommand{\arraystretch}{1.8}

\begin{xltabular}{\textwidth} {
        >{\hsize=1.70\hsize\linewidth=\hsize}X
        >{\hsize=0.5\hsize\linewidth=\hsize}X
        >{\hsize=0.5\hsize\linewidth=\hsize}X
        >{\hsize=0.5\hsize\linewidth=\hsize}X
        >{\hsize=0.5\hsize\linewidth=\hsize}X
        >{\hsize=0.5\hsize\linewidth=\hsize}X
        >{\hsize=0.5\hsize\linewidth=\hsize}X
        >{\hsize=0.8\hsize\linewidth=\hsize}X
    }
    \rowcolorhead
    \textbf{\color{white}Componente} &
    \textbf{\color{white}Re} &
    \textbf{\color{white}Pt} &
    \textbf{\color{white}An} &
    \textbf{\color{white}Am} &
    \textbf{\color{white}Pr} &
    \textbf{\color{white}Ve} &
    \textbf{\color{white}Totale} \\
    \hline
    \endfirsthead

    \hline
    \rowcolorhead
    \textbf{\color{white}Componente} &
    \textbf{\color{white}Re} &
    \textbf{\color{white}Pt} &
    \textbf{\color{white}An} &
    \textbf{\color{white}Am} &
    \textbf{\color{white}Pr} &
    \textbf{\color{white}Ve} &
    \textbf{\color{white}Totale} \\
    \hline
    \endhead

    \endfoot

    \endlastfoot

    Elia Pasquali           & 0 & 3 & 0 & 0 & 4 & 0 & 7 \\
    Ennio Italiano          & 0 & 3 & 0 & 1 & 5 & 0 & 9 \\
    Enrico Bacci Bonivento  & 1 & 3 & 0 & 1 & 5 & 0 & 9 \\
    Fabio Pantaleo          & 1 & 3 & 1 & 0 & 4 & 2 & 11 \\
    Nicolò Trinca           & 0 & 4 & 0 & 0 & 4 & 0 & 8 \\
    Sebastiano Sanson       & 0 & 4 & 0 & 0 & 4 & 2 & 10 \\
    Totale                  & 2 & 20 & 1 & 2 & 26 & 4 & 55 \\
    % & & & & & & &
    \rowcolor{white}
    \caption{Distribuzione delle ore nel quarto \textit{sprint}}
\end{xltabular}

\paragraph{Prospetto economico}
\rowcolors{2}{pari_alt}{dispari_alt}
\renewcommand{\arraystretch}{1.8}

\begin{xltabular}{\textwidth} {
        >{\hsize=1\hsize\linewidth=\hsize}X
        >{\hsize=0.5\hsize\linewidth=\hsize}X
        >{\hsize=0.5\hsize\linewidth=\hsize}X
    }
    \rowcolorhead
    \textbf{\color{white}Ruolo} &
    \textbf{\color{white}Totale ore} &
    \textbf{\color{white}Costo totale} \\
    \hline
    \endfirsthead

    \hline
    \rowcolorhead
    \textbf{\color{white}Ruolo} &
    \textbf{\color{white}Totale ore} &
    \textbf{\color{white}Costo totale} \\
    \hline
    \endhead

    \endfoot

    \endlastfoot

    Responsabile & 2 & 60 \\
    Progettista & 20 & 500 \\
    Analista & 1 & 25 \\
    Amministratore & 2 & 40 \\
    Programmatore & 26 & 390  \\
    Verificatore & 4 & 60 \\
    Totale & 48 & 1075 \\
    % & & & & & & &
    \rowcolor{white}
    \caption{Prospetto dei costi per ruolo nel quarto \textit{sprint}}
\end{xltabular}
\subsubsection{Sprint 5}
\paragraph{Prospetto orario}

\rowcolors{2}{pari_alt}{dispari_alt}
\renewcommand{\arraystretch}{1.8}

\begin{xltabular}{\textwidth} {
        >{\hsize=1.70\hsize\linewidth=\hsize}X
        >{\hsize=0.5\hsize\linewidth=\hsize}X
        >{\hsize=0.5\hsize\linewidth=\hsize}X
        >{\hsize=0.5\hsize\linewidth=\hsize}X
        >{\hsize=0.5\hsize\linewidth=\hsize}X
        >{\hsize=0.5\hsize\linewidth=\hsize}X
        >{\hsize=0.5\hsize\linewidth=\hsize}X
        >{\hsize=0.8\hsize\linewidth=\hsize}X
    }
    \rowcolorhead
    \textbf{\color{white}Componente} &
    \textbf{\color{white}Re} &
    \textbf{\color{white}Pt} &
    \textbf{\color{white}An} &
    \textbf{\color{white}Am} &
    \textbf{\color{white}Pr} &
    \textbf{\color{white}Ve} &
    \textbf{\color{white}Totale} \\
    \hline
    \endfirsthead

    \hline
    \rowcolorhead
    \textbf{\color{white}Componente} &
    \textbf{\color{white}Re} &
    \textbf{\color{white}Pt} &
    \textbf{\color{white}An} &
    \textbf{\color{white}Am} &
    \textbf{\color{white}Pr} &
    \textbf{\color{white}Ve} &
    \textbf{\color{white}Totale} \\
    \hline
    \endhead

    \endfoot

    \endlastfoot

    Elia Pasquali           & 1 & 0 & 0 & 1 & 3 & 1 & 6 \\
    Ennio Italiano          & 1 & 0 & 0 & 0 & 3 & 1 & 5 \\
    Enrico Bacci Bonivento  & 0 & 0 & 0 & 0 & 4 & 0 & 4 \\
    Fabio Pantaleo          & 0 & 0 & 0 & 0 & 4 & 0 & 4 \\
    Nicolò Trinca           & 0 & 0 & 1 & 1 & 3 & 1 & 6 \\
    Sebastiano Sanson       & 0 & 0 & 0 & 0 & 3 & 1 & 4 \\
    Totale                  & 2 & 0 & 1 & 2 & 20 & 4 & 29 \\
    % & & & & & & &
    \rowcolor{white}
    \caption{Distribuzione delle ore nel quinto \textit{sprint}}
\end{xltabular}

\paragraph{Prospetto economico}
\rowcolors{2}{pari_alt}{dispari_alt}
\renewcommand{\arraystretch}{1.8}

\begin{xltabular}{\textwidth} {
        >{\hsize=1\hsize\linewidth=\hsize}X
        >{\hsize=0.5\hsize\linewidth=\hsize}X
        >{\hsize=0.5\hsize\linewidth=\hsize}X
    }
    \rowcolorhead
    \textbf{\color{white}Ruolo} &
    \textbf{\color{white}Totale ore} &
    \textbf{\color{white}Costo totale} \\
    \hline
    \endfirsthead

    \hline
    \rowcolorhead
    \textbf{\color{white}Ruolo} &
    \textbf{\color{white}Totale ore} &
    \textbf{\color{white}Costo totale} \\
    \hline
    \endhead

    \endfoot

    \endlastfoot

    Responsabile & 2 & 60 \\
    Progettista & 0 & 0 \\
    Analista & 1 & 25 \\
    Amministratore & 2 & 40 \\
    Programmatore & 20 & 300  \\
    Verificatore & 4 & 60 \\
    Totale & 48 & 485 \\
    % & & & & & & &
    \rowcolor{white}
    \caption{Prospetto dei costi per ruolo nel quinto \textit{sprint}}
\end{xltabular}
\subsubsection{Sprint 6}
\paragraph{Prospetto orario}

\rowcolors{2}{pari_alt}{dispari_alt}
\renewcommand{\arraystretch}{1.8}

\begin{xltabular}{\textwidth} {
        >{\hsize=1.70\hsize\linewidth=\hsize}X
        >{\hsize=0.5\hsize\linewidth=\hsize}X
        >{\hsize=0.5\hsize\linewidth=\hsize}X
        >{\hsize=0.5\hsize\linewidth=\hsize}X
        >{\hsize=0.5\hsize\linewidth=\hsize}X
        >{\hsize=0.5\hsize\linewidth=\hsize}X
        >{\hsize=0.5\hsize\linewidth=\hsize}X
        >{\hsize=0.8\hsize\linewidth=\hsize}X
    }
    \rowcolorhead
    \textbf{\color{white}Componente} &
    \textbf{\color{white}Re} &
    \textbf{\color{white}Pt} &
    \textbf{\color{white}An} &
    \textbf{\color{white}Am} &
    \textbf{\color{white}Pr} &
    \textbf{\color{white}Ve} &
    \textbf{\color{white}Totale} \\
    \hline
    \endfirsthead

    \hline
    \rowcolorhead
    \textbf{\color{white}Componente} &
    \textbf{\color{white}Re} &
    \textbf{\color{white}Pt} &
    \textbf{\color{white}An} &
    \textbf{\color{white}Am} &
    \textbf{\color{white}Pr} &
    \textbf{\color{white}Ve} &
    \textbf{\color{white}Totale} \\
    \hline
    \endhead

    \endfoot

    \endlastfoot

    Elia Pasquali           & 0 & 1 & 0 & 0 & 3 & 1 & 4 \\
    Ennio Italiano          & 0 & 2 & 0 & 0 & 4 & 1 & 7 \\
    Enrico Bacci Bonivento  & 1 & 0 & 1 & 0 & 3 & 1 & 6\\
    Fabio Pantaleo          & 1 & 0 & 0 & 1 & 3 & 1 & 6 \\
    Nicolò Trinca           & 0 & 2 & 0 & 0 & 5 & 0 & 7 \\
    Sebastiano Sanson       & 0 & 2 & 1 & 1 & 5 & 0 & 9 \\
    Totale                  & 2 & 7 & 1 & 2 & 23 & 4 & 39 \\
    % & & & & & & &
    \rowcolor{white}
    \caption{Distribuzione delle ore nel sesto \textit{sprint}}
\end{xltabular}

\paragraph{Prospetto economico}
\rowcolors{2}{pari_alt}{dispari_alt}
\renewcommand{\arraystretch}{1.8}

\begin{xltabular}{\textwidth} {
        >{\hsize=1\hsize\linewidth=\hsize}X
        >{\hsize=0.5\hsize\linewidth=\hsize}X
        >{\hsize=0.5\hsize\linewidth=\hsize}X
    }
    \rowcolorhead
    \textbf{\color{white}Ruolo} &
    \textbf{\color{white}Totale ore} &
    \textbf{\color{white}Costo totale} \\
    \hline
    \endfirsthead

    \hline
    \rowcolorhead
    \textbf{\color{white}Ruolo} &
    \textbf{\color{white}Totale ore} &
    \textbf{\color{white}Costo totale} \\
    \hline
    \endhead

    \endfoot

    \endlastfoot

    Responsabile & 2 & 60 \\
    Progettista & 7 & 175 \\
    Analista & 1 & 25 \\
    Amministratore & 2 & 40 \\
    Programmatore & 23 & 345  \\
    Verificatore & 4 & 60 \\
    Totale & 48 & 705 \\
    % & & & & & & &
    \rowcolor{white}
    \caption{Prospetto dei costi per ruolo nel sesto \textit{sprint}}
\end{xltabular}
\subsubsection{Sprint 7}
\paragraph{Prospetto orario}

\rowcolors{2}{pari_alt}{dispari_alt}
\renewcommand{\arraystretch}{1.8}

\begin{xltabular}{\textwidth} {
        >{\hsize=1.70\hsize\linewidth=\hsize}X
        >{\hsize=0.5\hsize\linewidth=\hsize}X
        >{\hsize=0.5\hsize\linewidth=\hsize}X
        >{\hsize=0.5\hsize\linewidth=\hsize}X
        >{\hsize=0.5\hsize\linewidth=\hsize}X
        >{\hsize=0.5\hsize\linewidth=\hsize}X
        >{\hsize=0.5\hsize\linewidth=\hsize}X
        >{\hsize=0.8\hsize\linewidth=\hsize}X
    }
    \rowcolorhead
    \textbf{\color{white}Componente} &
    \textbf{\color{white}Re} &
    \textbf{\color{white}Pt} &
    \textbf{\color{white}An} &
    \textbf{\color{white}Am} &
    \textbf{\color{white}Pr} &
    \textbf{\color{white}Ve} &
    \textbf{\color{white}Totale} \\
    \hline
    \endfirsthead

    \hline
    \rowcolorhead
    \textbf{\color{white}Componente} &
    \textbf{\color{white}Re} &
    \textbf{\color{white}Pt} &
    \textbf{\color{white}An} &
    \textbf{\color{white}Am} &
    \textbf{\color{white}Pr} &
    \textbf{\color{white}Ve} &
    \textbf{\color{white}Totale} \\
    \hline
    \endhead

    \endfoot

    \endlastfoot

    Elia Pasquali           & 0 & 0 & 0 & 0 & 1 & 2 & 3 \\
    Ennio Italiano          & 1 & 0 & 0 & 0 & 1 & 1 & 3 \\
    Enrico Bacci Bonivento  & 0 & 0 & 0 & 1 & 0 & 2 & 3\\
    Fabio Pantaleo          & 0 & 0 & 0 & 1 & 0 & 2 & 3 \\
    Nicolò Trinca           & 1 & 0 & 0 & 0 & 1 & 1 & 3 \\
    Sebastiano Sanson       & 0 & 0 & 0 & 0 & 1 & 2 & 3 \\
    Totale                  & 2 & 0 & 0 & 2 & 4 & 10 & 18 \\
    % & & & & & & &
    \rowcolor{white}
    \caption{Distribuzione delle ore nel settimo \textit{sprint}}
\end{xltabular}

\paragraph{Prospetto economico}
\rowcolors{2}{pari_alt}{dispari_alt}
\renewcommand{\arraystretch}{1.8}

\begin{xltabular}{\textwidth} {
        >{\hsize=1\hsize\linewidth=\hsize}X
        >{\hsize=0.5\hsize\linewidth=\hsize}X
        >{\hsize=0.5\hsize\linewidth=\hsize}X
    }
    \rowcolorhead
    \textbf{\color{white}Ruolo} &
    \textbf{\color{white}Totale ore} &
    \textbf{\color{white}Costo totale} \\
    \hline
    \endfirsthead

    \hline
    \rowcolorhead
    \textbf{\color{white}Ruolo} &
    \textbf{\color{white}Totale ore} &
    \textbf{\color{white}Costo totale} \\
    \hline
    \endhead

    \endfoot

    \endlastfoot

    Responsabile & 2 & 60 \\
    Progettista & 0 & 0 \\
    Analista & 0 & 0 \\
    Amministratore & 2 & 40 \\
    Programmatore & 4 & 60  \\
    Verificatore & 10 & 150 \\
    Totale & 48 & 310 \\
    % & & & & & & &
    \rowcolor{white}
    \caption{Prospetto dei costi per ruolo nel settimo \textit{sprint}}
\end{xltabular}
\pagebreak