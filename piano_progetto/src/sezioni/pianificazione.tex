\section{Pianificazione}
\groupName  ha deciso di suddividere la pianificazione di progetto in quattro fasi:
\begin{itemize}
    \item Analisi;
    \item Progettazione di dettaglio e codifica;
    \item Validazione e collaudo.
\end{itemize}

\subsection{Analisi}
Periodo: dal \textbf{2022/11/04} al \textbf{2023/02/08} \newline
Questo periodo ha inizio con l'assegnazione del capitolato d'appalto e termina con l'inizio del periodo di \textit{Proof of Concept}.
Prima di tutto vengono definiti gli strumenti per la comunicazione del gruppo e gli strumenti da utilizzare per il lavoro collaborativo e
per la redazione dei documenti. Successivamente si analizza il capitolato per individuare i requisiti fondamentali, comunicando anche con 
il proponente.
Inoltre si è deciso di svolgere una fase di autoformazione per studiare ed apprendere le nuove tecnologie da utilizzare per lo sviluppo del prodotto.
Le attività di questo periodo sono:
\begin{itemize}
    \item \textbf{Autoformazione}: i membri del gruppo si impegnano ad apprendere le nuove tecnologie scelte;
    \item \textbf{Analisi dei requisiti}: l'\roleAnalyst, attraverso lo studio del capitolato e la comunicazione con il proponente,
            individua i requisiti iniziali del prodotto e ne ricava le funzioni principali. Le funzioni sono rappresentate nel documento 
            con l'ausilio di diagrammi UML\glo. I requisiti potrebbero evolvere nel tempo in base ai feedback del proponente;
    \item \textbf{Norme di progetto}: in questo documento vengono sono decisi gli strumenti da utilizzare nello sviluppo del prodotto 
            e le regole a cui il team dovrà attenersi per la stesura di tutti i documenti. L'\roleAdministrator  ha il compito di emanare
            le norme;
    \item \textbf{Piano di progetto}: il \roleProjectManager  illustra un prospetto di pianificazione dettagliata, con attività e compiti, 
                            a cui il gruppo dovrà attenersi;
    \item \textbf{Piano di qualifica}: il seguente documento ha lo scopo di decidere le procedura per la verifica della qualità e
            offre una visione sull'esito delle verifiche sul prodotto e i suoi componenti;
    \item \textbf{Glossario}: al fine di garantire chiarezza ed evitare ambiguità, viene redatto un glossario contenente tutti i termini necessari di definizione.
\end{itemize}
\subsubsection{\rom{1} Periodo}
Dal \textbf{2022/11/14} al \textbf{2022/11/19}
\newline
Nei primi giorni dell'analisi si discutono le norme per il lavoro collaborativo e si decidono i primi strumenti da utilizzare per
la stesura e la condivisione della documentazione relativa al prodotto. Inoltre si redigono i primi verbali relativi agli incontri
del gruppo.

\subsubsection{\rom{2} Periodo} 
Dal \textbf{2022/11/20} al \textbf{2023/01/28}
\newline
In questo periodo viene redatta la documentazione del prodotto. Il gruppo decide di tenere degli incontri settimanali per discutere di
eventuali problemi riscontrati e per verificare i progressi fatti. La comunicazione con il proponente avviene tramite un canale 
Discord\glo dedicato.

\subsubsection{\rom{3} Periodo} 
Dal \textbf{2023/01/29} al \textbf{2023/02/05}
\newline
Nell'ultimo periodo di analisi viene verificata la correttezza della documentazione basandosi anche su quanto scritto 
sulle \textit{Norme di progetto}. Viene aggiornato il \textit{Glossario} con gli ultimi termini.

\begin{figure}[H]
    \centering
    \includegraphics[scale=0.3]{src/img/Gantt analisi.png}
    \caption{Diagramma di Gantt per l'attività di analisi}
\end{figure}

\subsection{Proof of concept}
Periodo: dal \textbf{2023/02/06} al \textbf{2023/02/20} \newline
Questo periodo inizia al termine della verifica della documentazione e la sua fine coincide con la scadenza
di consegna dei documenti per la revisione di \textit{Requirements and Technology Baseline\glo} (RTB\glo).
Al termine di questo periodo verrà organizzato un incontro con il proponente per presentare un prototipo del prodotto.\newline
Si hanno due attività:
\begin{itemize}
        \item \textbf{Proof of Concept}: viene realizzato un \textit{Proof of Concept} che dovrà implementare la maggior 
        parte delle tecnologie necessarie e svolgerà alcune funzioni principali del prodotto. Il PoC\glo è un dimostrabile eseguibile
        e verrà usato come base di partenza per gli incrementi futuri.
        \item \textbf{Modifiche e verifica sui documenti}: i documenti redatti durante la fase di analisi vengono aggiornati e migliorati.
\end{itemize}
\subsubsection{\rom{1} Periodo}
Dal \textbf{2023/02/06} al \textbf{2023/02/08}
\newline
Il gruppo si impegna a studiare ed apprendere il funzionamento delle tecnologie utili e necessarie per la produzione del \textit{PoC}

\subsubsection{\rom{2} Periodo}
Dal \textbf{2023/02/09} al \textbf{2023/02/17}
\newline
Viene sviluppato il \textit{PoC} secondo le tecnologie scelte e si implementeranno le funzioni principali del prodotto.

\subsubsection{\rom{3} Periodo}
Dal \textbf{2023/02/18} al \textbf{2023/02/20}
\newline
Nell'ultimo periodo viene redatta la presentazione per la \textit{Requirements and Technology Baseline}

\begin{figure}[H]
    \centering
    \includegraphics[scale=0.4]{src/img/Gantt PoC.png}
    \caption{Diagramma di Gantt dell'attività di PoC}
\end{figure}

\subsection{Progettazione di dettaglio e codifica}
Periodo: dal \textbf{2023/02/27} al \textbf{2023/03/27} \newline
Questo periodo inizia solo se si è superata la fase di \textit{Requirements and Technology Baseline}. Viene utilizzato il \textit{PoC}
come base di partenza per il prodotto. Le attività che compongono questo periodo sono:
\begin{itemize}
        \item \textbf{Product Baseline\glo}: presenta l'architettura del prodotto attraverso il diagramma delle classi.
        \item \textbf{Codifica}: i programmatori sviluppano il codice delle funzionalità del prodotto, aggiornando e migliorando quello già presente nel \textit{PoC}.
        \item \textbf{Test}: vengono sviluppati i test di unità.
        \item \textbf{Manuali}: Redazione del \textit{Manuale utente} e del \textit{Manuale sviluppatore} per l'utilizzo del prodotto.
        \item \textbf{Modifiche ai documenti}: i documenti redatti durante le fasi precedenti vengono aggiornati e migliorati.
\end{itemize}

\subsubsection{\rom{1} Periodo}
Dal \textbf{2023/02/27} al \textbf{2023/03/03}
\newline
In questo primo periodo il gruppo sviluppa i diagrammi delle classi e applica i design pattern della \textit{Product Baseline}.
\subsubsection{\rom{2} Periodo}
Dal \textbf{2023/03/04} al \textbf{2023/03/20}
\newline
Questi giorni sono dedicati alla codifica del codice e all'implementazione di tutte le funzionalità. Si comincia la stesura di 
\textit{Manuale utente} e \textit{Manuale sviluppatore}.
\subsubsection{\rom{3} Periodo}
Dal \textbf{2023/03/21} al \textbf{2023/03/23}
\newline
I eseguono i test di unità codificati dai programmatori.
\subsubsection{\rom{4} Periodo}
Dal \textbf{2023/03/24} al \textbf{2023/03/27}
\newline
Si sviluppa la presentazione della \textit{Product Baseline}

\begin{figure}[H]
    \centering
    \includegraphics[scale=0.32]{src/img/Gantt progettazione.png}
    \caption{Diagramma di Gantt dell'attività di progettazione}
\end{figure}

% \subsection{VALIDAZIONE???}

\section{Preventivo dei costi}
\subsection{Analisi}
\subsubsection{Prospetto orario}
\rowcolors{2}{pari_alt}{dispari_alt}
\renewcommand{\arraystretch}{1.8}

\begin{xltabular}{\textwidth} {
    >{\hsize=1.70\hsize\linewidth=\hsize}X
    >{\hsize=0.5\hsize\linewidth=\hsize}X
    >{\hsize=0.5\hsize\linewidth=\hsize}X
    >{\hsize=0.5\hsize\linewidth=\hsize}X
    >{\hsize=0.5\hsize\linewidth=\hsize}X
    >{\hsize=0.5\hsize\linewidth=\hsize}X
    >{\hsize=0.5\hsize\linewidth=\hsize}X
    >{\hsize=0.8\hsize\linewidth=\hsize}X
    }
    \rowcolorhead
    \textbf{\color{white}Componente} &
    \textbf{\color{white}Re} &
    \textbf{\color{white}Pt} &
    \textbf{\color{white}An} &
    \textbf{\color{white}Am} &
    \textbf{\color{white}Pr} &
    \textbf{\color{white}Ve} &
    \textbf{\color{white}Totale} \\
    \hline
    \endfirsthead

    \hline
    \rowcolorhead
    \textbf{\color{white}Componente} &
    \textbf{\color{white}Re} &
    \textbf{\color{white}Pt} &
    \textbf{\color{white}An} &
    \textbf{\color{white}Am} &
    \textbf{\color{white}Pr} &
    \textbf{\color{white}Ve} &
    \textbf{\color{white}Totale} \\
    \hline
    \endhead

    \endfoot

    \endlastfoot

    Elia Pasquali & 4 & 0 & 11 & 7 & 0 & 6 & 28 \\
    Ennio Italiano & 4 & 0 & 13 & 6 & 0 & 7 & 30 \\
    Enrico Bacci Bonivento & 4 & 0 & 11 & 6 & 0 & 7 & 28 \\
    Fabio Pantaleo & 4 & 0 & 11 & 6 & 0 & 7 & 28 \\
    Nicolò Trinca & 4 & 0 & 12 & 7 & 0 & 6 & 29 \\
    Sebastiano Sanson & 6 & 0 & 12 & 7 & 0 & 6 & 31 \\
    Totale & 26 & 0 & 70 & 39 & 0 & 39 & 174\\
    % & & & & & & &
    \rowcolor{white}
    \caption{Distribuzione delle ore nel periodo di analisi}
\end{xltabular}

\subsubsection{Prospetto economico}
\rowcolors{2}{pari_alt}{dispari_alt}
\renewcommand{\arraystretch}{1.8}

\begin{xltabular}{\textwidth} {
    >{\hsize=1\hsize\linewidth=\hsize}X
    >{\hsize=0.5\hsize\linewidth=\hsize}X
    >{\hsize=0.5\hsize\linewidth=\hsize}X
    }
    \rowcolorhead
    \textbf{\color{white}Ruolo} &
    \textbf{\color{white}Totale ore} &
    \textbf{\color{white}Costo totale} \\
    \hline
    \endfirsthead

    \hline
    \rowcolorhead
    \textbf{\color{white}Ruolo} &
    \textbf{\color{white}Totale ore} &
    \textbf{\color{white}Costo totale} \\
    \hline
    \endhead

    \endfoot

    \endlastfoot

    Responsabile & 26 & 780 \\
    Progettista & 0 & 0 \\
    Analista & 70 & 1750\\
    Amministratore & 39 & 780 \\
    Programmatore & 0 & 0  \\
    Verificatore & 39 & 585 \\ 
    Totale & 174 & 3895 \\
    % & & & & & & &
    \rowcolor{white}
    \caption{Prospetto dei costi per ruolo nel periodo di analisi}
\end{xltabular}

\subsection{Proof of concept}
\subsubsection{Prospetto orario}
\rowcolors{2}{pari_alt}{dispari_alt}
\renewcommand{\arraystretch}{1.8}

\begin{xltabular}{\textwidth} {
    >{\hsize=1.70\hsize\linewidth=\hsize}X
    >{\hsize=0.5\hsize\linewidth=\hsize}X
    >{\hsize=0.5\hsize\linewidth=\hsize}X
    >{\hsize=0.5\hsize\linewidth=\hsize}X
    >{\hsize=0.5\hsize\linewidth=\hsize}X
    >{\hsize=0.5\hsize\linewidth=\hsize}X
    >{\hsize=0.5\hsize\linewidth=\hsize}X
    >{\hsize=0.8\hsize\linewidth=\hsize}X
    }
    \rowcolorhead
    \textbf{\color{white}Componente} &
    \textbf{\color{white}Re} &
    \textbf{\color{white}Pt} &
    \textbf{\color{white}An} &
    \textbf{\color{white}Am} &
    \textbf{\color{white}Pr} &
    \textbf{\color{white}Ve} &
    \textbf{\color{white}Totale} \\
    \hline
    \endfirsthead

    \hline
    \rowcolorhead
    \textbf{\color{white}Componente} &
    \textbf{\color{white}Re} &
    \textbf{\color{white}Pt} &
    \textbf{\color{white}An} &
    \textbf{\color{white}Am} &
    \textbf{\color{white}Pr} &
    \textbf{\color{white}Ve} &
    \textbf{\color{white}Totale} \\
    \hline
    \endhead

    \endfoot

    \endlastfoot

    Elia Pasquali & 2 & 4 & 2 & 5 & 5 & 2 & 20 \\
    Ennio Italiano & 1 & 4 & 2 & 6 & 6 & 2 & 21 \\
    Enrico Bacci Bonivento & 2 & 5 & 1 & 4 & 6 & 2 & 20 \\
    Fabio Pantaleo & 1 & 6 & 1 & 2 & 7 & 2 & 19 \\
    Nicolò Trinca & 1 & 5 & 0 & 3 & 8 & 2 & 19 \\
    Sebastiano Sanson & 2 & 6 & 0 & 3 & 6 & 2 & 19 \\
    Totale & 9 & 30 & 6 & 23 & 38 & 12 & 118 \\
    % & & & & & & &
    \rowcolor{white}
    \caption{Distribuzione delle ore nel periodo di Proof of Concept}
\end{xltabular}

\subsubsection{Prospetto economico}
\rowcolors{2}{pari_alt}{dispari_alt}
\renewcommand{\arraystretch}{1.8}

\begin{xltabular}{\textwidth} {
    >{\hsize=1\hsize\linewidth=\hsize}X
    >{\hsize=0.5\hsize\linewidth=\hsize}X
    >{\hsize=0.5\hsize\linewidth=\hsize}X
    }
    \rowcolorhead
    \textbf{\color{white}Ruolo} &
    \textbf{\color{white}Totale ore} &
    \textbf{\color{white}Costo totale} \\
    \hline
    \endfirsthead

    \hline
    \rowcolorhead
    \textbf{\color{white}Ruolo} &
    \textbf{\color{white}Totale ore} &
    \textbf{\color{white}Costo totale} \\
    \hline
    \endhead

    \endfoot

    \endlastfoot

    Responsabile & 9 & 270 \\
    Progettista & 30 & 750 \\
    Analista & 6 & 150\\
    Amministratore & 23 & 460 \\
    Programmatore & 38 & 570  \\
    Verificatore & 12 & 180 \\ 
    Totale & 118 & 2380 \\
    % & & & & & & &
    \rowcolor{white}
    \caption{Prospetto dei costi per ruolo nel periodo di Proof of Concept}
\end{xltabular}

\subsection{Progettazione di dettaglio e codifica dei requisiti}
\subsubsection{Prospetto orario}
\rowcolors{2}{pari_alt}{dispari_alt}
\renewcommand{\arraystretch}{1.8}

\begin{xltabular}{\textwidth} {
    >{\hsize=1.70\hsize\linewidth=\hsize}X
    >{\hsize=0.5\hsize\linewidth=\hsize}X
    >{\hsize=0.5\hsize\linewidth=\hsize}X
    >{\hsize=0.5\hsize\linewidth=\hsize}X
    >{\hsize=0.5\hsize\linewidth=\hsize}X
    >{\hsize=0.5\hsize\linewidth=\hsize}X
    >{\hsize=0.5\hsize\linewidth=\hsize}X
    >{\hsize=0.8\hsize\linewidth=\hsize}X
    }
    \rowcolorhead
    \textbf{\color{white}Componente} &
    \textbf{\color{white}Re} &
    \textbf{\color{white}Pt} &
    \textbf{\color{white}An} &
    \textbf{\color{white}Am} &
    \textbf{\color{white}Pr} &
    \textbf{\color{white}Ve} &
    \textbf{\color{white}Totale} \\
    \hline
    \endfirsthead

    \hline
    \rowcolorhead
    \textbf{\color{white}Componente} &
    \textbf{\color{white}Re} &
    \textbf{\color{white}Pt} &
    \textbf{\color{white}An} &
    \textbf{\color{white}Am} &
    \textbf{\color{white}Pr} &
    \textbf{\color{white}Ve} &
    \textbf{\color{white}Totale} \\
    \hline
    \endhead

    \endfoot

    \endlastfoot

    Elia Pasquali           & 2 & 11 & 1 & 2 & 16 & 5 & 37 \\
    Ennio Italiano          & 2 & 12 & 2 & 2 & 17 & 5 & 40 \\
    Enrico Bacci Bonivento  & 2 & 11 & 1 & 2 & 16 & 5 & 37 \\
    Fabio Pantaleo          & 3 & 12 & 1 & 2 & 16 & 5 & 39 \\
    Nicolò Trinca           & 3 & 13 & 2 & 2 & 16 & 5 & 41 \\
    Sebastiano Sanson       & 2 & 13 & 1 & 2 & 18 & 5 & 41 \\
    Totale                  & 14 & 72 & 8 & 12 & 99 & 30 & 235 \\
    % & & & & & & &
    \rowcolor{white}
    \caption{Distribuzione delle ore nel periodo di progettazione di dettaglio e codifica dei requisiti }
\end{xltabular}

\subsubsection{Prospetto economico}
\rowcolors{2}{pari_alt}{dispari_alt}
\renewcommand{\arraystretch}{1.8}

\begin{xltabular}{\textwidth} {
    >{\hsize=1\hsize\linewidth=\hsize}X
    >{\hsize=0.5\hsize\linewidth=\hsize}X
    >{\hsize=0.5\hsize\linewidth=\hsize}X
    }
    \rowcolorhead
    \textbf{\color{white}Ruolo} &
    \textbf{\color{white}Totale ore} &
    \textbf{\color{white}Costo totale} \\
    \hline
    \endfirsthead

    \hline
    \rowcolorhead
    \textbf{\color{white}Ruolo} &
    \textbf{\color{white}Totale ore} &
    \textbf{\color{white}Costo totale} \\
    \hline
    \endhead

    \endfoot

    \endlastfoot

    Responsabile & 14 & 420 \\
    Progettista & 72 & 1800 \\
    Analista & 8 & 200 \\
    Amministratore & 12 & 240 \\
    Programmatore & 99 & 1485  \\
    Verificatore & 30 & 450 \\ 
    Totale & 235 & 4595 \\
    % & & & & & & &
    \rowcolor{white}
    \caption{Prospetto dei costi per ruolo nel periodo di progettazione di dettaglio e codifica dei requisiti}
\end{xltabular}
