\section{Analisi dei rischi}
Analizzare, prevedere e gestire i rischi è fondamentale per la realizzazione del prodotto. \\
Si attuano le seguenti modalità per una corretta interpretazione dei rischi:
\begin{itemize}
    \item \textbf{Identificazione}: si cercano tutti i possibili rischi;
    \item \textbf{Analisi}: si studia il rischio e le eventuali conseguenze;
    \item \textbf{Piano di contingenza}: fornisce un piano da attuare in caso si verifichi un rischio previsto;
    \item \textbf{Controllo}: si utilizzano indicatori per il controllo continuo dei rischi.
\end{itemize}
\noindent
Per classificare la probabilità di occorrenza e la pericolosità dei rischi si utilizzano le seguenti sigle:
\begin{itemize}
    \item \textbf{A}: alta probabilità/pericolosità;
    \item \textbf{M}: media probabilità/pericolosità;
    \item \textbf{B}: bassa probabilità/pericolosità.
\end{itemize}
\pagebreak
\subsection{Rischi tecnologici}
\rowcolors{2}{pari_alt}{dispari_alt}
\renewcommand{\arraystretch}{1.8}

\begin{xltabular}{\textwidth} {
    >{\hsize=0.5\hsize\linewidth=\hsize}X
    >{\hsize=1\hsize\linewidth=\hsize}X
    >{\hsize=1.70\hsize\linewidth=\hsize}X
    >{\hsize=0.5\hsize\linewidth=\hsize}X
    >{\hsize=0.5\hsize\linewidth=\hsize}X
    }
    \rowcolorhead
    \textbf{\color{white}Codice} &
    \textbf{\color{white}Tipo} &
    \textbf{\color{white}Descrizione} &
    \textbf{\color{white}Prob} &
    \textbf{\color{white}Pericolo} \\
    \hline
    \endfirsthead

    \hline
    \rowcolorhead
    \textbf{\color{white}Codice} &
    \textbf{\color{white}Tipo} &
    \textbf{\color{white}Descrizione} &
    \textbf{\color{white}Prob} &
    \textbf{\color{white}Pericolo} \\
    \hline
    \endhead

    \endfoot

    \endlastfoot
    RT1 &  Inesperienza tecnologica & Dipende dal grado di conoscenza dei membri delle nuove tecnologie e il loro apprendimento. & A & M \\
    \hline
    RT2 & Problemi software e hardware & È dovuto a problemi hardware o software che i membri del gruppo potrebbero avere. & M & A \\
    \hline
    \rowcolor{white}
    \caption{Rischi tecnologici}\\
\end{xltabular}




\subsection{Rischi interni}
\rowcolors{2}{pari_alt}{dispari_alt}
\renewcommand{\arraystretch}{1.8}

\begin{xltabular}{\textwidth} {
    >{\hsize=0.5\hsize\linewidth=\hsize}X
    >{\hsize=1\hsize\linewidth=\hsize}X
    >{\hsize=1.70\hsize\linewidth=\hsize}X
    >{\hsize=0.5\hsize\linewidth=\hsize}X
    >{\hsize=0.5\hsize\linewidth=\hsize}X
    }
    \rowcolorhead
    \textbf{\color{white}Codice} &
    \textbf{\color{white}Tipo} &
    \textbf{\color{white}Descrizione} &
    \textbf{\color{white}Prob} &
    \textbf{\color{white}Pericolo} \\
    \hline
    \endfirsthead

    \hline
    \rowcolorhead
    \textbf{\color{white}Codice} &
    \textbf{\color{white}Tipo} &
    \textbf{\color{white}Descrizione} &
    \textbf{\color{white}Prob} &
    \textbf{\color{white}Pericolo} \\
    \hline
    \endhead

    \endfoot

    \endlastfoot
    RI1 & Impegni personali & Problema dovuto a impegni personali del singolo componente dei gruppo. & M & B \\
    \hline
    RI2 & Discussioni interne & Dovuto ai diversi modi di lavorare di ogni componente. & M & B \\
    \hline
    \rowcolor{white}
    \caption{Rischi interni}

\end{xltabular}


\subsection{Rischi organizzativi}
    
\rowcolors{2}{pari_alt}{dispari_alt}
\renewcommand{\arraystretch}{1.8}

\begin{xltabular}{\textwidth} {
    >{\hsize=0.5\hsize\linewidth=\hsize}X
    >{\hsize=1\hsize\linewidth=\hsize}X
    >{\hsize=1.70\hsize\linewidth=\hsize}X
    >{\hsize=0.5\hsize\linewidth=\hsize}X
    >{\hsize=0.5\hsize\linewidth=\hsize}X
    }
    \rowcolorhead
    \textbf{\color{white}Codice} &
    \textbf{\color{white}Tipo} &
    \textbf{\color{white}Descrizione} &
    \textbf{\color{white}Prob} &
    \textbf{\color{white}Pericolo} \\
    \hline
    \endfirsthead

    \hline
    \rowcolorhead
    \textbf{\color{white}Codice} &
    \textbf{\color{white}Tipo} &
    \textbf{\color{white}Descrizione} &
    \textbf{\color{white}Prob} &
    \textbf{\color{white}Pericolo} \\
    \hline
    \endhead

    \endfoot

    \endlastfoot
    RO1 & Organizzazione dei lavori  & Le attività potrebbero essere distribuite in modo errato tra i membri. & B & M \\
    \hline
    RO2 & Risorse sprecate & Le risorse di tempo e costo potrebbero essere stimate in modo errato. & M & A \\
    \hline
    \rowcolor{white}
    \caption{Rischi organizzativi}
\end{xltabular}
\pagebreak


\subsection{Rischi sui requisiti}
    
\rowcolors{2}{pari_alt}{dispari_alt}
\renewcommand{\arraystretch}{1.8}

\begin{xltabular}{\textwidth} {
    >{\hsize=0.5\hsize\linewidth=\hsize}X
    >{\hsize=1\hsize\linewidth=\hsize}X
    >{\hsize=1.70\hsize\linewidth=\hsize}X
    >{\hsize=0.5\hsize\linewidth=\hsize}X
    >{\hsize=0.5\hsize\linewidth=\hsize}X
    }
    \rowcolorhead
    \textbf{\color{white}Codice} &
    \textbf{\color{white}Tipo} &
    \textbf{\color{white}Descrizione} &
    \textbf{\color{white}Prob} &
    \textbf{\color{white}Pericolo} \\
    \hline
    \endfirsthead

    \hline
    \rowcolorhead
    \textbf{\color{white}Codice} &
    \textbf{\color{white}Tipo} &
    \textbf{\color{white}Descrizione} &
    \textbf{\color{white}Prob} &
    \textbf{\color{white}Pericolo} \\
    \hline
    \endhead

    \endfoot

    \endlastfoot
    RR1 & Requisiti incompleti & Problema dovuto al cambiamento dei requisiti nel tempo. & B & M \\
    \hline
    RR2 & Incomprensione dei requisiti & I requisiti richiesti potrebbero essere interpretati in modo errato. & M & A \\
    \hline
    RR3 & Mancato supporto del proponente & Il proponente del prodotto potrebbe essere poco disponibile. & B & B  \\
    \hline
    \rowcolor{white}
    \caption{Rischi requisti}
\end{xltabular}


\pagebreak
\noindent
\subsection{Piano di contingenza}
La seguente tabella mostra il piano di contingenza per ogni rischio.
Come riportato sopra, ogni rischio ha un grado di pericolosità: \textbf{A} (Alto), \textbf{M} (Medio), \textbf{B} (Basso).

\rowcolors{2}{pari_alt}{dispari_alt}
\renewcommand{\arraystretch}{1.8}

\begin{xltabular}{\textwidth} {
    >{\hsize=0.5\hsize\linewidth=\hsize}X
    >{\hsize=0.5\hsize\linewidth=\hsize}X
    >{\hsize=1.70\hsize\linewidth=\hsize}X
    }
    \rowcolorhead
    \textbf{\color{white}Codice} &
    \textbf{\color{white}Pericolo} &
    \textbf{\color{white}Piano di contingenza} \\
    \hline
    \endfirsthead

    \hline
    \rowcolorhead
    \textbf{\color{white}Codice} &
    \textbf{\color{white}Pericolo} &
    \textbf{\color{white}Piano di contingenza} \\
    \hline
    \endhead

    \endfoot

    \endlastfoot
        RT1 & M & Il gruppo si impegna ad apprendere le nuove tecnologie necessarie per la realizzazione del prodotto.\\
        \hline
        RT2 & A & I membri del gruppo devono avere un dispositivo alternativo in caso di guasto hardware.
        Mentre tutti i dati sono salvati in remoto su un \textit{repository}\glo\:comune e sono accessibili in qualsiasi momento. \\
        \hline
        RI1 & B & Il \roleProjectManager, al sopraggiungere di questo rischio, riorganizza il lavoro tra i membri. \\
        \hline    
        RI2 & B & Il \roleProjectManager\:deve cercare di mettere d'accordo i membri. \\
        \hline
        RO1 & M & Il gruppo dovrà discutere per suddividere il lavoro in modo equo. \\
        \hline
        RO2 & A & Si deve svolgere un'attenta previsione di tempi e costi e, qualora risultasse sbagliata, il \roleProjectManager\:deve ridistribuire il lavoro. \\
        \hline
        RR1 & M & I membri del gruppo si adoperano per una nuova analisi dei requisiti. \\
        \hline
        RR2 & A & Il gruppo dedica attenzione alla comprensione dei requisiti e cerca un riscontro con il proponente. \\
        \hline
        RR3 & B & Il gruppo deve trovare il modo di comunicare con il proponente. \\
        \hline
    \rowcolor{white}
    \caption{Piano di contingenza}
    \end{xltabular}
    

\pagebreak
