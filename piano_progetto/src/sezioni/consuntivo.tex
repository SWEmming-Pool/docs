\section{Consuntivo di periodo}
Di seguito vengono indicate le spese effettivamente sostenute.
Il bilancio potrà essere quindi:
\begin{itemize}
    \item \textbf{Positivo} se il preventivo supera il consuntivo;
    \item \textbf{Pari} se preventivo e consuntivo hanno ugual valore;
    \item \textbf{Negativo} se il consuntivo supera il preventivo.
\end{itemize}
\subsection{Periodo di analisi}
\rowcolors{2}{pari_alt}{dispari_alt}
\renewcommand{\arraystretch}{1.8}

\begin{xltabular}{\textwidth} {
    >{\hsize=1\hsize\linewidth=\hsize}X
    >{\hsize=0.5\hsize\linewidth=\hsize}X
    >{\hsize=0.5\hsize\linewidth=\hsize}X
    }
    \rowcolorhead
    \textbf{\color{white}Ruolo} &
    \textbf{\color{white}Totale ore} &
    \textbf{\color{white}Costo totale} \\
    \hline
    \endfirsthead

    \hline
    \rowcolorhead
    \textbf{\color{white}Ruolo} &
    \textbf{\color{white}Totale ore} &
    \textbf{\color{white}Costo totale} \\
    \hline
    \endhead

    \endfoot

    \endlastfoot

    Responsabile & 26(+0) & 780(+0) \\
    Progettista & 0 & 0 \\
    Analista & 70(+4) & 1750(+100)\\
    Amministratore & 39(+12) & 780(+240) \\
    Programmatore & 0 & 0  \\
    Verificatore & 39(+2) & 585(+30) \\ 
    Totale preventivo & 174 & 3895 \\
    Totale consuntivo & 192 & 4165\\
    Differenza & 18 & +270 \\

    % & & & & & & &
    \rowcolor{white}
    \caption{Consuntivo della fase di analisi }
\end{xltabular}
\subsubsection{Conclusioni}
Dal consuntivo di analisi emerge che i ruoli che hanno richiesto un investimento maggiore di ore rispetto a quanto preventivato sono l'analista, l'amministratore e il verificatore. I motivi di tali aumenti sono stati i seguenti:
\begin{itemize}
    \item \roleAnalyst: alcuni requisiti si sono rivelati di non facile comprensione, e sono state necessarie più ore di lavoro per la discussione interna ed esterna;
    \item \roleAdministrator: la stesura di alcune sezioni delle \textit{Norme di Progetto} ha subito rallentamenti causati dalla comprensione di determinate problematiche;
    \item \roleVerifier: l'\textit{Analisi dei Requisiti} e le \textit{Norme di Progetto} hanno subito notevoli variazioni nel corso del progetto, questo perciò ha implicato un ulteriore verifica di essi.
\end{itemize}

\subsection{Periodo di Proof of Concept}
\rowcolors{2}{pari_alt}{dispari_alt}
\renewcommand{\arraystretch}{1.8}

\begin{xltabular}{\textwidth} {
    >{\hsize=1\hsize\linewidth=\hsize}X
    >{\hsize=0.5\hsize\linewidth=\hsize}X
    >{\hsize=0.5\hsize\linewidth=\hsize}X
    }
    \rowcolorhead
    \textbf{\color{white}Ruolo} &
    \textbf{\color{white}Totale ore} &
    \textbf{\color{white}Costo totale} \\
    \hline
    \endfirsthead

    \hline
    \rowcolorhead
    \textbf{\color{white}Ruolo} &
    \textbf{\color{white}Totale ore} &
    \textbf{\color{white}Costo totale} \\
    \hline
    \endhead

    \endfoot

    \endlastfoot

    Responsabile & 9(+1) & 270(+30) \\
    Progettista & 30(+6) & 750(+150) \\
    Analista & 6(+2) & 150(+50)\\
    Amministratore & 23(+2) & 460(+40) \\
    Programmatore & 38(+8) & 570(+120)  \\
    Verificatore & 12(+6) & 180 (+90)\\ 
    Totale preventivo & 118 & 2380 \\
    Totale consuntivo & 143 & 2860\\
    Differenza & 25 & 480 \\

    % & & & & & & &
    \rowcolor{white}
    \caption{Consuntivo della fase del \textit{PoC}}
\end{xltabular}
\subsubsection{Conclusioni}
Dal consuntivo del \textit{PoC} emerge che i ruoli che hanno richiesto un investimento maggiore di ore rispetto a quanto preventivato sono il responsabile, il progettista, l'analista, l'amministratore, il programmatore e il verificatore. I motivi di tali aumenti sono stati i seguenti:
\begin{itemize}
    \item \roleProjectManager: il responsabile ha dovuto occuparsi di problematiche che non erano state previste, e che hanno richiesto un investimento di tempo maggiore;
    \item \roleDesigner: data la mole di difficoltà richiesta nei diagrammi il ruolo è risultato più complesso del previsto;
    \item \roleAnalyst: alcuni requisiti sono variati nel tempo;
    \item \roleAdministrator: la gestione dei documenti ha richiesto un maggior investimento di tempo;
    \item \roleProgrammer: a causa della scarsa esperienza con le nuove tecnologie il lavoro non è risultato efficiente come previsto;
    \item \roleVerifier: la verifica dei documenti ha richiesto un maggior investimento di tempo.
\end{itemize}

\subsection{Periodo di progettazione di dettaglio e codifica}
Attraverso l'uso degli \textit{sprint} il gruppo è riuscito a tenere monitorata la situazione economica/oraria del progetto, effettuando all'occorrenza variazioni sui periodi futuri.
\subsubsection{Sprint 1}
\rowcolors{2}{pari_alt}{dispari_alt}
\renewcommand{\arraystretch}{1.8}

\begin{xltabular}{\textwidth} {
    >{\hsize=1\hsize\linewidth=\hsize}X
    >{\hsize=0.5\hsize\linewidth=\hsize}X
    >{\hsize=0.5\hsize\linewidth=\hsize}X
    }
    \rowcolorhead
    \textbf{\color{white}Ruolo} &
    \textbf{\color{white}Totale ore} &
    \textbf{\color{white}Costo totale} \\
    \hline
    \endfirsthead

    \hline
    \rowcolorhead
    \textbf{\color{white}Ruolo} &
    \textbf{\color{white}Totale ore} &
    \textbf{\color{white}Costo totale} \\
    \hline
    \endhead

    \endfoot

    \endlastfoot

    Responsabile & 2(+0) & 60(+0) \\
    Progettista & 6(+1) & 150(+25) \\
    Analista & 2(+0) & 50(+0)\\
    Amministratore & 1(+0) & 20(+0) \\
    Programmatore & 2(+0) & 30(+0)  \\
    Verificatore & 4(+0) & 60 (+0)\\ 
    Totale preventivo & 17 & 370 \\
    Totale consuntivo & 18 & 395\\
    Differenza & 1 & 25 \\

    % & & & & & & &
    \rowcolor{white}
    \caption{Consuntivo del primo \textit{sprint}}
\end{xltabular}
\paragraph{Conclusioni}~

\noindent Il primo \textit{sprint} in linea di massima ha rispecchiato la previsione iniziale in quanto l'attività principale svolta consisteva nell'effettuare correzioni mirate alla documentazione già prodotta.
Tuttavia la fase di progettazione ha richiesto un coinvolgimento maggiore che ha portato il gruppo ad un leggero ritardo.
Il preventivo perciò risulta essere in negativo di 25€.



\subsubsection{Sprint 2}
\rowcolors{2}{pari_alt}{dispari_alt}
\renewcommand{\arraystretch}{1.8}

\begin{xltabular}{\textwidth} {
    >{\hsize=1\hsize\linewidth=\hsize}X
    >{\hsize=0.5\hsize\linewidth=\hsize}X
    >{\hsize=0.5\hsize\linewidth=\hsize}X
    }
    \rowcolorhead
    \textbf{\color{white}Ruolo} &
    \textbf{\color{white}Totale ore} &
    \textbf{\color{white}Costo totale} \\
    \hline
    \endfirsthead

    \hline
    \rowcolorhead
    \textbf{\color{white}Ruolo} &
    \textbf{\color{white}Totale ore} &
    \textbf{\color{white}Costo totale} \\
    \hline
    \endhead

    \endfoot

    \endlastfoot

    Responsabile & 2(+0) & 60(+0) \\
    Progettista & 23(+2) & 575(+50) \\
    Analista & 2(+0) & 50(+0)\\
    Amministratore & 1(+0) & 20(+0) \\
    Programmatore & 1(+0) & 15(+0)  \\
    Verificatore & 0(+0) & 0 (+0)\\ 
    Totale preventivo & 29 & 670 \\
    Totale consuntivo & 31 & 720\\
    Differenza & 2 & 50 \\

    % & & & & & & &
    \rowcolor{white}
    \caption{Consuntivo del secondo \textit{sprint}}
\end{xltabular}
\paragraph{Conclusioni}~

\noindent Dopo aver effettuato l'incontro con il proponente per la risoluzione di alcuni dubbi il gruppo si è trovato a dover rivedere la struttura generale dell'applicativo. Questo ha causato una rivisitazione obbligata di alcune componenti già parzialmente sviluppate. Ciò ha comportato un maggiore impegno nel ruolo di progettista, per un costo complessivo di 50€.

Il gruppo si ritiene soddisfatto dell'incontro svolto, in quanto è stato utile per individuare prontamente errori che avrebbero potuto causare ritardi ben maggiori negli \textit{sprint} futuri.

\subsubsection{Sprint 3}
\rowcolors{2}{pari_alt}{dispari_alt}
\renewcommand{\arraystretch}{1.8}

\begin{xltabular}{\textwidth} {
    >{\hsize=1\hsize\linewidth=\hsize}X
    >{\hsize=0.5\hsize\linewidth=\hsize}X
    >{\hsize=0.5\hsize\linewidth=\hsize}X
    }
    \rowcolorhead
    \textbf{\color{white}Ruolo} &
    \textbf{\color{white}Totale ore} &
    \textbf{\color{white}Costo totale} \\
    \hline
    \endfirsthead

    \hline
    \rowcolorhead
    \textbf{\color{white}Ruolo} &
    \textbf{\color{white}Totale ore} &
    \textbf{\color{white}Costo totale} \\
    \hline
    \endhead

    \endfoot

    \endlastfoot

    Responsabile & 2(+0) & 60(+0) \\
    Progettista & 16(+1) & 400(+25) \\
    Analista & 1(+0) & 25(+0)\\
    Amministratore & 2(+0) & 40(+0) \\
    Programmatore & 23(+2) & 345(+30)  \\
    Verificatore & 4(+0) & 60 (+0)\\ 
    Totale preventivo & 48 & 930 \\
    Totale consuntivo & 51 & 985\\
    Differenza & 3 & 55 \\

    % & & & & & & &
    \rowcolor{white}
    \caption{Consuntivo del terzo \textit{sprint}}
\end{xltabular}
\paragraph{Conclusioni}~

\noindent In questo \textit{sprint} si è continuata la progettazione e si è iniziato lo sviluppo delle prime funzionalità dello \textit{smart contract}. Ci si è trovati ad affrontare un tipo di programmazione nuovo che ha richiesto uno studio preliminare e non ha avuto uno sviluppo fluido come previsto. Si rileva quindi un dispendio di 2 ore aggiuntive nel ruolo di programmatore e di 1 ora aggiuntiva per il ruolo di progettista.
Il preventivo risulta quindi in negativo di 55€.
\subsubsection{Sprint 4}
\rowcolors{2}{pari_alt}{dispari_alt}
\renewcommand{\arraystretch}{1.8}

\begin{xltabular}{\textwidth} {
    >{\hsize=1\hsize\linewidth=\hsize}X
    >{\hsize=0.5\hsize\linewidth=\hsize}X
    >{\hsize=0.5\hsize\linewidth=\hsize}X
    }
    \rowcolorhead
    \textbf{\color{white}Ruolo} &
    \textbf{\color{white}Totale ore} &
    \textbf{\color{white}Costo totale} \\
    \hline
    \endfirsthead

    \hline
    \rowcolorhead
    \textbf{\color{white}Ruolo} &
    \textbf{\color{white}Totale ore} &
    \textbf{\color{white}Costo totale} \\
    \hline
    \endhead

    \endfoot

    \endlastfoot

    Responsabile & 2(+0) & 60(+0) \\
    Progettista & 20(+0) & 500(+0) \\
    Analista & 1(+0) & 25(+0)\\
    Amministratore & 2(+0) & 40(+0) \\
    Programmatore & 26(+0) & 390(+0)  \\
    Verificatore & 4(+0) & 60 (+0)\\ 
    Totale preventivo & 48 & 1075 \\
    Totale consuntivo & 48 & 1075\\
    Differenza & 0 & 0 \\

    % & & & & & & &
    \rowcolor{white}
    \caption{Consuntivo del quarto \textit{sprint}}
\end{xltabular}
\paragraph{Conclusioni}~

\noindent In questo \textit{sprint}, essendo a cavallo delle vacanze pasquali, sono state previste meno attività sapendo che ciascun membro avrebbe avuto meno tempo a disposizione da dedicare allo sviluppo del progetto. Le attività programmate sono state comunque rispettate pienamente e non si riscontrano ritardi.
\subsubsection{Sprint 5}
\rowcolors{2}{pari_alt}{dispari_alt}
\renewcommand{\arraystretch}{1.8}

\begin{xltabular}{\textwidth} {
    >{\hsize=1\hsize\linewidth=\hsize}X
    >{\hsize=0.5\hsize\linewidth=\hsize}X
    >{\hsize=0.5\hsize\linewidth=\hsize}X
    }
    \rowcolorhead
    \textbf{\color{white}Ruolo} &
    \textbf{\color{white}Totale ore} &
    \textbf{\color{white}Costo totale} \\
    \hline
    \endfirsthead

    \hline
    \rowcolorhead
    \textbf{\color{white}Ruolo} &
    \textbf{\color{white}Totale ore} &
    \textbf{\color{white}Costo totale} \\
    \hline
    \endhead

    \endfoot

    \endlastfoot

    Responsabile & 2(+0) & 60(+0) \\
    Progettista & 0(+0) & 0(+0) \\
    Analista & 1(+0) & 25(+0)\\
    Amministratore & 2(+1) & 40(+20) \\
    Programmatore & 20(+4) & 300(+60)  \\
    Verificatore & 4(+1) & 60 (+15)\\ 
    Totale preventivo & 48 & 485 \\
    Totale consuntivo & 54 & 580\\
    Differenza & 6 & 95 \\

    % & & & & & & &
    \rowcolor{white}
    \caption{Consuntivo del quinto \textit{sprint}}
\end{xltabular}
\paragraph{Conclusioni}~

\noindent In questo \textit{sprint} si è cominciata la stesura della \textit{Specifica tecnica} e continuato lo sviluppo della parte di \textit{frontend}.

Integrare il frontend con le varie parti di \textit{API REST} e \textit{smart contract} ha messo in difficoltà i membri del gruppo che si sono dovuti interfacciare con codice scritto da colleghi, che ha richiesto uno studio maggiore di quanto preventivato.
Si rileva perciò un ritardo di 6 ore complessive divise tra il ruolo di amministratore, programmatore e verificatore causato probabilmente dall'inesperienza del gruppo.
Il preventivo ha quindi subito un incremento di 95€ sul costo finale.

\subsubsection{Sprint 6}
\rowcolors{2}{pari_alt}{dispari_alt}
\renewcommand{\arraystretch}{1.8}

\begin{xltabular}{\textwidth} {
    >{\hsize=1\hsize\linewidth=\hsize}X
    >{\hsize=0.5\hsize\linewidth=\hsize}X
    >{\hsize=0.5\hsize\linewidth=\hsize}X
    }
    \rowcolorhead
    \textbf{\color{white}Ruolo} &
    \textbf{\color{white}Totale ore} &
    \textbf{\color{white}Costo totale} \\
    \hline
    \endfirsthead

    \hline
    \rowcolorhead
    \textbf{\color{white}Ruolo} &
    \textbf{\color{white}Totale ore} &
    \textbf{\color{white}Costo totale} \\
    \hline
    \endhead

    \endfoot

    \endlastfoot

    Responsabile & 2(+2) & 60(+30) \\
    Progettista & 7(+5) & 175(+125) \\
    Analista & 1(+0) & 25(+0)\\
    Amministratore & 2(+0) & 40(+0) \\
    Programmatore & 23(+1) & 345(+15)  \\
    Verificatore & 4(+0) & 60 (+0)\\ 
    Totale preventivo & 48 & 705 \\
    Totale consuntivo & 56 & 875\\
    Differenza & 8 & 170 \\

    % & & & & & & &
    \rowcolor{white}
    \caption{Consuntivo del sesto \textit{sprint}}
\end{xltabular}
\paragraph{Conclusioni}~

\noindent In questo \textit{sprint}, dopo un colloquio con il \Cardin, si sono riscontrate alcune inesattezze nella progettazione di dettaglio e perciò è stato richiesto un intervento su di essa.
Questo ha portato ad un ritardo di 8 ore complessive divise tra il ruolo di responsabile, progettista e programmatore.
In una riunione interna il gruppo ha perciò riconosciuto l'errore di non aver svolto il colloquio con il docente in una fase precedente e ha dovuto comunicare al \Vardanega un ritardo nella consegna.
\subsubsection{Sprint 7}
\rowcolors{2}{pari_alt}{dispari_alt}
\renewcommand{\arraystretch}{1.8}

\begin{xltabular}{\textwidth} {
    >{\hsize=1\hsize\linewidth=\hsize}X
    >{\hsize=0.5\hsize\linewidth=\hsize}X
    >{\hsize=0.5\hsize\linewidth=\hsize}X
    }
    \rowcolorhead
    \textbf{\color{white}Ruolo} &
    \textbf{\color{white}Totale ore} &
    \textbf{\color{white}Costo totale} \\
    \hline
    \endfirsthead

    \hline
    \rowcolorhead
    \textbf{\color{white}Ruolo} &
    \textbf{\color{white}Totale ore} &
    \textbf{\color{white}Costo totale} \\
    \hline
    \endhead

    \endfoot

    \endlastfoot

    Responsabile & 2(+0) & 60(+0) \\
    Progettista & 0(+1) & 0(+25) \\
    Analista & 0(+0) & 0(+0)\\
    Amministratore & 2(+0) & 40(+0) \\
    Programmatore & 4(+0) & 60(+0)  \\
    Verificatore & 10(+2) & 150 (+30)\\ 
    Totale preventivo & 48 & 310 \\
    Totale consuntivo & 50 & 340\\
    Differenza & 2 & 30 \\

    % & & & & & & &
    \rowcolor{white}
    \caption{Consuntivo del settimo \textit{sprint}}
\end{xltabular}
\paragraph{Conclusioni}~

\noindent In questo \textit{sprint} sono stati effettuati i test e le verifiche finali che hanno portato alla correzione di piccole inesattezze sia nella documentazione che nel codice.
Si è rilevato un ritardo nel ruolo di verificatore a causa delle sopracitate modifiche.
\subsection{Consuntivo a finire}
\rowcolors{2}{pari_alt}{dispari_alt}
\renewcommand{\arraystretch}{1.8}

\begin{xltabular}{\textwidth} {
    >{\hsize=1\hsize\linewidth=\hsize}X
    >{\hsize=0.5\hsize\linewidth=\hsize}X
    >{\hsize=0.5\hsize\linewidth=\hsize}X
    }
    \rowcolorhead
    \textbf{\color{white}Ruolo} &
    \textbf{\color{white}Totale ore} &
    \textbf{\color{white}Costo totale} \\
    \hline
    \endfirsthead

    \hline
    \rowcolorhead
    \textbf{\color{white}Ruolo} &
    \textbf{\color{white}Totale ore} &
    \textbf{\color{white}Costo totale} \\
    \hline
    \endhead

    \endfoot

    \endlastfoot

    Responsabile & 49(+3) & 1470(+90) \\
    Progettista & 102(+16) & 2550(+400) \\
    Analista & 84(+6) & 2100(+150)\\
    Amministratore & 74(+15) & 1480(+300) \\
    Programmatore & 137(+15) & 2055(+225)  \\
    Verificatore & 81(+11) & 1215 (+165)\\ 
    Totale preventivo & 527 & 11970 \\
    Totale consuntivo & 603 & 13300\\
    Differenza & 66 & 1330 \\

    % & & & & & & &
    \rowcolor{white}
    \caption{Consuntivo a finire}

\end{xltabular}

\paragraph{Conclusioni}~

    Il progetto svolto ha portato un ritardo complessivo di 66 ore per una differenza totale di 1330€ rispetto a quanto preventivato ad ottobre 2022.
    Dopo un'attenta analisi effettuata al completamento del progetto il gruppo ha riconosciuto il notevole ritardo nella consegna e ha individuato le principali cause nell'approccio alle nuove tecnologie e nell'inesperienza nella pianificazione di attività di questa mole su un intervallo di tempo prolungato.





