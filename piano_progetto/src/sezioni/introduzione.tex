\section{Introduzione}

\subsection{Scopo del documento}
Il \textit{Piano di Progetto} è utilizzato per la pianificazione delle attività necessarie per la realizzazione del prodotto, per prevedere tutti i possibili problemi che si potrebbero incontrare e per stimare tempi e costi del progetto.

\subsection{Scopo del prodotto}
Il prodotto ha lo scopo di garantire la veridicità e la affidabilità
delle recensioni rilasciate dagli utenti a un servizio. L'interfaccia verrà sviluppata con \textit{Angular}\glo, mentre per garantire veridicità e affidabilità si utilizzerà il linguaggio \textit{Solidity}\glo\ su rete \textit{Ethereum}\glo.

\subsection{Glossario}
Al fine di evitare ambiguità nella terminologia usata all'interno del seguente documento è stato redatto un glossario, in cui vengono riportate le definizioni di termini tecnici, rilevanti o con un significato particolare.

Per indicare la presenza di un termine all'interno del glossario si è scelto di contrassegnarlo con \glo; per non appesantire la lettura della documentazione verrà così contrassegnata solo la prima occorrenza di ogni termine in ciascun
documento.

Per una consultazione completa si rimanda al \textit{Glossario v2.0.0}.

\subsection{Riferimenti}
\subsubsection{Riferimenti normativi}
\begin{itemize}
    \item \textbf{\textit{Norme di progetto v2.0.0}};
    \item \textbf{Regolamento del progetto didattico}: \\
          \url{https://www.math.unipd.it/~tullio/IS-1/2022/Dispense/PD02.pdf}.
\end{itemize}
\subsubsection{Riferimenti informativi}
\begin{itemize}
    \item \textbf{\textit{Analisi dei requisiti v2.0.0}};
    \item Capitolato d'appalto C7: \textbf{Trustify - Authentic and verifiable reviews platform}: \\
          \url{https://www.math.unipd.it/~tullio/IS-1/2022/Progetto/C7.pdf} \hfill\break [Ultimo accesso: \today]; % TODO: aggiungere data valida
    \item \textbf{Il ciclo di vita del software - Slide T03 del corso di Ingegneria dei software}: \\
          \url{https://www.math.unipd.it/~tullio/IS-1/2022/Dispense/T03.pdf} \hfill\break [Ultimo accesso: \today]; % TODO: aggiungere data valida
    \item \textbf{Gestione di progetto - Slide T04 del corso di Ingegneria dei software}: \\
          \url{https://www.math.unipd.it/~tullio/IS-1/2022/Dispense/T04.pdf} \hfill\break [Ultimo accesso: \today]. % TODO: aggiungere data valida
\end{itemize}
