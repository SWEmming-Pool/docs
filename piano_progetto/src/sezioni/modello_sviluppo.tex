\section{Modello di sviluppo iniziale}
Nel periodo iniziale fino alla fase di RTB, il team ha deciso di adottare un modello di sviluppo \textbf{incrementale}. In questo
modello le funzionalità principali sono introdotte all'inizio e verranno verificate 
ad ogni incremento che si svolgerà. Un incremento prevede sempre l'introduzione di nuove
funzionalità che vengono integrate nel sistema e quest'ultimo viene verificato in modo da
controllare eventuali bug dovuti all'incremento. I principali vantaggi di questo modello sono:
\begin{itemize}
    \item Utilizzo di un approccio adattivo nel caso in cui i requisiti cambino nel tempo;
    \item In caso di errore si può sempre tornare allo stato precedente all'incremento;
    \item Favorisce il versionamento del prodotto;
    \item Riduce il rischio di fallimento.
\end{itemize} 
Di seguito, sono riportati in forma tabellare gli incrementi individuati e i relativi obiettivi raggiunti. Ogni requisito
e caso d'uso è riportato tramite il proprio identificativo. I requisiti riportati includono anche i sotto requisiti ad essi collegati.
Per maggiori dettagli consultare \textit{Analisi dei requisiti v2.0.0}.

\rowcolors{2}{pari_alt}{dispari_alt}
\renewcommand{\arraystretch}{1.8}

\begin{xltabular}{\textwidth} {
    >{\hsize=0.3\hsize\linewidth=\hsize}X
    >{\hsize=1\hsize\linewidth=\hsize}X
    >{\hsize=0.6\hsize\linewidth=\hsize \centering}X
    >{\hsize=0.6\hsize\linewidth=\hsize}X
    }
    \rowcolorhead
    \textbf{\color{white}Increm.} &
    \textbf{\color{white}Obiettivo} &
    \textbf{\color{white}Requisiti} &
    \textbf{\color{white}Caso d'uso} \\
    \hline
    \endfirsthead

    \hline
    \rowcolorhead
    \textbf{\color{white}Increm.} &
    \textbf{\color{white}Obiettivo} &
    \textbf{\color{white}Requisiti} &
    \textbf{\color{white}Casi d'uso} \\
    \hline
    \endhead

    \endfoot

    \endlastfoot

    \rom{1} & Studio della blockchain \textit{Ethereum}, confrontandola anche con altre tecnologie simili.
    Studio di \textit{Solidity} come linguaggio per lo \textit{smart contract}. & R1V12, R1V1, R2V3, R2V6, R2V7, R1V8, R1V14 & - \\
    \rom{2} & Studio di \textit{Angular} per l'interfaccia utente e studio delle \textit{API REST}\glo. & R2V5, R1Q3, R2V4, R2V7, R1V13 & - \\
    \rom{3} & Studio del capitolato nel dettaglio e inizio individuazione requisiti principali. & R1F1, R1F2, R1F3, R1F4, R1F5, R1F7, R1F8, R1F10, R1F11, R1F12, R1F13, R1F14, R1F15, R1F16 & UC01, UC03, UC05, UC07, UC07, UC10, UC11, UC13, UC13.1, UC14, UC15, UC16, UC17, UC18. \\
    % & & & & & & &
    \rowcolor{white}
    \caption{Incrementi individuati}
\end{xltabular}



\section{Modello di sviluppo attuale}
A seguito delle considerazioni fatte dal \Vardanega durante la revisione RTB il gruppo ha deciso di attuare delle modifiche rilevanti al modello di sviluppo.
Si è deciso quindi di passare da un modello incrementale ad un modello più \textit{agile} che si avvale di alcuni elementi caratteristici del framework \textit{Scrum}.
Questo modello si presta meglio alle nostre esigenze data la difficoltà riscontrata nel realizzare una pianificazione accurata con obiettivi concreti con scadenze ravvicinate.
Data l'inesperienza da parte del gruppo si terranno in considerazione le seguenti possibilità:
\begin{itemize}
    \item Dei cicli non vadano a buon fine causando lo slittamento di alcuni requisiti;
    \item Delle attività critiche blocchino le altre portando a una perdita di parallelismo all'interno del gruppo.
\end{itemize}

I cicli potranno avere cadenza settimanale o al massimo bisettimanale a seconda delle disponibilità dei componenti del gruppo. Giornalmente i componenti del gruppo partecipano ad un breve meeting di aggiornamento della durata massima di 10 minuti nella quale ogni componente riporterà:
\begin{itemize}
    \item Cosa ha fatto ieri;
    \item Cosa farà oggi;
    \item Eventuali impedimenti.
\end{itemize}
I cicli di \textit{Scrum}, detti anche \textit{Sprint}, verranno elencati nella sezione successiva.
Per ovvie ragioni il primo \textit{Sprint} partirà subito dopo la revisione \textit{Requirements and Technology Baseline} avvenuta il 17 Marzo 2023.

\subsection{Elenco Sprint}
In seguito viene riportata una tabella riassuntiva con tutti gli \textit{Sprint} individuati, con il rispettivo obiettivo raggiunto e i requisiti e casi d'uso ad esso associati. I requisiti riportati includono tutti i requisiti figli.
Ogni requisito e caso d'uso è definito tramite il suo codice identificativo, per maggiori informazioni fare riferimento all'\textit{Analisi dei Requisiti v2.0.0}.

\rowcolors{2}{pari_alt}{dispari_alt}
\renewcommand{\arraystretch}{1.8}

\begin{xltabular}{\textwidth} {
    >{\hsize=0.3\hsize\linewidth=\hsize}X
    >{\hsize=1\hsize\linewidth=\hsize}X
    >{\hsize=0.6\hsize\linewidth=\hsize \centering}X
    >{\hsize=0.6\hsize\linewidth=\hsize}X
    }
    \rowcolorhead
    \textbf{\color{white}Sprint} &
    \textbf{\color{white}Obiettivo} &
    \textbf{\color{white}Requisiti} &
    \textbf{\color{white}Caso d'uso} \\
    \hline
    \endfirsthead

    \hline
    \rowcolorhead
    \textbf{\color{white}Sprint} &
    \textbf{\color{white}Obiettivo} &
    \textbf{\color{white}Requisiti} &
    \textbf{\color{white}Casi d'uso} \\
    \hline
    \endhead

    \endfoot

    \endlastfoot

    \rom{1} & Correzione e aggiornamento della documentazione in base alle critiche ricevute nella valutazione del \Vardanega e impostazione delle attività da svolgere. & - & - \\
    \rom{2} & Studio approfondito delle tecnologie e inizio progettazione di dettaglio. & - & - \\
    \rom{3} & Stesura preliminare Smart Contract con relativi metodi principali. & R1F2, R1F4, R1V1, R2V3, R2V6, R1V8 & UC03, UC04, UC05, UC06  \\
    \rom{4} & Implementazione frontend della webapp in base all'output dello sprint precedente e stesura test.  & RIF1, R1F3, R1F4, R1F5, R1F6, R1F7, R1F8, R1F9, R1F10, R1F11, R1F12, R1F13 & UC01, UC05, UC07, UC08, UC09, UC10, UC11, UC12, UC13, UC14 \\
    \rom{5} & Inizio stesura \textit{Specifica Tecnica} e continuazione sviluppo frontend. & - & - \\
    \rom{6} & Revisione progettazione con \Cardin, sviluppo ultime funzionalità smart contract e API. & R1F14, R1F15, R1Q3, R2V4, R2V7, R1V13 & UC15, UC16 \\
    \rom{7} & Ultimata documentazione e test del prodotto. & R1Q2 & - \\
    % & & & & & & &
    \rowcolor{white}
    \caption{Sprint individuati}
\end{xltabular}

