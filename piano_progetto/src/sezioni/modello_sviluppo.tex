\section{Modello di sviluppo iniziale}
Nel periodo iniziale fino alla fase di RTB, il team ha deciso di adottare un modello di sviluppo \textbf{incrementale}. In questo
modello le funzionalità principali sono introdotte all'inizio e verranno verificate 
ad ogni incremento che si svolgerà. Un incremento prevede sempre l'introduzione di nuove
funzionalità che vengono integrate nel sistema e quest'ultimo viene verificato in modo da
controllare eventuali bug dovuti all'incremento. I principali vantaggi di questo modello sono:
\begin{itemize}
    \item utilizzo di un approccio adattivo nel caso in cui i requisiti cambino nel tempo;
    \item in caso di errore si può sempre tornare allo stato precedente all'incremento;
    \item favorisce il versionamento del prodotto;
    \item riduce il rischio di fallimento.
\end{itemize} 
Di seguito, sono riportati in forma tabellare gli incrementi individuati e i relativi obiettivi raggiunti. Ogni requisito
e caso d'uso è riportato tramite il proprio identificativo. I requisiti riportati includono anche i sotto requisiti ad essi collegati.
Per maggiori dettagli consultare \textit{Analisi dei requisiti}.

\rowcolors{2}{pari_alt}{dispari_alt}
\renewcommand{\arraystretch}{1.8}

\begin{xltabular}{\textwidth} {
    >{\hsize=0.3\hsize\linewidth=\hsize}X
    >{\hsize=1\hsize\linewidth=\hsize}X
    >{\hsize=0.6\hsize\linewidth=\hsize \centering}X
    >{\hsize=0.6\hsize\linewidth=\hsize}X
    }
    \rowcolorhead
    \textbf{\color{white}Increm.} &
    \textbf{\color{white}Obiettivo} &
    \textbf{\color{white}Requisiti} &
    \textbf{\color{white}Caso d'uso} \\
    \hline
    \endfirsthead

    \hline
    \rowcolorhead
    \textbf{\color{white}Increm.} &
    \textbf{\color{white}Obiettivo} &
    \textbf{\color{white}Requisiti} &
    \textbf{\color{white}Casi d'uso} \\
    \hline
    \endhead

    \endfoot

    \endlastfoot

    \rom{1} & Studio della blockchain Ethereum\glo, confrontandola anche con altre tecnologie simili.
    Studio di Solidity\glo come linguaggio per lo smart contract. & R1V12, R1V1, R2V3, R2V6, R2V7, R1V8, R1V14 & - \\
    \rom{2} & Studio di Angular\glo per l'interfaccia utente e studio delle API & R2V5, R1Q3, R2V4, R2V7, R1V13 & - \\
    \rom{3} & Studio del capitolato nel dettaglio e inizio individuazione requisiti principali. & R1F1, R1F2, R1F3, R1F4, R1F5, R1F7, R1F8, R1F10, R1F11, R1F12, R1F13, R1F14, R1F15, R1F16 & UC01, UC03, UC05, UC07, UC07, UC10, UC11, UC13, UC13.1, UC14, UC15, UC16, UC17, UC18. \\
    % & & & & & & &
    \rowcolor{white}
    \caption{Incrementi individuati}
\end{xltabular}



\section{Modello di sviluppo attuale}
Il modello di sviluppo adottato è quello Agile. Questo modello è caratterizzato da una filosofia che valorizza la flessibilità e la rapidità
di risposta ai cambiamenti dei requisiti. In questo modello, il cliente è coinvolto attivamente
nel processo di sviluppo, e le funzionalità vengono sviluppate in modo costante, con cicli di sviluppo
iterativi e incrementali. Ogni ciclo di sviluppo, chiamato sprint, si concentra sull'ottenimento
di valore per il cliente e prevede la realizzazione di una o più funzionalità complete e verificabili.
I principali vantaggi di questo modello sono:
\begin{itemize}
    \item una maggiore flessibilità nel rispondere ai cambiamenti dei requisiti se necessario;
    \item un coinvolgimento attivo del cliente nel processo di sviluppo;
    \item la possibilità di effettuare test frequenti e di rivedere costantemente le priorità;
    \item un focus sul valore per il cliente;
    \item una riduzione del rischio di fallimento grazie alla possibilità di effettuare test frequenti e di rivedere costantemente le priorità.
\end{itemize}