\newglossaryentry{angular}{
    name={Angular},
    description={Framework open-source per lo sviluppo di applicazioni web}
    }

\newglossaryentry{apirest}{
    name={API REST},
    description={Interfaccia di programmazione delle applicazioni conforme ai vincoli dello stile architetturale REST (REpresentational State Transfer)}
    }

\newglossaryentry{blockchain}{
    name={Blockchain},
    description={Insieme di tecnologie in cui il registro è strutturato come una catena di blocchi contenenti le transazioni e il consenso è distribuito su tutti i nodi della rete. Tutti i nodi possono partecipare al processo di validazione delle transazioni da includere nel registro}
    }

\newglossaryentry{browser}{
    name={Browser},
    description={Programma per navigare in internet che inoltra la richiesta di un documento alla rete e ne consente la visualizzazione}
    }

\newglossaryentry{Cloud}{
    name={Cloud},
    description={Server a cui si accede tramite internet e il software e i database che si eseguono su quei server}
    }

\newglossaryentry{Criptovaluta}{
    name={Criptovaluta},
    description={Valuta virtuale che costituisce una rappresentazione digitale di valore ed è utilizzata come mezzo di scambio o detenuta a scopo di investimento. Le criptovalute possono essere trasferite, conservate o negoziate elettronicamente}
    }

\newglossaryentry{CSS}{
    name={CSS},
    description={Acronimo di Cascading Style Sheets, è un linguaggio usato per gestire il design e la presentazione delle pagine web e lavora in combinazione con HTML}
    }

\newglossaryentry{Database}{
    name={Database},
    description={Raccolta organizzata di dati strutturati per renderli facilmente accessibili, gestibili e aggiornabili}
    }

\newglossaryentry{PoC}{
    name={PoC},
    description={Acronimo di Proof of Concept, è un prototipo di un prodotto o di un servizio che viene sviluppato per verificare la fattibilità tecnica e commerciale di un progetto}
    }

\newglossaryentry{E-commerce}{
    name={E-commerce},
    description={Transazione o scambio di beni e servizi effettuati mediante l'impiego della tecnologia delle telecomunicazioni e dell'informatica}
    }

\newglossaryentry{Endpoint}{
    name={Endpoint}, 
    description={Qualsiasi dispositivo che possa connettersi a internet sia fisicamente che in cloud}
    }

\newglossaryentry{UML}{
    name={UML},
    description={Acronimo di Unified Modeling Language, è un linguaggio di modellazione grafica utilizzato per la progettazione di sistemi software}
    }

\newglossaryentry{Ethereum}{
    name={Ethereum},
    description={Piattaforma decentralizzata del web 3.0 per la creazione e pubblicazioni peer-to-peer di smart contracts creati in un linguaggio di programmazione Turing-completo}
    }
    
\newglossaryentry{Framework}{
    name={Framework},
    description={Architettura logica di supporto sulla quale un software può essere progettato e realizzato}
    }

\newglossaryentry{HTML}{
    name={HTML},
    description={Acronimo di HyperText Markup Language permette di immaginare e formattare pagine collegate fra di loro attraverso link}
    }

\newglossaryentry{Internet}{
    name={Internet},
    description={Rete di comunicazione ad accesso pubblico che connette vari dispositivi o terminali in tutto il mondo}
    }

\newglossaryentry{Infura}{
    name={Infura},
    description={Piattaforma che fornisce gli strumenti per il collegamento alla blockchain Ethereum}
    }

\newglossaryentry{JavaScript}{
    name={JavaScript},
    description={Linguaggio di programmazione per la realizzazione di pagine web interattive}
    }

\newglossaryentry{Java Spring}{
    name={Java Spring},
    description={Framework open source per lo sviluppo di applicazioni su piattaforma Java}
    }

\newglossaryentry{MetaMask}{
    name={MetaMask},
    description={Portafoglio per custodire criptovalute, il quale permette all'utente di interagire con applicazioni decentralizzate basate su Ethereum.}
    }

\newglossaryentry{Nodo}{
    name={Nodo},
    description={Qualsiasi dispositivo hardware del sistema in grado di comunicare con gli altri dispositivi che fanno parte della rete}
    }

\newglossaryentry{Open source}{
    name={Open source},
    description={Software il cui codice sorgente è disponibile per essere modificato e distribuito da chiunque}
    }

\newglossaryentry{Peer-to-Peer}{
    name={Peer-to-peer},
    description={Modello di architettura logica di rete informatica in cui i nodi non sono gerarchizzati unicamente sotto forma di client o server fissi ma anche sotto forma di nodi equivalenti o "paritari" (peer), potendo fungere al contempo da client e server perso gli altri nodi terminali della rete}
    }

\newglossaryentry{Recensione}{
    name={Recensione},
    description={Forma di commento o voto a un'attività, un giudizio valutativo e interpretativo espresso in forma discorsiva e argomentativa}
    }

\newglossaryentry{Review bombing}{
    name={Review bombing},
    description={Fenomeno in cui un grande gruppo di persone, o anche una singola persona con molteplici account, lascia recensione negativa per un'attività nel tentativo di boicottarla o danneggiarla.}
    }

\newglossaryentry{RPC}{
    name={RPC},
    description={Remote Procedure Call è uno strumento centrale per realizzare strutture operative e basate sulle ripartizioni del lavoro in reti e architetture client-server}
    }

\newglossaryentry{Smart contract}{
    name={Smart contract},
    description={Programma seguito sulla blockchain di Ethereum; è una raccolta di codice e dati che risiede a un indirizzo specifico sulla blockchain Ethereum. Sono un tipo di account Ethereum, quindi hanno un saldo e possono essere oggetto di transazioni. Possono definire regole, come un normale contratto, e imporle automaticamente tramite codice.}
    }

\newglossaryentry{Software}{
    name={Software},
    description={Insieme delle componenti immateriali a livello logico/intangibile di un sistema elettronico di elaborazione.}
    }

\newglossaryentry{Solidity}{
    name={Solidity},
    description={Linguaggio orientato ad oggetti per lo sviluppo di smart contract}
    }

\newglossaryentry{Wallet}{
    name={Wallet},
    description={Portofogli elettronico per la valuta digitale}
    }

\newglossaryentry{UI}{
    name={UI},
    description={Acronimo di User Interface, è l'insieme delle interfacce grafiche che permettono all'utente di interagire con un sistema informatico}
    }

\newglossaryentry{Web app}{
    name={Web app},
    description={Dall'inglese “web application” significa applicazione web e si basa sui codici HTML, JavaScript o CSS. Non hanno bisogno di alcuna installazione, visto che sono caricate da un web server ed eseguite in un browser}
    }

\newglossaryentry{Webserver}{
    name={Webserver},
    description={Applicazione software che in esecuzione su un server è in grado di gestire le richieste di trasferimento di pagine web di un client}
    }

\newglossaryentry{Web}{
    name={Web},
    description={Sottorete di internet che riunisce i siti che permettono un sistema di navigazione ipertestuale e visualizzati per mezzo di browser}
    }

\newglossaryentry{JSON}{
    name={JSON},
    description={Acronimo di JavaScript Object Notation, è un formato adatto all'interscambio di dati fra applicazioni client/server}
    }

\newglossaryentry{Web3}{
    name={Web3},
    description={Idea per una nuova iterazione del World Wide Web, che include concetti come decentralizzazione, tecnologie blockchain ed economia basata su token}
    }

\newglossaryentry{Test coverage}{
    name={Test coverage},
    description={Misura percentuale del grado con cui il codice sorgente di un programma è eseguito con dei test in esecuzione}
    }

\newglossaryentry{GitHub}{
    name={GitHub},
    description={Servizio di hosting per progetti software}
    }

\newglossaryentry{Continuous integration}{
    name={Continuous integration},
    description={Pratica che si applica in contesti in cui lo sviluppo del software avviene attraverso un sistema di controllo versione. Consiste nell'allineamento frequente dagli ambienti di lavoro degli sviluppatori verso l'ambiente condiviso}
    }

\newglossaryentry{web3j}{
    name={web3j},
    description={Libreria Java per lavorare con smart contracts e per l'integrazione dei nodi sulla rete Ethereum}
    }

\newglossaryentry{web3js}{
    name={web3js},
    description={Collezione di librerie JavaScript che permettono di interagire con un nodo Ethereum locale o remoto usando HTTP}
    }
    
\newglossaryentry{discord}{
    name={Discord},
    description={Piattaforma di messaggistica istantanea vocale e testuale}
    }

\newglossaryentry{RTB}{
    name={RTB},
    description={Requirements and Technology Baseline}
    }

\newglossaryentry{PB}{
    name={PB},
    description={Product Baseline}
    }

\newglossaryentry{CA}{
    name={CA},
    description={Customer Acceptance}
}