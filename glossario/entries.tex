\newglossaryentry{angular} { name=Angular, description={Framework open-source per lo sviluppo di applicazioni web} }
\newglossaryentry{apirest} { name={API REST}, description={Interfaccia di programmazione delle applicazioni conforme ai vincoli dello stile architetturale REST (REpresentational State Transfer)} }
\newglossaryentry{blockchain} { name=Blockchain, description={Insieme di tecnologie in cui il registro è strutturato come una catena di blocchi contenenti le transazioni e il consenso è distribuito su tutti i nodi della rete. Tutti i nodi possono partecipare al processo di validazione delle transazioni da includere nel registro} }
\newglossaryentry{browser} { name=Browser, description={Programma per navigare in internet che inoltra la richiesta di un documento alla rete e ne consente la visualizzazione} }
\newglossaryentry{Cloud} { name={Cloud}, description={Server a cui si accede tramite internet e il software e i database che si eseguono su quei server} }
\newglossaryentry{Criptovaluta} { name={Criptovaluta}, description={valuta virtuale che costituisce una rappresentazione digitale di valore ed è utilizzata come mezzo di scambio o detenuta a scopo di investimento. Le criptovalute possono essere trasferite, conservate o negoziate elettronicamente} }
\newglossaryentry{CSS} { name={CSS}, description={sigla di cascading style sheets, è un linguaggio usato per gestire il design e la presentazione delle pagine web e lavora in combinazione con HTML} }
\newglossaryentry{Database} { name={Database}, description={raccolta organizzata di dati strutturati per renderli facilmente accessibili, gestibili e aggiornabili} }
\newglossaryentry{E-commerce} { name={E-commerce}, description={transazione o scambio di beni e servizi effettuati mediante l'impiego della tecnologia delle telecomunicazioni e dell'informatica} }
\newglossaryentry{Endpoint} { name={Endpoint}, description={qualsiasi dispositivo che possa connettersi a internet sia fisicamente che in cloud} }
\newglossaryentry{Ethereum} { name={Ethereum}, description={piattaforma decentralizzata del web 3.0 per la creazione e pubblicazioni peer-to-peer di smart contracts creati in un linguaggio di programmazione Turing-completo} }
\newglossaryentry{Framework} { name={Framework}, description={architettura logica di supporto sulla quale un software può essere progettato e realizzato} }
\newglossaryentry{HTML} { name={HTML}, description={sigla di HyperText Markup Language permette di immaginare e formattare pagine collegate fra di loro attraverso link} }
\newglossaryentry{Internet} { name={Internet}, description={rete di comunicazione ad accesso pubblico che connette vari dispositivi o terminali in tutto il mondo} }
\newglossaryentry{JavaScript} { name={JavaScript}, description={linguaggio di programmazione per la realizzazione di pagine web interattive} }
\newglossaryentry{Java Spring} { name={Java Spring}, description={framework open source per lo sviluppo di applicazioni su piattaforma java} }
\newglossaryentry{Metamask} { name={Metamask}, description={portafoglio per custodire criptovalute, il quale permette all'utente di interagire con applicazioni decentralizzate basate su Ethereum.} }
\newglossaryentry{Nodo} { name={Nodo}, description={Qualsiasi dispositivo hardware del sistema in grado di comunicare con gli altri dispositivi che fanno parte della rete} }
\newglossaryentry{Open-source} { name={Open-source}, description={sistema di sviluppo software decentralizzato basato sulla condivisione dei file sorgenti} }
\newglossaryentry{Peer-to-peer} { name={Peer-to-peer}, description={modello di architettura logica di rete informatica in cui i nodi non sono gerarchizzati unicamente sotto forma di client o server fissi ma anche sotto forma di nodi equivalenti o "paritari" (peer), potendo fungere al contempo da client e server perso gli altri nodi terminali della rete} }
\newglossaryentry{Recensione} { name={Recensione}, description={forma di commento o voto a un'attività , un giudizio valutativo e interpretativo espresso in forma discorsiva e argomentativa} }
\newglossaryentry{Review bombing} { name={Review bombing}, description={fenomeno in cui un grande gruppo di persone, o anche una singola persona con molteplici account, lascia recensione negativa per un'attività nel tentativo di boicottarla o dannegiarla.} }
\newglossaryentry{RPC} { name={RPC}, description={Remote Procedure Call è uno strumento centrale per realizzare strutture operative e basate sulle ripartizioni del lavoro in reti e architetture client-server} }
\newglossaryentry{Smart contract} { name={Smart contract}, description={programma seguito sulla blockchain di Ethereum; è una raccolta di codice e dati che risiede a un indirizzo specifico sulla blockchain Ethereum. Sono un tipo di account Ethereum, quindi hanno un saldo e possono essere oggetto di transazioni. Possono definire regole, come un normale contratto, e imporle automaticamente tramite codice.} }
\newglossaryentry{Software} { name={Software}, description={insieme delle componenti immateriali a livello logico/intangibile di un sistema elettronico di elaborazione.} }
\newglossaryentry{UI} { name={UI}, description={User Interface è l'interfaccia grafica} }
\newglossaryentry{Webapp} { name={Webapp}, description={dall'inglese “web application” significa applicazione web e si basa sui codici HTML, JavaScript o CSS. Non hanno bisogno di alcuna installazione, visto che sono caricate da un web server ed eseguite in un browser} }
\newglossaryentry{Webserver} { name={Webserver}, description={applicazione software che in esecuzione su un server è in grado di gestire le richieste di trasferimento di pagine web di un client} }
\newglossaryentry{Web} { name={Web}, description={sottorete di internet che riunisce i siti che permettono un sistema di navigazione ipertestuale e visualizzati per mezzo di browser} }
