\section{Introduzione}


\subsection{Scopo del documento}
Il seguente documento ha l'obbiettivo di definire i requisiti di progetto. \\ Al suo interno sono presenti i requisiti 
identificati dal gruppo: \groupName, per lo svolgimento del lavoro richiesto. \\ I requisiti sono stati individuati dal capitolato \textit{Trustify}
e dalle comunicazioni ricevute direttamente dall'azienda \textit{Synclab}.

\subsection{Scopo del prodotto}
Il prodotto ha lo scopo di garantire la veridicità e la affidabilità 
delle recensioni rilasciate dagli utenti a un servizio. L'interfaccia verrà
sviluppata con \textit{Angular}, per garantire la veridicità e la affidabilità 
invece si utilizzerà il linguaggio \textit{Solidity} su rete \textit{Ethereum}.


\subsection{Glossario}
Onde evitare ambiguità nella terminologia dei documenti viene introdotto un glossario.
Il glossario introduce una spiegazione più approfondita dei vocaboli complessi 
per facilitare la lettura.

\subsection{Riferimenti}
    \subsubsection{Riferimenti normativi}
        \begin{itemize}
            \item Capitolato d'appalto C7: \textbf{Trustify - Authentic and verifiable reviews platform}: \\
            \url{https://www.math.unipd.it/~tullio/IS-1/2022/Progetto/C7.pdf}
        \end{itemize}
    \subsubsection{Riferimenti informativi}
    \begin{itemize}
        \item Capitolato d'appalto C7: \textbf{Trustify - Authentic and verifiable reviews platform}: \\
        \url{https://www.math.unipd.it/~tullio/IS-1/2022/Progetto/C7.pdf}
    \end{itemize}