\section{Descrizione generale}
    \subsection{Obiettivi del prodotto}
    Lo scopo del prodotto è quello di fornire un servizio di pagamento che includa la possibilità di rilasciare una recensione univoca da parte dell'utente pagante. \\
    Per natura dello \textit{Smart Contract}, che verrà utilizzato per la gestione del pagamento, non è possibile modificare o censurare la recensione una volta che questa è stata rilasciata. 

    \subsection{Caratteristiche del prodotto}
    Il prodotto atteso dovrà essere composto da:

        \subsubsection{Backend}
        \begin{itemize}
            \item \textbf{Smart Contract}: gestisce la logica dei pagamenti e delle recensioni
            \item \textbf{API REST}: permette di ottenere le informazioni da parte dell'utente che vuole usufruire del servizio, precisamente recuperare le recensioni a lui associate 
        \end{itemize}

        \subsubsection{Frontend}
        \begin{itemize}
            \item \textbf{Web App}: permette agli utenti di interagire con il servizio, ovvero effettuare pagamenti e rilasciare o leggere le recensioni
        \end{itemize}

        VALUTARE ULTERIORI APPROFONDIMENTI

    \subsubsection{Caratteristiche degli utenti}
    Gli utenti utilizzatori del prodotto possono essere di tre tipologie:
    \begin{itemize}
        \item \textbf{Utente generico}: utente che non possiede un account metamask e ha dunque possibilità di usufruire del servizio solo in lettura
        \item \textbf{Utente non autenticato}: utente che possiede un account metamask ma non è autenticato, dunque può usufruire del servizio solo in lettura o autenticarsi per poter effettuare pagamenti e rilasciare recensioni
        \item \textbf{Utente autenticato}: utente che possiede un account metamask e che è autenticato, dunque può effettuare pagamenti e rilasciare recensioni
    \end{itemize}

