\section{Requisiti}

    I requisiti vengono classificati nel seguente modo:
    \begin{itemize}
        \item \textbf{codice identificativo}: ogni codice identificativo è univoco e definito seguendo lo standard di codifica \textbf{R[Importanza][Tipologia][Codice]}  il significato delle cui voci è:
        \begin{itemize}
            \item \textbf{Importanza}:
            \begin{center}
                \rowcolors[]{1}{pari}{dispari}
                \renewcommand{\arraystretch}{1.8}
                \renewcommand\tabularxcolumn[1]{m{#1}}
                \begin{tabularx}{0.85\textwidth} {
                    >{\hsize=0.1\hsize\linewidth=\hsize}X
                    >{\hsize=1.9\hsize\linewidth=\hsize}X
                }
                    \hline
                    \textbf{1} & Requisito obbligatorio: irrinunciabile per qualcuno degli stakeholder. \\
                    \hline
                    \textbf{2} & Requisito desiderabile: non strettamente necessario ma  a valore aggiunto riconoscibile. \\
                    \hline
                    \textbf{3} &  Requisito opzionale: relativamente utile oppure contrattabile più avanti nel progetto. \\
                    \hline
                \end{tabularx}
            \end{center}

            \item \textbf{Tipologia}:
            \begin{center}
                \rowcolors[]{1}{pari}{dispari}
                \renewcommand{\arraystretch}{1.5}
                \begin{tabular}{m{2em} m{10em}}
                    \hline
                    \textbf{F} & Funzionale \\
                    \hline
                    \textbf{P} & Prestazionale \\
                    \hline
                    \textbf{Q} & Qualitativo \\
                    \hline
                    \textbf{V} &  Vincolo \\
                    \hline
                \end{tabular}
            \end{center}
            \item \textbf{Codice}: identificatore univoco del requisito in forma gerarchica.
        \end{itemize}

        \item \textbf{classificazione}: viene riportata l'importanza del requisito per facilitare la lettura;
        \item \textbf{descrizione};
        \item \textbf{fonte}: origine del requisito.
    \end{itemize}

    \subsection{Requisiti funzionali}

        \rowcolors{2}{pari_alt}{dispari_alt}
        \renewcommand{\arraystretch}{1.8}

        \begin{xltabular}{\textwidth} {
            >{\hsize=0.5\hsize\linewidth=\hsize}X
            >{\hsize=0.8\hsize\linewidth=\hsize}X
            >{\hsize=2.10\hsize\linewidth=\hsize}X
            >{\hsize=0.60\hsize\linewidth=\hsize}X
            }
            \rowcolorhead
            \textbf{\color{white}Codice} &
            \textbf{\color{white}Tipo} &
            \textbf{\color{white}Descrizione} &
            \textbf{\color{white}Fonti} \\
            \hline
            \endfirsthead

            \hline
            \rowcolorhead
            \textbf{\color{white}Codice} &
            \textbf{\color{white}Tipo} &
            \textbf{\color{white}Descrizione} &
            \textbf{\color{white}Fonti} \\
            \hline
            \endhead

            \endfoot

            \endlastfoot

            R1F1 &
            Obbligatorio &
            La \textit{web app} fornisce all'utente non autenticato la possibilità effettuare il login tramite \textit{MetaMask}. &
            \hyperref[UC01]{UC01} \\
            \hline

            R1F1.1 &
            Obbligatorio &
            La \textit{web app} visualizza un errore in caso di autenticazione fallita. &
            \hyperref[UC02]{UC02} \\
            \hline

            R1F1.2 &
            Obbligatorio &
            L'utente non autenticato visualizza un messaggio di errore in caso di assenza di \textit{MetaMask} installato. &
            \hyperref[UC02]{UC02} \\
            \hline

            R1F3 &
            Obbligatorio &
            L'utente autenticato deve poter eseguire un pagamento di una certa somma, tramite \textit{MetaMask} verso un altro indirizzo wallet. &
            \hyperref[UC05]{UC05} \\
            \hline

            R1F3.1 &
            Obbligatorio &
            Il pagamento fallisce a causa di fondi insufficienti. &
            \hyperref[UC06]{UC06} \\
            \hline

            R1F3.2 &
            Obbligatorio &
            Il pagamento fallisce a causa di non validità dell'indirizzo dell'utente ricevente. &
            \hyperref[UC06]{UC06} \\
            \hline

            R1F4 &
            Obbligatorio &
            L'utente autenticato può visualizzare una lista dei pagamenti effettuati, senza una recensione, tra cui scegliere quale recensire. &
            \hyperref[UC07]{UC07} \\
            \hline

            R1F5 &
            Obbligatorio &
            L'utente autenticato ha la possibilità di rilasciare una recensione nel caso abbia acquisti non ancora recensiti. &
            \hyperref[UC07]{UC07} \\
            \hline

            R1F5.1 &
            Obbligatorio &
            Per ogni transazione effettuata può esistere al massimo una recensione ad esso collegata. &
            \hyperref[UC07]{UC07} \\
            \hline

            R1F5.2 &
            Obbligatorio &
            L'utente deve scegliere la transazione, effettuata, da associare alla recensione.&
            \hyperref[UC07]{UC07} \\
            \hline

            R1F5.3 &
            Obbligatorio &
            La recensione deve avere un titolo inserito dall'utente.&
            \hyperref[UC07.1]{UC07.1} \\
            \hline

            R1F5.4 &
            Obbligatorio &
            La recensione deve avere un voto inserito dall'utente.&
            \hyperref[UC07.2]{UC07.2} \\
            \hline

            R1F5.5 &
            Obbligatorio &
            La recensione deve avere un testo inserito dall'utente.&
            \hyperref[UC07.3]{UC07.3} \\
            \hline
        
            R1F5.6 &
            Obbligatorio &
            La \textit{web app} visualizza un errore se l'utente non rispetta la validità del formato di una recensione.&
            \hyperref[UC08]{UC08} \\
            \hline

            R1F5.6.1 &
            Obbligatorio &
            La recensione deve avere un titolo, non superiore ai 50 caratteri.&
            \hyperref[UC08]{UC08} \\
            \hline

            R1F5.6.2 &
            Obbligatorio &
            La recensione deve esprimere un voto, compreso tra 1(minimo) e 5 (massimo).&
            \hyperref[UC08]{UC08} \\
            \hline

            R1F5.6.3 &
            Obbligatorio &
            La recensione deve avere un testo, di al massimo 500 caratteri.&
            \hyperref[UC08]{UC08} \\
            \hline

            R1F6 &
            Obbligatorio &
            L'utente generico può ricercare recensioni.  &
            \hyperref[UC09]{UC09} \\
            \hline

            R1F6.1 &
            Obbligatorio &
            L'utente generico può ricercare recensioni per indirizzo wallet. &
            \hyperref[UC09.1]{UC09.1} \\
            \hline

            R1F6.1.1 &
            Obbligatorio &
            L'utente generico può ricercare recensioni per indirizzo wallet appartenente all'autore. &
            \hyperref[UC09.2]{UC09.2} \\
            \hline

            R1F6.1.2 &
            Obbligatorio &
            L'utente generico può ricercare recensioni per indirizzo wallet appartenente al destinatario. &
            \hyperref[UC09.3]{UC09.3} \\
            \hline

            R3F6.2 &
            Opzionale &
            L'utente generico può ricercare recensioni per voto. &
            \hyperref[UC09.4]{UC09.4} \\
            \hline

            R3F6.3 &
            Opzionale &
            L'utente generico può ricercare recensioni per titolo. &
            \hyperref[UC09.5]{UC09.5} \\
            \hline

            R3F6.4 &
            Opzionale &
            L'utente generico può ricercare recensioni per data. &
            \hyperref[UC09.6]{UC09.6} \\
            \hline

            R1F7 &
            Obbligatorio &
            La \textit{web app} fornisce all'utente autenticato la possibilità di ricercare le recensioni rilasciate.&
            \hyperref[UC10]{UC10} \\
            \hline

            R1F8 &
            Obbligatorio &
            La \textit{web app} fornisce all'utente autenticato la possibilità di ricercare le recensioni ricevute.&
            \hyperref[UC11]{UC11} \\
            \hline

            R1F9 &
            Obbligatorio &
            La \textit{web app} notifica all'utente un errore nella ricerca.&
            \hyperref[UC12]{UC12} \\
            \hline

            R1F9.1 &
            Obbligatorio &
            La \textit{web app} notifica all'utente un errore nella ricerca per indirizzo wallet.&
            \hyperref[UC12]{UC12} \\
            \hline

            R3F9.2 &
            Opzionale &
            La \textit{web app} notifica all'utente un errore nella ricerca per voto.&
            \hyperref[UC12]{UC12} \\
            \hline

            R3F9.3 &
            Opzionale &
            La \textit{web app} notifica all'utente un errore nella ricerca per titolo.&
            \hyperref[UC12]{UC12} \\
            \hline

            R3F9.4 &
            Opzionale &
            La \textit{web app} notifica all'utente un errore nella ricerca per data.&
            \hyperref[UC12]{UC12} \\
            \hline

            R1F10 &
            Obbligatorio &
            La \textit{web app} fornisce all'utente generico la possibilità di visualizzare una lista di recensioni.&
            \hyperref[UC13]{UC13} \\
            \hline

            R1F11 &
            Obbligatorio &
            La \textit{web app} fornisce all'utente generico la possibilità di visualizzare una singola recensione.&
            \hyperref[UC13.1]{UC13.1} \\
            \hline

            R1F11.1 &
            Obbligatorio &
            La \textit{web app} fornisce all'utente generico la possibilità di visualizzare l'autore di una singola recensione.&
            \hyperref[UC13.1.1]{UC13.1.1} \\
            \hline

            R1F11.2 &
            Obbligatorio &
            La \textit{web app} fornisce all'utente generico la possibilità di visualizzare il destinatario di una singola recensione.&
            \hyperref[UC13.1.2]{UC13.1.2} \\
            \hline

            R1F11.3 &
            Obbligatorio &
            La \textit{web app} fornisce all'utente generico la possibilità di visualizzare il titolo di una singola recensione.&
            \hyperref[UC13.1.3]{UC13.1.3} \\
            \hline      
            
            R1F11.4 &
            Obbligatorio &
            La \textit{web app} fornisce all'utente generico la possibilità di visualizzare la data di una singola recensione.&
            \hyperref[UC13.1.4]{UC13.1.4} \\
            \hline

            R1F11.5 &
            Obbligatorio &
            La \textit{web app} fornisce all'utente generico la possibilità di visualizzare il voto di una singola recensione.&
            \hyperref[UC13.1.5]{UC13.1.5} \\
            \hline

            R1F11.6 &
            Obbligatorio &
            La \textit{web app} fornisce all'utente generico la possibilità di visualizzare il testo di una singola recensione.&
            \hyperref[UC13.1.6]{UC13.1.6} \\
            \hline

            R1F12 &
            Obbligatorio &
            La \textit{web app} fornisce all'utente autenticato la possibilità di visualizzare le recensioni rilasciate.&
            \hyperref[UC14]{UC14} \\
            \hline
            
            R1F13 &
            Obbligatorio &
            La \textit{web app} fornisce all'utente autenticato la possibilità di visualizzare le recensioni ricevute.&
            \hyperref[UC15]{UC15} \\
            \hline

            R1F14 &
            Obbligatorio &
            La \textit{web app} notifica all'utente un errore nella visualizzazione delle recensioni.&
            \hyperref[UC16]{UC16} \\
            \hline

            R1F15 &
            Obbligatorio &
            Il servizio di \textit{API REST} fornisce all'utente, utilizzatore dell'API, la possibilità di ottenere le recensioni ad esso riferite.&
            \hyperref[UC17]{UC17} \\
            \hline

            R1F16 &
            Obbligatorio &
            Il servizio di \textit{API REST} notifica all'utente un errore, nel caso non esistano recensioni a lui collegate.&
            \hyperref[UC18]{UC18} \\
            \hline

            R3F17 &
            Opzionale &
            La \textit{web app} fornisce all'utente autenticato la possibilità di visualizzare la lista di tutti i pagamenti effettuati.&
            \hyperref[UC19]{UC19} \\
            \hline

            R3F18 &    
            Opzionale &
            La \textit{web app} fornisce all'utente autenticato la possibilità di visualizzare una singola transazione, selezionata dalla lista.&
            \hyperref[UC19.1]{UC19.1} \\
            \hline

            R3F18.1 &   
            Opzionale &
            La \textit{web app} fornisce all'utente autenticato la possibilità di visualizzare l'ID di una singola transazione.&
            \hyperref[UC19.1.1]{UC19.1.1} \\
            \hline

            R3F18.2 &   
            Opzionale &
            La \textit{web app} fornisce all'utente autenticato la possibilità di visualizzare la data di una singola transazione.&
            \hyperref[UC19.1.2]{UC19.1.2} \\
            \hline

            R3F18.3 &   
            Opzionale &
            La \textit{web app} fornisce all'utente autenticato la possibilità di visualizzare l'importo di una singola transazione.&
            \hyperref[UC19.1.3]{UC19.1.3} \\
            \hline

            R3F18.4 &   
            Opzionale &
            La \textit{web app} fornisce all'utente autenticato la possibilità di visualizzare l'utente pagante di una singola transazione.&
            \hyperref[UC19.1.4]{UC19.1.4} \\
            \hline

            R3F18.5 &   
            Opzionale &
            La \textit{web app} fornisce all'utente autenticato la possibilità di visualizzare l'utente ricevente di una singola transazione.&
            \hyperref[UC19.1.5]{UC19.1.5} \\
            \hline

            R3F19 &   
            Opzionale &
            La \textit{web app} notifica all'utente un errore nella visualizzazione dei pagamenti.&
            \hyperref[UC20]{UC20} \\
            \hline

            R3F20 &
            Opzionale &
            La \textit{web app} fornisce all'utente generico la possibilità di visualizzare una lista di recensioni, ordinate.&
            \hyperref[UC21]{UC21} \\
            \hline

            R3F20.1 &
            Opzionale &
            La \textit{web app} fornisce all'utente generico la possibilità di visualizzare una lista di recensioni, ordinate dal meno recente.&
            \hyperref[UC22]{UC22} \\
            \hline

            R3F20.2 &
            Opzionale &
            La \textit{web app} fornisce all'utente generico la possibilità di visualizzare una lista di recensioni, ordinate dal più recente.&
            \hyperref[UC23]{UC23} \\
            \hline

            R3F21 &   
            Opzionale &
            La \textit{web app} fornisce all'utente autenticato la possibilità di visualizzare la lista di tutti i pagamenti effettuati in ordine in base alla data del pagamento.&
            \hyperref[UC24]{UC24} \\
            \hline

            R3F21.1 &   
            Opzionale &
            La \textit{web app} fornisce all'utente autenticato la possibilità di visualizzare la lista di tutti i pagamenti effettuati, ordinandoli dal meno recente.&
            \hyperref[UC25]{UC25} \\
            \hline

            R3F21.2 &   
            Opzionale &
            La \textit{web app} fornisce all'utente autenticato la possibilità di visualizzare la lista di tutti i pagamenti effettuati, ordinandoli dal più recente.&
            \hyperref[UC26]{UC26} \\
            \hline

            R3F22 &   
            Opzionale &
            La \textit{web app} fornisce all'utente autenticato la possibilità di visualizzare la lista di tutti i pagamenti effettuati in ordine in base all'importo.&
            \hyperref[UC27]{UC27} \\
            \hline

            R3F22.1 &   
            Opzionale &
            La \textit{web app} fornisce all'utente autenticato la possibilità di visualizzare la lista di tutti i pagamenti effettuati, ordinandoli per importo più economico.&
            \hyperref[UC28]{UC28} \\
            \hline

            R3F22.2 &   
            Opzionale &
            La \textit{web app} fornisce all'utente autenticato la possibilità di visualizzare la lista di tutti i pagamenti effettuati, ordinandoli per importo meno economico.&
            \hyperref[UC29]{UC29} \\
            \hline

            \rowcolor{white}
            \caption{Requisiti funzionali}
        \end{xltabular}

    \subsection{Requisiti di qualità}

        \rowcolors{2}{pari_alt}{dispari_alt}
        \renewcommand{\arraystretch}{1.8}
        \begin{xltabular}{\textwidth} {
            >{\hsize=0.5\hsize\linewidth=\hsize}X
            >{\hsize=0.8\hsize\linewidth=\hsize}X
            >{\hsize=2.10\hsize\linewidth=\hsize}X
            >{\hsize=0.60\hsize\linewidth=\hsize}X
            }
            \rowcolorhead
            \textbf{\color{white}Codice} &
            \textbf{\color{white}Tipo} &
            \textbf{\color{white}Descrizione} &
            \textbf{\color{white}Fonti} \\
            \hline
            \endfirsthead

            \hline
            \rowcolorhead
            \textbf{\color{white}Codice} &
            \textbf{\color{white}Tipo} &
            \textbf{\color{white}Descrizione} &
            \textbf{\color{white}Fonti} \\
            \hline
            \endhead

            \endfoot
            \endlastfoot

            R1Q1 &
            Obbligatorio &
            Documentazione relativa alle scelte implementative e alle rispettive motivazioni; problemi aperti e soluzioni proposte. &
            Capitolato \\
            \hline

            R1Q2 &
            Obbligatorio &
            \textit{Copertura di test}\glo  $\geq$ 80\% correlata di report. &
            Capitolato \\
            \hline

            R1Q3 &
            Obbligatorio &
            Documentazione su \textit{endpoint}\glo  del server \textit{API REST}. &
            Capitolato \\
            \hline

            R1Q4& Obbligatorio &
            Documentazione su problemi aperti e eventuali soluzioni proposte da esplorare. &
            Capitolato \\
            \hline

            R2Q5 & Desiderabile &
            Tutto il codice prodotto disponibile sulla piattaforma \textit{GitHub}\glo . &
            Scelta interna \\
            \hline

            R2Q6 & Desiderabile &
            Sistema di \textit{Continuous Integration}\glo  per la gestione degli artefatti. &
            Scelta interna \\
            \hline

            R1Q6 &
            Obbligatorio &
            Produzione di un manuale utente. &
            Capitolato \\
            \hline

            R1Q7 &
            Obbligatorio &
            Produzione di un manuale tecnico per manutenzione ed estensione. &
            Capitolato \\
            \hline

            R1Q8 &
            Obbligatorio &
            Il prodotto deve essere sviluppato seguendo quanto stabilito nelle \textit{Norme di Progetto v1.0.0}. &
            Scelta interna \\
            \hline

            \rowcolor{white}
            \caption{Requisiti di qualità}
        \end{xltabular}

    \subsection{Requisiti di vincolo}

        \rowcolors{2}{pari_alt}{dispari_alt}
        \renewcommand{\arraystretch}{1.8}
        \begin{xltabular}{\textwidth} {
            >{\hsize=0.5\hsize\linewidth=\hsize}X
            >{\hsize=0.8\hsize\linewidth=\hsize}X
            >{\hsize=2.10\hsize\linewidth=\hsize}X
            >{\hsize=0.60\hsize\linewidth=\hsize}X
        }
            \rowcolorhead
            \textbf{\color{white}Codice} &
            \textbf{\color{white}Tipo} &
            \textbf{\color{white}Descrizione} &
            \textbf{\color{white}Fonti} \\
            \hline
            \endfirsthead

            \hline
            \rowcolorhead
            \textbf{\color{white}Codice} &
            \textbf{\color{white}Tipo} &
            \textbf{\color{white}Descrizione} &
            \textbf{\color{white}Fonti} \\
            \hline
            \endhead

            \endfoot
            \endlastfoot

            R1V1 &
            Obbligatorio &
            Pagamento e recensione devono essere gestiti tramite \textit{smart contract}. &
            Capitolato \\
            \hline

            R2V2 &
            Desiderabile &
            Utilizzo di \textit{Infura} come fornitore terzo per \textit{RPC} a nodo. &
            Capitolato \\
            \hline

            R2V3 &
            Desiderabile &
            Utilizzo di \textit{Solidity} per lo sviluppo di \textit{smart contract}. &
            Capitolato \\
            \hline

            R2V4 &
            Desiderabile &
            Utilizzo di \textit{Java Spring}\glo  per lo sviluppo di \textit{API REST}. &
            Capitolato \\
            \hline

            R2V5 &
            Desiderabile &
            Utilizzo di \textit{Angular} per lo sviluppo della \textit{web app}. &
            Capitolato \\
            \hline

            R2V6 &
            Desiderabile &
            La \textit{web app} utilizza la libreria \textit{web3js}\glo  per l'interazione con lo \textit{smart contract}. &
            Capitolato \\
            \hline

            R2V7 &
            Desiderabile &
            Il \textit{server API REST} utilizza la libreria \textit{web3j}\glo  per l'interazione con lo \textit{smart contract}. &
            Capitolato \\
            \hline

            R1V8 &
            Obbligatorio &
            Utilizzo di \textit{MetaMask} come \textit{wallet} per l'interazione con lo \textit{smart contract}. &
            Capitolato \\
            \hline

            R1V9 &
            Obbligatorio &
            Assicurare compatibilità \textit{webapp} con \textit{Chrome $\ge$v100}. &
            R1V8 \\
            \hline

            R1V10 &
            Obbligatorio &
            Assicurare compatibilità \textit{webapp} con 
             \textit{Firefox $\ge$v91}. &
            R1V8 \\
            \hline

            R1V11 &
            Obbligatorio &
            Utilizzo di un fornitore terzo per \textit{RPC} a nodo. &
            Capitolato \\
            \hline

            R1V12 &
            Obbligatorio &
            I contenuti delle recensioni sono salvati su una \textit{blockchain} pubblica Ethereum compatibile. &
            Capitolato \\
            \hline

            R1V13 &
            Obbligatorio &
            Sviluppo di un servizio \textit{API} per il reperimento delle recensioni. &
            Capitolato \\
            \hline

            R1V14 &
            Obbligatorio &
            Sviluppo di una \textit{web app} per l'interazione con lo \textit{smart contract}. &
            Capitolato \\
            \hline

            \rowcolor{white}
            \caption{Requisiti di vincolo}
        \end{xltabular}

    \subsection{Requisiti Prestazionali}

    Da parte del proponente non sono state fornite specifiche esigenze di prestazioni.

    \subsection{Tracciamento fonte-requisiti}

        \rowcolors{2}{pari_alt}{dispari_alt}
        \renewcommand{\arraystretch}{1.8}
        \begin{xltabular}{0.9\textwidth} {
            >{\hsize=0.4\hsize\linewidth=\hsize}X
            >{\hsize=1.6\hsize\linewidth=\hsize}X
            }
            \rowcolorhead
            \textbf{\color{white}Fonte} &
            \textbf{\color{white}Requisiti} \\
            \hline
            \endfirsthead

            \hline
            \rowcolorhead
            \textbf{\color{white}Fonte} &
            \textbf{\color{white}Requisiti} \\
            \hline
            \endhead

            \endfoot
            \endlastfoot

            Capitolato &
            R1Q1, R1Q2, R1Q3, R1Q4, R1Q6, R1Q7, R1V1, R2V2, R2V3, R2V4, R2V5, R2V6, R2V7, R1V8, R1V10, R1V11, R1V12, R1V13 \\
            \hline

            Scelta Interna &
            R2Q5, R216, R1Q8 \\
            \hline

            R1V8 &
            R1V9 \\
            \hline

            \hyperref[UC01]{UC01} &
            R1F1 \\
            \hline

            \hyperref[UC02]{UC02} &
            R1F1.1, R1F1.2 \\
            \hline

            \hyperref[UC05]{UC05} &
            R1F3 \\
            \hline

            \hyperref[UC06]{UC06} &
            R1F3.1, R1F3.2 \\
            \hline

            \hyperref[UC07]{UC07} &
            R1F4, R1F5, R1F5.1, R1F5.2 \\
            \hline

            \hyperref[UC07.1]{UC07.1} &
            R1F5.3 \\
            \hline

            \hyperref[UC07.2]{UC07.2} &
            R1F5.4 \\
            \hline

            \hyperref[UC07.3]{UC07.3} &
            R1F5.5 \\
            \hline

            \hyperref[UC08]{UC08} &
            R1F5.6, R1F5.6.1, R1F5.6.2, R1F5.6.3 \\
            \hline

            \hyperref[UC09]{UC09} &
            R1F6    \\
            \hline

            \hyperref[UC09.1]{UC09.1} &
            R1F6.1 \\
            \hline

            \hyperref[UC09.2]{UC09.2} &
            R1F6.1.1 \\
            \hline

            \hyperref[UC09.3]{UC09.3} &
            R1F6.1.2 \\
            \hline

            \hyperref[UC09.4]{UC09.4} &
            R1F6.2 \\
            \hline

            \hyperref[UC09.5]{UC09.5} &
            R1F6.3 \\
            \hline

            \hyperref[UC09.6]{UC09.6} &
            R1F6.4 \\
            \hline

            \hyperref[UC10]{UC10} &
            R1F7 \\
            \hline

            \hyperref[UC11]{UC11} &
            R1F8 \\
            \hline

            \hyperref[UC12]{UC12} &
            R1F9, R1F9.1, R1F9.2, R1F9.3, R1F9.4 \\
            \hline

            \hyperref[UC13]{UC13} &
            R1F10 \\
            \hline
            
            \hyperref[UC13.2]{UC13.2} &
            R3F10.1 \\
            \hline

            \hyperref[UC13.3]{UC13.3} &
            R3F10.2 \\
            \hline

            \hyperref[UC13.1]{UC13.1} &
            R1F11 \\
            \hline

            \hyperref[UC13.1.1]{UC13.1.1} &
            R1F11.1 \\
            \hline

            \hyperref[UC13.1.2]{UC13.1.2} &
            R1F11.2 \\
            \hline

            \hyperref[UC13.1.3]{UC13.1.3} &
            R1F11.3 \\
            \hline

            \hyperref[UC13.1.4]{UC13.1.4} &
            R1F11.4 \\
            \hline

            \hyperref[UC13.1.5]{UC13.1.5} &
            R1F11.5 \\
            \hline

            \hyperref[UC13.1.6]{UC13.1.6} &
            R1F11.6 \\
            \hline

            \hyperref[UC14]{UC14} &
            R1F12 \\
            \hline

            \hyperref[UC15]{UC15} &
            R1F13 \\
            \hline

            \hyperref[UC16]{UC16} &
            R1F14 \\
            \hline

            \hyperref[UC17]{UC17} &
            R1F15 \\
            \hline

            \hyperref[UC18]{UC18} &
            R1F16 \\
            \hline

            \hyperref[UC19]{UC19} &
            R3F17 \\
            \hline

            \hyperref[UC19.1]{UC19.1} &
            R3F17.1 \\
            \hline

            \hyperref[UC19.2]{UC19.2} &
            R3F17.2 \\
            \hline

            \hyperref[UC19.3]{UC19.3} &
            R3F17.3 \\
            \hline

            \hyperref[UC19.4]{UC19.4} &
            R3F17.4 \\
            \hline

            \hyperref[UC19.5]{UC19.5} &
            R3F18 \\
            \hline

            \hyperref[UC19.5.1]{UC19.5.1} &
            R3F18.1 \\
            \hline

            \hyperref[UC19.5.2]{UC19.5.2} &
            R3F18.2 \\
            \hline

            \hyperref[UC19.5.3]{UC19.5.3} &
            R3F18.3 \\
            \hline

            \hyperref[UC19.5.4]{UC19.5.4} &
            R3F18.4 \\
            \hline

            \hyperref[UC19.5.5]{UC19.5.5} &
            R3F18.5 \\
            \hline

            \hyperref[UC20]{UC20} &
            R3F19 \\
            \hline

            \hyperref[UC21]{UC21} &
            R3F20\\
            \hline

            \hyperref[UC22]{UC22} &
            R3F20.1 \\
            \hline

            \hyperref[UC23]{UC23} &
            R3F20.2 \\
            \hline

            \hyperref[UC24]{UC24} &
            R3F21 \\
            \hline

            \hyperref[UC25]{UC25} &
            R3F21.1 \\
            \hline

            \hyperref[UC26]{UC26} &
            R3F21.2 \\
            \hline

            \hyperref[UC27]{UC27} &
            R3F22 \\
            \hline

            \hyperref[UC28]{UC28} &
            R3F22.1 \\
            \hline

            \hyperref[UC29]{UC29} &
            R3F22.2 \\
            \hline

            \rowcolor{white}
            \caption{Tracciamento fonte-requisiti}
        \end{xltabular}

    \subsection{Riepilogo}
        \rowcolors{2}{pari_alt}{dispari_alt}
        \renewcommand{\arraystretch}{1.8}

        \begin{table}[H]
            \centering
            \begin{tabularx}{0.9\textwidth} {
                >{\centering\arraybackslash}X
                >{\centering\arraybackslash}X
                >{\centering\arraybackslash}X
                >{\centering\arraybackslash}X
                >{\centering\arraybackslash}X
                }
                \rowcolorhead
                \textbf{\color{white}Tipologia} &
                \textbf{\color{white}Obbligatorio} &
                \textbf{\color{white}Desiderabile} &
                \textbf{\color{white}Opzionale} &
                \textbf{\color{white}Totale} \\
                \hline

                Funzionale & 41 & - & 22 & 63 \\
                \hline

                Di qualità & 7 & 2 & - & 9 \\
                \hline

                Di vincolo & 7 & 6 & - & 13 \\
                \hline

                Prestazionale & - & - & - & - \\
                \hline
            \end{tabularx}
            \caption{Riepilogo}
        \end{table}