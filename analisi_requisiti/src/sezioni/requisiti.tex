\section{Requisiti}

    I requisiti vengono classificati nel seguente modo:
    \begin{itemize}
        \item \textbf{codice identificativo}: ogni codice identificativo è univoco e definito seguendo lo standard di codifica \textbf{R[Importanza][Tipologia][Codice]}  il significato delle cui voci è:
        \begin{itemize}
            \item \textbf{Importanza}:
            \begin{center}
                \rowcolors[]{1}{pari}{dispari}
                \renewcommand{\arraystretch}{1.8}
                \renewcommand\tabularxcolumn[1]{m{#1}}
                \begin{tabularx}{0.85\textwidth} {
                    >{\hsize=0.1\hsize\linewidth=\hsize}X
                    >{\hsize=1.9\hsize\linewidth=\hsize}X
                }
                    \hline
                    \textbf{1} & Requisito obbligatorio: irrinunciabile per qualcuno degli stakeholder. \\
                    \hline
                    \textbf{2} & Requisito desiderabile: non strettamente necessario ma  a valore aggiunto riconoscibile. \\
                    \hline
                    \textbf{3} &  Requisito opzionale: relativamente utile oppure contrattabile più avanti nel progetto. \\
                    \hline
                \end{tabularx}
            \end{center}

            \item \textbf{Tipologia}:
            \begin{center}
                \rowcolors[]{1}{pari}{dispari}
                \renewcommand{\arraystretch}{1.5}
                \begin{tabular}{m{2em} m{10em}}
                    \hline
                    \textbf{F} & Funzionale \\
                    \hline
                    \textbf{P} & Prestazionale \\
                    \hline
                    \textbf{Q} & Qualitativo \\
                    \hline
                    \textbf{V} &  Vincolo \\
                    \hline
                \end{tabular}
            \end{center}
            \item \textbf{Codice}: identificatore univoco del requisito in forma gerarchica.
        \end{itemize}

        \item \textbf{classificazione}: viene riportata l'importanza del requisito per facilitare la lettura;
        \item \textbf{descrizione};
        \item \textbf{fonte}: origine del requisito.
    \end{itemize}

    \subsection{Requisiti funzionali}

        \rowcolors{2}{pari_alt}{dispari_alt}
        \renewcommand{\arraystretch}{1.8}

        \begin{xltabular}{\textwidth} {
            >{\hsize=0.5\hsize\linewidth=\hsize}X
            >{\hsize=0.8\hsize\linewidth=\hsize}X
            >{\hsize=2.10\hsize\linewidth=\hsize}X
            >{\hsize=0.60\hsize\linewidth=\hsize}X
            }
            \rowcolorhead
            \textbf{\color{white}Codice} &
            \textbf{\color{white}Tipo} &
            \textbf{\color{white}Descrizione} &
            \textbf{\color{white}Fonti} \\
            \hline
            \endfirsthead

            \hline
            \rowcolorhead
            \textbf{\color{white}Codice} &
            \textbf{\color{white}Tipo} &
            \textbf{\color{white}Descrizione} &
            \textbf{\color{white}Fonti} \\
            \hline
            \endhead

            \endfoot

            \endlastfoot

            R1F2 &
            Obbligatorio &
            L'utente può effettuare il login. &
            \hyperref[UC01]{UC01} \\
            \hline

            R1F2.1 &
            Obbligatorio &
            L'utente deve effettuare il login tramite \textit{MetaMask}. &
            \hyperref[UC01]{UC01} \\
            \hline

            R1F2.2 &
            Obbligatorio &
            La \textit{web app} visualizza un errore in caso di autenticazione fallita. &
            \hyperref[UC10]{UC10} \\
            \hline

            R1F3 &
            Obbligatorio &
            L'utente generico può ricercare recensioni di un determinato utente tramite il suo indirizzo del wallet &
            \hyperref[UC03]{UC03} \\
            \hline

            R1F3.1 &
            Obbligatorio &
            La \textit{web app} visualizza un errore di ricerca di recensioni verso un indirizzo inesistente. &
            \hyperref[UC11]{UC11} \\
            \hline

            R1F3.2 &
            Obbligatorio &
            La \textit{web app} visualizza un avviso se l'indirizzo cercato non presenta recensioni collegate. &
            \hyperref[UC12]{UC12} \\
            \hline

            R1F4 &
            Obbligatorio &
            L'utente autenticato deve poter eseguire un pagamento tramite \textit{MetaMask}. &
            \hyperref[UC09]{UC09} \\
            \hline

            R1F4.1 &
            Obbligatorio &
            Se il pagamento fallisce, la \textit{web app} visualizza un errore con la causa. &
            \hyperref[UC14]{UC14} \\
            \hline

            R1F4.2 &
            Obbligatorio &
            L'utente autenticato deve poter visualizzare un errore nel caso in cui la blockchain selezionata non sia corretta. &
            \hyperref[UC04]{UC04} \\
            \hline

            R1F5 &
            Obbligatorio &
            L'utente autenticato deve poter rilasciare una recensione. &
            \hyperref[UC08]{UC08} \\
            \hline

            R1F5.1 &
            Obbligatorio &
            Per ogni recensione deve esistere un'effettiva transazione economica &
            \hyperref[UC08]{UC08} \\
            \hline

            R1F5.2 &
            Obbligatorio &
            L'utente deve scegliere il pagamento da associare alla recensione.&
            \hyperref[UC08]{UC08} \\
            \hline

            R1F5.2.1 &
            Obbligatorio &
            La recensione potrà essere pubblicata solo se il pagamento associato è stato confermato. &
            \hyperref[UC08]{UC08} \\
            \hline

            R1F5.3 &
            Obbligatorio &
            Se il rilascio della recensione non va a buon fine viene visualizzato un errore con  i dettagli sul motivo del fallimento. &
            \hyperref[UC13]{UC13} \\
            \hline
            
            R1F5.3.1 &
            Obbligatorio &
            La \textit{web app} visualizza un errore se l'utente seleziona un pagamento a cui è collegata già una recensione.&
            \hyperref[UC13]{UC13} \\
            \hline
            
            R1F6 &
            Obbligatorio &
            La \textit{web app} fornisce all'utente autenticato la possibilità di visualizzare le recensioni ricevute.&
            \hyperref[UC06]{UC06} \\
            \hline

            R1F7 &
            Obbligatorio &
            La \textit{web app} fornisce all'utente autenticato la possibilità di visualizzare le recensioni rilasciate.&
            \hyperref[UC05]{UC05} \\
            \hline

            R1F8 &
            Obbligatorio &
            I contenuti delle recensioni sono salvati su una \textit{blockchain} pubblica Ethereum compatibile. &
            Capitolato \\
            \hline

            R1F9 &
            Obbligatorio &
            Sviluppo di un servizio di \textit{API REST} per il recupero delle recensioni da parte di utenti esterni. &
            Capitolato \\
            \hline

            R1F10 &
            Obbligatorio &
            L'utente autenticato può visualizzare una lista di tutti i pagamenti effettuati tra cui scegliere quale recensire. &
            \hyperref[UC07]{UC07} \\
            \hline

            R1F11 &
            Obbligatorio &
            L'utente non autenticato deve visualizzare un messaggio di errore in caso di assenza di \textit{MetaMask} installato. &
            \hyperref[UC02]{UC02} \\
            \hline

            \rowcolor{white}
            \caption{Requisiti funzionali}
        \end{xltabular}

    \subsection{Requisiti di qualità}

        \rowcolors{2}{pari_alt}{dispari_alt}
        \renewcommand{\arraystretch}{1.8}
        \begin{xltabular}{\textwidth} {
            >{\hsize=0.5\hsize\linewidth=\hsize}X
            >{\hsize=0.8\hsize\linewidth=\hsize}X
            >{\hsize=2.10\hsize\linewidth=\hsize}X
            >{\hsize=0.60\hsize\linewidth=\hsize}X
            }
            \rowcolorhead
            \textbf{\color{white}Codice} &
            \textbf{\color{white}Tipo} &
            \textbf{\color{white}Descrizione} &
            \textbf{\color{white}Fonti} \\
            \hline
            \endfirsthead

            \hline
            \rowcolorhead
            \textbf{\color{white}Codice} &
            \textbf{\color{white}Tipo} &
            \textbf{\color{white}Descrizione} &
            \textbf{\color{white}Fonti} \\
            \hline
            \endhead

            \endfoot
            \endlastfoot

            R1Q1 &
            Obbligatorio &
            Documentazione relativa alle scelte implementative e alle rispettive motivazioni; problemi aperti e soluzioni proposte. &
            Capitolato \\
            \hline

            R1Q2 &
            Obbligatorio &
            Copertura di \textit{test}\glo $\geq$ 80\% correlata di report. &
            Capitolato \\
            \hline

            R1Q3 &
            Obbligatorio &
            Documentazione su \textit{endpoint}\glo del server \textit{API REST}. &
            Capitolato \\
            \hline

            R1Q4& Obbligatorio &
            Documentazione su problemi aperti e eventuali soluzioni proposte da esplorare. &
            Capitolato \\
            \hline

            R2Q5 & Desiderabile &
            Tutto il codice prodotto disponibile sulla piattaforma \textit{GitHub}\glo. &
            Scelta interna \\
            \hline

            R2Q6 & Desiderabile &
            Sistema di \textit{Continuous Integration}\glo per la gestione degli artefatti. &
            Scelta interna \\
            \hline

            R1Q6 &
            Obbligatorio &
            Produzione di un manuale utente. &
            Capitolato \\
            \hline

            R1Q7 &
            Obbligatorio &
            Produzione di un manuale tecnico per manutenzione ed estensione. &
            Capitolato \\
            \hline

            R1Q8 &
            Obbligatorio &
            Il prodotto deve essere sviluppato seguendo quanto stabilito nelle Norme di progetto. &
            Scelta interna \\
            \hline

            \rowcolor{white}
            \caption{Requisiti di qualità}
        \end{xltabular}

    \subsection{Requisiti di vincolo}

        \rowcolors{2}{pari_alt}{dispari_alt}
        \renewcommand{\arraystretch}{1.8}
        \begin{xltabular}{\textwidth} {
            >{\hsize=0.5\hsize\linewidth=\hsize}X
            >{\hsize=0.8\hsize\linewidth=\hsize}X
            >{\hsize=2.10\hsize\linewidth=\hsize}X
            >{\hsize=0.60\hsize\linewidth=\hsize}X
        }
            \rowcolorhead
            \textbf{\color{white}Codice} &
            \textbf{\color{white}Tipo} &
            \textbf{\color{white}Descrizione} &
            \textbf{\color{white}Fonti} \\
            \hline
            \endfirsthead

            \hline
            \rowcolorhead
            \textbf{\color{white}Codice} &
            \textbf{\color{white}Tipo} &
            \textbf{\color{white}Descrizione} &
            \textbf{\color{white}Fonti} \\
            \hline
            \endhead

            \endfoot
            \endlastfoot

            R1V1 &
            Obbligatorio &
            Pagamento e recensione devono essere gestiti tramite \textit{smart contract}. &
            Capitolato \\
            \hline

            R2V2 &
            Desiderabile &
            Sviluppo nodo \textit{RPC} tramite \textit{Infura}. &
            Capitolato \\
            \hline

            R2V3 &
            Desiderabile &
            Utilizzo di \textit{Solidity} per lo sviluppo di \textit{smart contract}. &
            Capitolato \\
            \hline

            R2V4 &
            Desiderabile &
            Utilizzo di \textit{Java} e \textit{Spring Boot}\glo per lo sviluppo di \textit{API REST}. &
            Capitolato \\
            \hline

            R2V5 &
            Desiderabile &
            Utilizzo di \textit{Angular} per lo sviluppo della \textit{web app}. &
            Capitolato \\
            \hline

            R2V6 &
            Desiderabile &
            La \textit{web app} utilizza la libreria \textit{web3js}\glo per l'interazione con lo \textit{smart contract}. &
            Capitolato \\
            \hline

            R2V7 &
            Desiderabile &
            Il \textit{server API REST} utilizza la libreria \textit{web3j}\glo per l'interazione con lo \textit{smart contract}. &
            Capitolato \\
            \hline

            R1V8 &
            Obbligatorio &
            Utilizzo di \textit{MetaMask} come \textit{wallet} per l'interazione con lo \textit{smart contract}. &
            Capitolato \\
            \hline

            R1V9 &
            Obbligatorio &
            Utilizzo di un fornitore terzo per \textit{RPC} a nodo. &
            Capitolato \\
            \hline

            \rowcolor{white}
            \caption{Requisiti di vincolo}
        \end{xltabular}

    \subsection{Tracciamento fonte-requisiti}

        \rowcolors{2}{pari_alt}{dispari_alt}
        \renewcommand{\arraystretch}{1.8}
        \begin{xltabular}{0.9\textwidth} {
            >{\hsize=0.4\hsize\linewidth=\hsize}X
            >{\hsize=1.6\hsize\linewidth=\hsize}X
            }
            \rowcolorhead
            \textbf{\color{white}Fonte} &
            \textbf{\color{white}Requisiti} \\
            \hline
            \endfirsthead

            \hline
            \rowcolorhead
            \textbf{\color{white}Fonte} &
            \textbf{\color{white}Requisiti} \\
            \hline
            \endhead

            \endfoot
            \endlastfoot

            Capitolato &
            R1F1, R1F9, R1F9.1, R1F10, R1F11, R1F12, R1F13, R1Q1, R1Q2, R1Q3, R1Q4, R1Q5, R1Q6, R1Q7, R1Q8, R1V1, R2V2, R2V3, R2V4, R2V5, R2V6, R2V7, R1V8, R1V9 \\
            \hline

            \hyperref[UC01]{UC01} &
            R1F2 \\
            \hline

            \hyperref[UC10]{UC10} &
            R1F2.1 \\
            \hline

            \hyperref[UC03]{UC03} &
            R1F3 \\
            \hline

            \hyperref[UC11]{UC11} &
            R1F3.1 \\
            \hline

            \hyperref[UC12]{UC12} &
            R1F3.2 \\
            \hline

            \hyperref[UC09]{UC09} &
            R1F4 \\
            \hline

            \hyperref[UC14]{UC14} &
            R1F4.1 \\
            \hline

            \hyperref[UC08]{UC08} &
            R1F5, R1F5.1, R1F6.1 \\
            \hline

            \hyperref[UC13]{UC13} &
            R1F6.2, R1F5.1.1, R1F6.3 \\
            \hline

            \hyperref[UC06]{UC06} &
            R1F7 \\
            \hline

            \hyperref[UC05]{UC05} &
            R1F8 \\
            \hline

            \rowcolor{white}
            \caption{Tracciamento fonte-requisiti}
        \end{xltabular}

    \subsection{Riepilogo}
        \rowcolors{2}{pari_alt}{dispari_alt}
        \renewcommand{\arraystretch}{1.8}

        \begin{table}[H]
            \centering
            \begin{tabularx}{0.9\textwidth} {
                >{\centering\arraybackslash}X
                >{\centering\arraybackslash}X
                >{\centering\arraybackslash}X
                >{\centering\arraybackslash}X
                >{\centering\arraybackslash}X
                }
                \rowcolorhead
                \textbf{\color{white}Tipologia} &
                \textbf{\color{white}Obbligatorio} &
                \textbf{\color{white}Desiderabile} &
                \textbf{\color{white}Opzionale} &
                \textbf{\color{white}Totale} \\
                \hline

                Funzionale & 22 & - & - & 22 \\
                \hline

                Di qualità & 8 & - & - & 8 \\
                \hline

                Di vincolo & 3 & 6 & - & 9 \\
                \hline

                Prestazionale & - & - & - & - \\
                \hline
            \end{tabularx}
            \caption{Riepilogo}
        \end{table}