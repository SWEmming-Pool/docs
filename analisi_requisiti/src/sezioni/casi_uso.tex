\section{Casi d'uso}

\subsection{Attori}
\subsubsection{Attori primari}

\begin{figure}[htbp]
    \centering
    \includesvg[scale=0.9]{src/img/attoriPrimari.drawio.svg}
    \caption{Gerarchia degli attori primari}
  \end{figure}


\paragraph*{Utente generico:} utente che non possiede un account \textit{MetaMask} e ha dunque possibilità di usufruire del servizio solo in lettura.
\paragraph*{Utente non autenticato:} utente che possiede un account \textit{MetaMask} ma non è autenticato, dunque può usufruire del servizio solo in lettura o autenticarsi per poter effettuare pagamenti e rilasciare recensioni.
\paragraph*{Utente autenticato:} utente che possiede un account \textit{MetaMask} e che è autenticato, dunque può effettuare pagamenti e rilasciare recensioni.
\subsubsection{Attori secondari}
\paragraph*{\textit{MetaMask}:} \textit{wallet}\glo\: di criptovalute usato per interagire con la \textit{blockchain} \textit{Ethereum}. Permette di accedere al proprio \textit{wallet} attraverso applicazione mobile o, come nel nostro caso, estensione per browser.
\paragraph*{\textit{Infura}\glo:} backend \textit{Web3}\glo\: che offre servizi e strumenti per sviluppatori \textit{blockchain}; nel nostro caso sarà ciò che metterà in comunicazione \textit{web app} e \textit{API REST} con nodi\glo\: sulla \textit{blockchain}.

\subsection{Elenco}

\subsubsection{UC01 - Autenticazione}
\label{UC01}

\begin{figure}[H]
    \centering
    \includesvg[width=1\textwidth]{src/img/UC01.drawio.svg}
    \caption{UC01}
 \end{figure}

\begin{center}
\rowcolors[]{1}{pari}{dispari}
\renewcommand{\arraystretch}{1.5}
\begin{tabular}{ | m{10em} | m{20em} | }
    \hline
    \textbf{Attore primario} & Utente non autenticato \\
    \hline
    \textbf{Attore secondario} & \textit{MetaMask} \\
    \hline
    \textbf{Precondizioni} & L'utente non è autenticato. \\
    \hline
    \textbf{Postcondizioni} & L'utente è autenticato. \\
    \hline
    \textbf{Scenario principale} & L'utente accede a \textit{MetaMask}. \\
    \hline
    \textbf{Estensioni} & Se l'accesso a \textit{MetaMask} non va a buon fine, si verifica \hyperref[UC10]{UC10}. \\
    \hline
   \end{tabular}
\end{center}

\subsubsection{UC02 - Logout}
\label{UC02}

\begin{figure}[H]
    \centering
    \includesvg[scale=0.9]{src/img/UC02.drawio.svg}
    \caption{UC02}
 \end{figure}

\begin{center}
\rowcolors[]{1}{pari}{dispari}
\renewcommand{\arraystretch}{1.5}
\begin{tabular}{ | m{10em} | m{20em} | }
    \hline
    \textbf{Attore primario} & Utente autenticato \\
    \hline
    \textbf{Precondizioni} & L'utente è autenticato. \\
    \hline
    \textbf{Postcondizioni} & L'utente non è più autenticato. \\
    \hline
    \textbf{Scenario principale} & L'utente effettua il logout. \\
    \hline
   \end{tabular}
\end{center}

\subsubsection{UC03 - Ricerca Recensioni}
\label{UC03}

\begin{figure}[H]
    \centering
    \includesvg[scale=0.9]{src/img/UC03.drawio.svg}
    \caption{UC03}
 \end{figure}

\begin{center}
\rowcolors[]{1}{pari}{dispari}
\renewcommand{\arraystretch}{1.5}
\begin{tabular}{ | m{10em} | m{20em} | }
    \hline
    \textbf{Attore primario} & Utente generico \\
    \hline
    \textbf{Attore secondario} & Infura \\
    \hline
    \textbf{Precondizioni} & L'utente conosce l'indirizzo di cui vuole visualizzare le recensioni. \\
    \hline
    \textbf{Postcondizioni} & L'utente visualizza le recensioni associate a quell'indirizzo. \\
    \hline
    \textbf{Scenario principale} & L'utente richiede la lista delle recensioni di un determinato indirizzo. \\
    \hline
    \textbf{Estensioni} & Se l'utente inserisce un indirizzo inesistente, si verifica \hyperref[UC11]{UC11}. \\
    \hline
   \end{tabular}
\end{center}

\subsubsection{UC04 - Visualizzazione Recensioni}
\label{UC04}

\begin{figure}[H]
    \centering
    \includesvg[width=1\textwidth]{src/img/UC04.drawio.svg}
    \caption{UC04}
 \end{figure}
\begin{center}
\rowcolors[]{1}{pari}{dispari}
\renewcommand{\arraystretch}{1.5}
\begin{tabular}{ | m{10em} | m{20em} | }
    \hline
    \textbf{Attore primario} & Utente generico \\
    \hline
    \textbf{Precondizioni} & L'utente ha effettuato una ricerca per indirizzo. \\
    \hline
    \textbf{Postcondizioni} & L'utente visualizza le recensioni associate a quell'indirizzo. \\
    \hline
    \textbf{Scenario principale} & Viene visualizzata una lista di recensioni corrispondenti. \\
    \hline
    \textbf{Estensioni} & Se l'utente inserisce un indirizzo inesistente, si verifica \hyperref[UC12]{UC12}. \\
    \hline
   \end{tabular}
\end{center}

\subsubsection{UC05 - Visualizzazione Recensioni Rilasciate}
\label{UC05}

\begin{figure}[H]
    \centering
    \includesvg[width=1\textwidth]{src/img/UC05.drawio.svg}
    \caption{UC05}
 \end{figure}

\begin{center}
\rowcolors[]{1}{pari}{dispari}
\renewcommand{\arraystretch}{1.5}
\begin{tabular}{ | m{10em} | m{20em} | }
    \hline
    \textbf{Attore primario} & Utente autenticato \\
    \hline
    \textbf{Attore secondario} & \textit{MetaMask} \\
    \hline
    \textbf{Precondizioni} & Nessuna. \\
    \hline
    \textbf{Postcondizioni} & L'utente visualizza le recensioni che ha rilasciato. \\
    \hline
    \textbf{Scenario principale} & Viene mostrata la lista di recensioni rilasciate dall'utente. \\
    \hline
   \end{tabular}
\end{center}

\subsubsection{UC06 - Visualizzazione Recensioni Ricevute}
\label{UC06}

\begin{figure}[H]
    \centering
    \includesvg[width=1\textwidth]{src/img/UC06.drawio.svg}
    \caption{UC06}
 \end{figure}

\begin{center}
\rowcolors[]{1}{pari}{dispari}
\renewcommand{\arraystretch}{1.5}
\begin{tabular}{ | m{10em} | m{20em} | }
    \hline
    \textbf{Attore primario} & Utente autenticato \\
    \hline
    \textbf{Attore secondario} & \textit{MetaMask} \\
    \hline
    \textbf{Precondizioni} & Nessuna. \\
    \hline
    \textbf{Postcondizioni} & L'utente visualizza le recensioni che ha ricevuto. \\
    \hline
    \textbf{Scenario principale} & Viene mostrata la lista di recensioni ricevute dall'utente. \\
    \hline
   \end{tabular}
\end{center}

\subsubsection{UC07 - Visualizzazione Singola Recensione}
\label{UC07}

\begin{figure}[H]
    \centering
    \includesvg[width=1\textwidth]{src/img/UC07.drawio.svg}
    \caption{UC07}
 \end{figure}

\begin{center}
\rowcolors[]{1}{pari}{dispari}
\renewcommand{\arraystretch}{1.5}
\begin{tabular}{ | m{10em} | m{20em} | }
    \hline
    \textbf{Attore primario} & Utente generico \\
    \hline
    \textbf{Precondizioni} & Si sta visualizzando una lista di recensioni. \\
    \hline
    \textbf{Postcondizioni} & Viene visualizzata una singola recensione della lista. \\
    \hline
    \textbf{Scenario principale} & Viene visualizzata una recensione appartenente alla lista richiesta dall'utente. \\
    \hline
   \end{tabular}
\end{center}

\subsubsection{UC07.1 - Visualizzazione Autore Recensione}
\label{UC07.1}
\begin{center}
\rowcolors[]{1}{pari}{dispari}
\renewcommand{\arraystretch}{1.5}
\begin{tabular}{ | m{10em} | m{20em} | }
    \hline
    \textbf{Attore primario} & Utente generico \\
    \hline
    \textbf{Precondizioni} & Viene visualizzata una recensione. \\
    \hline
    \textbf{Postcondizioni} & Viene visualizzato l'autore della recensione. \\
    \hline
    \textbf{Scenario principale} & Viene visualizzato l'autore all'interno della recensione corrispondente. \\
    \hline
   \end{tabular}
\end{center}

\subsubsection{UC07.2 - Visualizzazione Testo Recensione}
\label{UC07.2}
\begin{center}
\rowcolors[]{1}{pari}{dispari}
\renewcommand{\arraystretch}{1.5}
\begin{tabular}{ | m{10em} | m{20em} | }
    \hline
    \textbf{Attore primario} & Utente generico \\
    \hline
    \textbf{Precondizioni} & Viene visualizzata una recensione. \\
    \hline
    \textbf{Postcondizioni} & Viene visualizzato il testo della recensione. \\
    \hline
    \textbf{Scenario principale} & Viene visualizzato il testo all'interno della recensione corrispondente. \\
    \hline
   \end{tabular}
\end{center}

\subsubsection{UC07.3 - Visualizzazione Voto Recensione}
\label{UC07.3}
\begin{center}
    \rowcolors[]{1}{pari}{dispari}
    \renewcommand{\arraystretch}{1.5}
    \begin{tabular}{ | m{10em} | m{20em} | }
        \hline
        \textbf{Attore primario} & Utente generico \\
        \hline
        \textbf{Precondizioni} & Viene visualizzata una recensione. \\
        \hline
        \textbf{Postcondizioni} & Viene visualizzato il voto della recensione. \\
        \hline
        \textbf{Scenario principale} & Viene visualizzato il voto all'interno della recensione corrispondente. \\
        \hline
       \end{tabular}
    \end{center}

\subsubsection{UC08 - Rilascio Recensione}
\label{UC08}

\begin{figure}[H]
    \centering
    \includesvg[scale=0.9]{src/img/UC08.drawio.svg}
    \caption{UC08}
 \end{figure}

\begin{center}
\rowcolors[]{1}{pari}{dispari}
\renewcommand{\arraystretch}{1.5}
\begin{tabular}{ | m{10em} | m{20em} | }
    \hline
    \textbf{Attore primario} & Utente autenticato \\
    \hline
    \textbf{Attore secondario} & \textit{MetaMask} \\
    \hline
    \textbf{Precondizioni} & \begin{itemize}
        \item L'utente è autenticato;
        \item l'utente ha effettuato almeno un acquisto;
        \item l'utente ha acquisti non recensiti.
    \end{itemize} \\
    \hline
    \textbf{Postcondizioni} & L'utente ha effettuato una recensione sull'acquisto. \\
    \hline
    \textbf{Scenario principale} & \begin{enumerate}
        \item L'utente seleziona un pagamento da associare alla recensione;
        \item l'utente inserisce la recensione.
    \end{enumerate} \\
    \hline
    \textbf{Estensioni} & Se l'utente seleziona una pagamento già recensito, si verifica \hyperref[UC13]{UC13}. \\
    \hline
   \end{tabular}
\end{center}

\subsubsection{UC09 - Pagamento}
\label{UC09}

\begin{figure}[H]
    \centering
    \includesvg[scale=0.9]{src/img/UC09.drawio.svg}
    \caption{UC09}
 \end{figure}

\begin{center}
\rowcolors[]{1}{pari}{dispari}
\renewcommand{\arraystretch}{1.5}
\begin{tabular}{ | m{10em} | m{20em} | }
    \hline
    \textbf{Attore primario} & Utente autenticato \\
    \hline
    \textbf{Attore secondario} & \textit{MetaMask} \\
    \hline
    \textbf{Precondizioni} & L'utente non ha ancora pagato. \\
    \hline
    \textbf{Postcondizioni} & L'utente ha completato il pagamento. \\
    \hline
    \textbf{Scenario principale} & L'utente effettua il pagamento. \\
    \hline
    \textbf{Estensioni} & Se l'utente non ha fondi sufficienti nel \textit{wallet}, si verifica \hyperref[UC14]{UC14}.\\
    \hline
   \end{tabular}
\end{center}

\subsubsection{UC10 - Visualizzazione Errore Autenticazione}
\label{UC10}
\begin{center}
\rowcolors[]{1}{pari}{dispari}
\renewcommand{\arraystretch}{1.5}
\begin{tabular}{ | m{10em} | m{20em} | }
    \hline
    \textbf{Attore primario} & Utente non autenticato \\
    \hline
    \textbf{Attore secondario} & \textit{MetaMask} \\
    \hline
    \textbf{Precondizioni} & L'utente sta tentando di autenticarsi. \\
    \hline
    \textbf{Postcondizioni} & L'operazione fallisce. \\
    \hline
    \textbf{Scenario principale} & \begin{enumerate}
        \item Si verificano problemi con l'accesso a \textit{MetaMask};
        \item viene mostrato un errore che informa l'utente del fallimento dell'operazione;
        \item vengono mostrati dei consigli sulla risoluzione del problema e si invita
              l'utente a riprovare.
    \end{enumerate} \\
    \hline
   \end{tabular}
\end{center}

\subsubsection{UC11 - Errore Ricerca Recensioni}
\label{UC11}
\begin{center}
\rowcolors[]{1}{pari}{dispari}
\renewcommand{\arraystretch}{1.5}
\begin{tabular}{ | m{10em} | m{20em} | }
    \hline
    \textbf{Attori primari} & Utente generico, utente autenticato. \\
    \hline
    \textbf{Attore secondario} & \textit{Infura} \\
    \hline
    \textbf{Precondizioni} & L'utente richiede le recensioni collegate ad un indirizzo. \\
    \hline
    \textbf{Postcondizioni} & La ricerca fallisce. \\
    \hline
    \textbf{Scenario principale} & L'utente inserisce un indirizzo inesistente e viene visualizzato un errore. \\
    \hline
   \end{tabular}
\end{center}

\subsubsection{UC12 - Errore Visualizzazione Recensioni}
\label{UC12}
\begin{center}
\rowcolors[]{1}{pari}{dispari}
\renewcommand{\arraystretch}{1.5}
\begin{tabular}{ | m{10em} | m{20em} | }
    \hline
    \textbf{Attori primari} & Utente generico, utente autenticato. \\
    \hline
    \textbf{Precondizioni} & L'utente ha effettuato una ricerca per indirizzo. \\
    \hline
    \textbf{Postcondizioni} & L'utente non visualizza nessuna recensione. \\
    \hline
    \textbf{Scenario principale} & \begin{enumerate}
        \item L'utente richiede la lista delle recensioni di un determinato indirizzo;
    \item l'indirizzo cercato non presenta recensioni collegate e l'utente viene
          informato di questo.
    \end{enumerate} \\
    \hline
   \end{tabular}
\end{center}

\subsubsection{UC13 - Errore Rilascio Recensione}
\label{UC13}
\begin{center}
\rowcolors[]{1}{pari}{dispari}
\renewcommand{\arraystretch}{1.5}
\begin{tabular}{ | m{10em} | m{20em} | }
    \hline
    \textbf{Attore primario} & Utente autenticato \\
    \hline
    \textbf{Attore secondario} & \textit{MetaMask} \\
    \hline
    \textbf{Precondizioni} & L'utente vuole rilasciare una recensione. \\
    \hline
    \textbf{Postcondizioni} & L'operazione fallisce. \\
    \hline
    \textbf{Scenario principale} & \begin{enumerate}
        \item L'utente inserisce una recensione che non può rilasciare:
              \begin{itemize}
                  \item non ha un acquisto valido legato all'oggetto che vuole recensire;
                  \item la recensione non rispetta un formato valido.
              \end{itemize}
        \item Viene mostrato un errore che informa l'utente sul perché la recensione non può
              essere rilasciata;
        \item vengono mostrati dei consigli sulla risoluzione del problema e si blocca
              l'attuale operazione.
    \end{enumerate} \\
    \hline
   \end{tabular}
\end{center}

\subsubsection{UC14 - Errore Pagamento}
\label{UC14}
\begin{center}
\rowcolors[]{1}{pari}{dispari}
\renewcommand{\arraystretch}{1.5}
\begin{tabular}{ | m{10em} | m{20em} | }
    \hline
    \textbf{Attore primario} & Utente autenticato \\
    \hline
    \textbf{Attore secondario} & \textit{MetaMask} \\
    \hline
    \textbf{Precondizioni} & L'utente sta tentando di effettuare un pagamento. \\
    \hline
    \textbf{Postcondizioni} & Il pagamento fallisce. \\
    \hline
    \textbf{Scenario principale} & \begin{enumerate}
        \item L'utente tenta di effettuare un pagamento che non va a buon fine;
        \item viene mostrato un errore che informa l'utente sul motivo del fallimento
              dell'operazione;
        \item l'operazione viene annullata.
    \end{enumerate} \\
    \hline
   \end{tabular}
\end{center}

\subsubsection{UC15 - Richiesta Lista Recensioni}
\label{UC15}

\begin{figure}[H]
    \centering
    \includesvg[width=1\textwidth]{src/img/UC15.drawio.svg}
    \caption{UC15}
 \end{figure}

\begin{center}
\rowcolors[]{1}{pari}{dispari}
\renewcommand{\arraystretch}{1.5}
\begin{tabular}{ | m{10em} | m{20em} | }
    \hline
    \textbf{Attore primario} & Utilizzatore \textit{API} \\
    \hline
    \textbf{Attore secondario} & \textit{Infura} \\
    \hline
    \textbf{Precondizioni} & L'utente conosce l'indirizzo di cui vuole visualizzare le recensioni. \\
    \hline
    \textbf{Postcondizioni} & L'utente ottiene un file contenente una lista di recensioni. \\
    \hline
    \textbf{Scenario principale} & \begin{enumerate}
        \item L'utente fa una richiesta al \textit{server API REST}\glo\: specificando l'indirizzo di cui
              ottenere le recensioni;
        \item il server recupera le recensioni legate all'utente richiesto;
        \item viene restituita una lista di recensioni;
    \end{enumerate} \\
    \hline
    \textbf{Estensioni} & Se la richiesta fallisce, si verifica \hyperref[UC16]{UC16}. \\
    \hline
   \end{tabular}
\end{center}

\subsubsection{UC16 - Errore Ottenimento Recensioni}
\label{UC16}
\begin{center}
\rowcolors[]{1}{pari}{dispari}
\renewcommand{\arraystretch}{1.5}
\begin{tabular}{ | m{10em} | m{20em} | }
    \hline
    \textbf{Attore primario} & Utilizzatore \textit{API} \\
    \hline
    \textbf{Attore secondario} & \textit{Infura} \\
    \hline
    \textbf{Precondizioni} & L'utente sta tentando di ottenere la lista recensioni. \\
    \hline
    \textbf{Postcondizioni} & L'operazione fallisce. \\
    \hline
    \textbf{Scenario principale} & \begin{enumerate}
        \item L'utente tenta di ottenere la lista di recensioni;
        \item viene mostrato un errore che informa l'utente sul motivo del fallimento
              dell'operazione.
    \end{enumerate} \\
    \hline
   \end{tabular}
\end{center}