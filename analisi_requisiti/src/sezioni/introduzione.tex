\section{Introduzione}


\subsection{Scopo del documento}
Il seguente documento ha l'obiettivo di definire i requisiti di progetto. Al suo interno sono presenti i requisiti
identificati dal gruppo \groupName\: per lo svolgimento del lavoro richiesto.

I requisiti sono stati individuati a partire dal capitolato Trustify
e dalle comunicazioni ricevute direttamente dal proponente.

\subsection{Scopo del prodotto}
Il prodotto ha lo scopo di garantire garantire l'autenticità
delle recensioni rilasciate dagli utenti a una attività o servizio di e-commerce. L'interfaccia della web-app sarà
sviluppata con Angular, mentre l'affidabilità di recensioni e pagamenti sarà garantita utilizzando il linguaggio Solidity su Blockchain Ethereum.


\subsection{Glossario}
Al fine di evitare ambiguità nella terminologia usata all'interno del seguente
documento è stato redatto un glossario, in cui vengono riportate le definizioni
di termini tecnici, rilevanti o con un significato particolare.

Per indicare
la presenza di un termine all'interno del glossario si è scelto di
contrassegnarlo con \glo. Per non appesantire la lettura della documentazione
verrà così contrassegnata solo la prima occorrenza di ogni termine in ciascun
documento.

\subsection{Riferimenti}
    \subsubsection{Riferimenti normativi}
        \begin{itemize}
            \item \textbf{Norme di progetto}
            \item \textbf{Capitolato d'appalto C7}: Trustify - Authentic and verifiable reviews platform \\
            \url{https://www.math.unipd.it/~tullio/IS-1/2022/Progetto/C7.pdf}
            \item \textbf{Verbali esterni}:
            \begin{itemize}
                \item Verbale esterno 2022-10-24;
                \item Verbale esterno 2022-12-14;
            \end{itemize}
        \end{itemize}
    \subsubsection{Riferimenti informativi}
    \begin{itemize}
        \item \textbf{Capitolato d'appalto C7}: Trustify - Authentic and verifiable reviews platform \\
        \url{https://www.math.unipd.it/~tullio/IS-1/2022/Progetto/C7.pdf}
        \item \textbf{Slide T06 del corso di Ingegneria del Software: Analisi dei Requisiti} \\ \url{https://www.math.unipd.it/~tullio/IS-1/2022/Dispense/T06.pdf}
        \item \textbf{Ian Sommerville, Software Engineering (Ninth Edition)}: Chapter 4, Requirements Engineering
    \end{itemize}