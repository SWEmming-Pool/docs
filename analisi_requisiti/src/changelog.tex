\section*{Diario delle modifiche}
\renewcommand{\arraystretch}{1.5}
\rowcolors{2}{pari_change}{dispari_change}
	\begin{longtable}{
			>{\centering}p{0.09\textwidth}
			>{\centering}p{0.13\textwidth}
			>{\centering}p{0.2\textwidth}
			>{\centering}p{0.16\textwidth}
			>{}p{0.2775\textwidth} }

		\rowcolorhead
		\textbf{\color{white}Versione} &
		\textbf{\color{white}Data} &
		\textbf{\color{white}Nominativo} &
		\textbf{\color{white}Ruolo} &
		\centering \textbf{\color{white}Descrizione}
		\tabularnewline
		\endfirsthead
		\rowcolorhead
		\textbf{\color{white}Versione} &
		\textbf{\color{white}Data} &
		\textbf{\color{white}Nominativo} &
		\textbf{\color{white}Ruolo} &
		\centering \textbf{\color{white}Descrizione}
		\tabularnewline
		\endhead

		0.2.3 & 2022-01-16 & Ennio Italiano &
		 &
		Aggiunte sezioni di tracciamento e riepilogo.
		\tabularnewline

		0.2.2 & 2022-01-16 & Ennio Italiano &
		 &
		Aggiunti riferimenti glossario.
		\tabularnewline

		0.2.1 & 2022-01-09 & Ennio Italiano &
		 &
		Aggiunti diagrammi per ogni caso d'uso.
		\tabularnewline

		0.2.0 & 2022-01-08 & Ennio Italiano &
		 &
		Aggiunti riferimenti informativi e normativi, vincoli generali e descrizione attori. Casi d'uso in forma tabellare.
		\tabularnewline

        0.1.0 & 2022-11-25 & Sebastiano Sanson &
		 &
		Revisione casi d'uso e requisiti.
		\tabularnewline

        0.0.6 & 2022-11-24 & Ennio Italiano, Elia Pasquali &
		 &
		Stesura iniziale casi d'uso.
		\tabularnewline

        0.0.5 & 2022-11-23 & Fabio Pantaleo, Nicolò Trinca &
		 &
		Stesura requisiti di vincolo.
		\tabularnewline

        0.0.4 & 2022-11-22 & Fabio Pantaleo, Nicolò Trinca &
		 &
		Stesura requisiti di qualità.
		\tabularnewline

        0.0.3 & 2022-11-21 & Fabio Pantaleo, Nicolò Trinca &
		 &
		Stesura requisiti funzionali.
		\tabularnewline

		0.0.2 & 2022-11-20 & Fabio Pantaleo, Nicolò Trinca &
		 &
		Stesura introduzione.
		\tabularnewline

		0.0.1 & 2022-11-18 & Fabio Pantaleo, Nicolò Trinca &
		 &
		Creata struttura del documento in \LaTeX{}.
		\tabularnewline


		%\end{tabularx}

	\end{longtable}
\renewcommand{\arraystretch}{1}
