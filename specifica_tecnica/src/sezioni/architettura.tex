\section{Architettura del prodotto}

\subsection{Architettura logica}
Il sistema \capName\ consiste di uno smart contract per gestire i pagamenti e le recensioni, e di una web app che permette l'interazione con esso tramite il wallet MetaMask. È anche presente un servizio API REST che permette agli e-commerce interessati di accedere alle recensioni per facilitarne l'integrazione nel loro sito web.

La web app interagisce direttamente con la blockchain tramite Remote Procedure Call utilizzando un nodo per dialogare con lo smart contract. D'altra parte, il servizio API è separato dalla logica dell'applicazione e ha come unico scopo quello di fornire un'interfaccia standard e facile da usare per gli e-commerce, che permette loro di visualizzare le recensioni sul proprio sito.

\subsection{Architettura di deployment}
L'architettura di deployment del prodotto prevede la suddivisione dei componenti in tre nodi distinti:
\begin{itemize}
    \item Server web app: ospita la webapp e comunica con il contratto smart attraverso l'utilizzo della libreria web3js;
    \item Server API: ospita l'API REST e comunica con il contratto smart attraverso l'utilizzo della libreria web3j;
    \item Nodo RPC: nodo per dialogare direttamente com la blockchain Ethereum tramite Remote Procedure Call.
\end{itemize}

\subsection{Diagrammi delle classi}
\begin{figure}[H]
    \includesvg[width=\linewidth]{src/img/webapp.svg}
    \caption{Diagramma delle classi della web app}\label{fig:webapp}
\end{figure}

\begin{figure}[H]
    \includesvg[width=\linewidth]{src/img/contract.svg}
    \caption{Diagramma delle classi dello smart contract}\label{fig:contract}
\end{figure}

\begin{figure}[H]
    \includesvg[width=\linewidth]{src/img/apirest.svg}
    \caption{Diagramma delle classi delle API REST}\label{fig:apirest}
\end{figure}

\subsection{Design patterns}

\subsubsection{Architetturali}
- MVC
- Dependency Injection

\subsubsection{Altri patterns}
- Repository
