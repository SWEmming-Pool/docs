\section{Architettura del prodotto}

\subsection{Architettura logica}
Il sistema \capName\ consiste di uno \textit{smart contract} per gestire i pagamenti e le recensioni, e di una \textit{web app} che permette l'interazione con esso tramite il \textit{wallet} \textit{MetaMask}. È anche presente un servizio \textit{API REST} che permette agli \textit{e-commerce} interessati di accedere alle recensioni per facilitarne l'integrazione nel loro sito web.

La web app interagisce direttamente con la \textit{blockchain} tramite \textit{Remote Procedure Call} utilizzando un nodo per dialogare con lo smart contract. D'altra parte, il servizio \textit{API REST} è separato dalla logica dell'applicazione e ha come unico scopo quello di fornire un'interfaccia standard e facile da usare per gli \textit{e-commerce}, che permette loro di visualizzare le recensioni sul proprio sito.

\subsection{Architettura di deployment}
L'architettura di \textit{deployment} del prodotto prevede la suddivisione dei componenti in tre nodi distinti:
\begin{itemize}
    \item \textit{Server web app}: ospita la \textit{web app} e comunica con lo \textit{smart contract} attraverso l'utilizzo della libreria \texttt{web3js};
    \item \textit{Server API}: ospita l'\textit{API REST} e comunica con il \textit{smart contract} attraverso l'utilizzo della libreria \texttt{web3j};
    \item \textit{Nodo RPC}: nodo per dialogare direttamente com la blockchain Ethereum tramite \textit{Remote Procedure Call}.
\end{itemize}
\pagebreak

\subsection{Dettaglio architettura}
\subsubsection{Smart contract}
\paragraph{Diagramma delle classi}~
\begin{figure}[H]
    \includesvg[width=\linewidth]{src/img/contract.svg}
    \caption{Diagramma delle classi dello \textit{smart contract}}\label{fig:contract}
\end{figure}
\paragraph{Design patterns}

\subsubsection{Web app}
\paragraph{Diagramma delle classi}~
\begin{figure}[H]
    \includesvg[width=\linewidth]{src/img/webapp.svg}
    \caption{Diagramma delle classi della \textit{web app}}\label{fig:webapp}
\end{figure}
\paragraph{Design patterns}~

\noindent Per la \textit{web app} si utilizzano i pattern \textit{MVVM} (\textit{Model-View-ViewModel}) e \textit{Dependency Injection}, per garantire un alto livello di separazione delle responsabilità e una maggiore modularità. Tali pattern costituiscono caratteristiche intrinseche di \textit{Angular}, il \textit{framework} utilizzato per lo sviluppo della \textit{web app}.

\subparagraph{MVVM}~

\noindent Il pattern \textit{MVVM} divide il sistema in tre parti principali: il \textit{Model}, la \textit{View} e il \textit{ViewModel}. All'interno della \textit{web app}, il \textit{Model} è costituito dalle rappresentazioni di recensioni e transazioni, il \textit{ViewModel} è costituito dai servizi che gestiscono le interazioni con lo smart contract e l'autenticazione, mentre la \textit{View} è costituita dai componenti che compongono l'interfaccia grafica.

\subparagraph{Dependency Injection}~

\noindent In \textit{Angular}, il pattern \textit{Dependency Injection} viene applicato per fornire un meccanismo di interazione tra i consumatori e i fornitori di dipendenze mediante un'astrazione chiamata \textit{Injector}. In pratica, un \textit{injector} viene utilizzato per cercare se esiste già un'istanza disponibile di una determinata dipendenza richiesta; se l'istanza non è già presente, viene creata e salvata. Un esempio di utilizzo di \textit{Dependency Injection} si ha nel nostro caso con il servizio di \textit{Angular} \textit{Router}, che viene iniettato nei componenti che lo necessitano per gestire la navigazione tra le varie pagine della \textit{web app}.

\subsubsection{API REST}
\paragraph{Diagramma delle classi}~
\begin{figure}[H]
    \includesvg[width=\linewidth]{src/img/apirest.svg}
    \caption{Diagramma delle classi delle \textit{API REST}}\label{fig:apirest}
\end{figure}

\paragraph{Design patterns}~

\noindent Il pattern architetturale \textit{MVC (Model-View-Controller)} e la \textit{Dependency Injection (DI)} sono due concetti chiave utilizzati nello sviluppo di una \textit{API REST} in Java con \textit{Spring Boot}.

\subparagraph{MVC}~

\noindent Il pattern MVC divide l'applicazione in tre componenti distinti: il \textit{Model}, la \textit{View} e il \textit{Controller}. Il Model rappresenta la logica di business dell'applicazione e contiene la logica per accedere ai dati, il Controller gestisce le richieste HTTP e il flusso di controllo dell'applicazione, mentre la View si occupa di presentare i dati all'utente.

\subparagraph{Dependency Injection}~

\noindent In Spring Boot, la Dependency Injection è implementata attraverso l'uso di un container di inversione di controllo (IoC), che si occupa di gestire la creazione e l'inizializzazione dei componenti dell'applicazione e di fornire le dipendenze tra essi. In particolare, Spring Boot utilizza l'annotazione \texttt{@Autowired} per iniettare automaticamente le dipendenze dei componenti, come ad esempio i servizi che gestiscono l'accesso ai dati, all'interno dei Controller.

In pratica, questo significa che è possibile definire un Controller che dipende da un servizio di accesso ai dati, e automaticamente Spring si occuperà di creare e iniettare l'istanza del servizio all'interno di esso, senza che sia necessario gestire manualmente la creazione e l'inizializzazione delle dipendenze.

% TO DO: aggiungere altri patterns
% Repository
% DTO
% Guard Check
% Guard-are il doc di winning
% Viva la blockchain
\pagebreak